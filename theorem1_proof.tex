%% LyX 2.1.3 created this file.  For more info, see http://www.lyx.org/.
%% Do not edit unless you really know what you are doing.
\documentclass[english]{article}
\usepackage[T1]{fontenc}
\usepackage[latin9]{inputenc}
\usepackage[landscape]{geometry}
\geometry{verbose,tmargin=3cm,bmargin=3cm,lmargin=3cm,rmargin=3cm}
\usepackage{amsbsy}
\usepackage{amssymb}
\usepackage{esint}
\usepackage{babel}
\begin{document}

\title{Training/Validation Split Theorem}

\maketitle
We are interested in bounding the error of the selected model when
tuning penalty parameters by a training validation split. We will
concern ourselves with the error over the observed covariates in the
validation set. Under sufficient entropy conditions, the error of
the selected model will converge to the error of the oracle. 

We will suppose that the data is generated from the model:
\[
y=g^{*}(x)+\epsilon
\]


where $\epsilon$ are independent, sub-gaussian errors. The penalized
regression models are 
\[
\hat{g}(\cdot|\boldsymbol{\lambda})=\arg\min_{g\in\mathcal{G}}L_{T}(g|\boldsymbol{\lambda})
\]


Let the model class after fitting on the training data be 
\[
\mathcal{G}(T)=\left\{ \hat{g}(\cdot|\boldsymbol{\lambda}):\boldsymbol{\lambda}\in\Lambda\right\} 
\]


The selected penalty parameters are
\[
\hat{\boldsymbol{\lambda}}=\arg\min_{\boldsymbol{\lambda}\in\Lambda}\left\Vert y-\hat{g}(\cdot|\boldsymbol{\lambda})\right\Vert _{V}^{2}
\]


Suppose the ``oracle'' penalty parameters are 
\[
\tilde{\boldsymbol{\lambda}}=\arg\min_{\boldsymbol{\lambda}\in\Lambda}\|\hat{g}(\cdot|\boldsymbol{\lambda})-g^{*}\|_{V}
\]


We will provide sharp oracle inequalities of the form
\[
\left\Vert \hat{g}(\cdot|\hat{\boldsymbol{\lambda}})-g^{*}\right\Vert _{V}^{2}\le\left\Vert \hat{g}(\cdot|\tilde{\boldsymbol{\lambda}})-g^{*}\right\Vert _{V}^{2}+\delta
\]


The document is organized as follows
\begin{enumerate}
\item Theorem 3 proves the convergence of the selected model to the best
model under general entropy conditions where the convergence rate
$\delta$ is given as a random variable.
\item Theorem 1 applies Theorem 3 to the special case when the fitted functions
are Lipschitz in the penalty parameters
\item Lemma 1 applies Theorem 1 to the special case where $\lambda_{min}$
and $\lambda_{max}$ are polynomial in the dataset size.
\item Understand the behavior of the oracle error $\min_{\lambda\in\Lambda}\left\Vert \hat{g}(\cdot|\boldsymbol{\lambda})-g^{*}\right\Vert _{V}$.
\item An extension of Theorem 3 where the convergence rate $\delta$ does
not have a random lower bound.
\end{enumerate}

\section{Theorem 3}

Suppose there exists some $r>0$ such that the entropy of $\mathcal{G}(T)$
for any training dataset $T$ such that $\|\epsilon\|_{T}\le2\sigma$
satisfies 
\[
\sup_{T:\|\epsilon\|_{T}\le2\sigma}\int_{0}^{R}H{}^{1/2}(u,\mathcal{G}(T),\|\cdot\|_{V})du\le\psi_{\sigma}(R)\mbox{ }\forall R>r
\]


and

\[
\frac{\psi_{\sigma}\left(u\right)}{u^{2}}
\]


is nonincreasing wrt to $u$ for all $u>r$.

Then there is some constant $c>0$ (only dependent on the sub-gassian
param) such that for all $\delta>r$ satisfying
\[
\sqrt{n_{V}}\delta^{2}\ge c\left(\psi_{\sigma}\left(\delta\right)\vee\delta\vee\psi_{\sigma}\left(4\left\Vert \hat{g}(\cdot|\tilde{\boldsymbol{\lambda}})-g^{*}\right\Vert _{V}\right)\vee4\left\Vert \hat{g}(\cdot|\tilde{\boldsymbol{\lambda}})-g^{*}\right\Vert _{V}\right)
\]


we have
\begin{eqnarray}
Pr\left(\left\Vert \hat{g}(\cdot|\hat{\boldsymbol{\lambda}})-g^{*}\right\Vert _{V}^{2}-\left\Vert \hat{g}(\cdot|\tilde{\boldsymbol{\lambda}})-g^{*}\right\Vert _{V}^{2}\ge\delta^{2}\right) & \le & c\exp\left(-\frac{n_{V}\delta^{4}}{c^{2}\left\Vert \hat{g}(\cdot|\tilde{\boldsymbol{\lambda}})-g^{*}\right\Vert _{V}^{2}}\right)+c\exp\left(-\frac{n_{V}\delta^{2}}{c^{2}}\right)+c\exp\left(-\frac{n_{T}\sigma^{2}}{c^{2}}\right)
\end{eqnarray}



\subsubsection*{Proof}

We use the usual peeling argument
\begin{eqnarray*}
 &  & Pr\left(\left.\left\Vert \hat{g}(\cdot|\hat{\boldsymbol{\lambda}})-g^{*}\right\Vert _{V}^{2}-\left\Vert \hat{g}(\cdot|\tilde{\boldsymbol{\lambda}})-g^{*}\right\Vert _{V}^{2}\ge\delta^{2}\right|T,\|\epsilon\|_{T}\le2\sigma\right)\\
 & = & \sum_{s=0}^{\infty}Pr\left(\left.2^{2s}\delta^{2}\le\left\Vert \hat{g}(\cdot|\hat{\boldsymbol{\lambda}})-g^{*}\right\Vert _{V}^{2}-\left\Vert \hat{g}(\cdot|\tilde{\boldsymbol{\lambda}})-g^{*}\right\Vert _{V}^{2}\le2^{2s+2}\delta^{2}\right|T,\|\epsilon\|_{T}\le2\sigma\right)\\
 & \le & \sum_{s=0}^{\infty}Pr\left(\left.2^{2s}\delta^{2}\le2\left\langle \epsilon,\hat{g}(\cdot|\hat{\boldsymbol{\lambda}})-\hat{g}(\cdot|\tilde{\boldsymbol{\lambda}})\right\rangle _{V}\wedge\left\Vert \hat{g}(\cdot|\hat{\boldsymbol{\lambda}})-g^{*}\right\Vert _{V}^{2}-\left\Vert \hat{g}(\cdot|\tilde{\boldsymbol{\lambda}})-g^{*}\right\Vert _{V}^{2}\le2^{2s+2}\delta^{2}\right|T,\|\epsilon\|_{T}\le2\sigma\right)\\
 & = & \sum_{s=0}^{\infty}Pr\left(\left.2^{2s}\delta^{2}\le2\left\langle \epsilon,\hat{g}(\cdot|\hat{\boldsymbol{\lambda}})-\hat{g}(\cdot|\tilde{\boldsymbol{\lambda}})\right\rangle _{V}\wedge\left\Vert \hat{g}(\cdot|\hat{\boldsymbol{\lambda}})-\hat{g}(\cdot|\tilde{\boldsymbol{\lambda}})\right\Vert _{V}^{2}+2\left\langle \hat{g}(\cdot|\tilde{\boldsymbol{\lambda}})-\hat{g}(\cdot|\hat{\boldsymbol{\lambda}}),\hat{g}(\cdot|\tilde{\boldsymbol{\lambda}})-g^{*}\right\rangle _{V}\le2^{2s+2}\delta^{2}\right|T,\|\epsilon\|_{T}\le2\sigma\right)\\
 & \le & \sum_{s=0}^{\infty}Pr\left(\left.2^{2s}\delta^{2}\le2\left\langle \epsilon,\hat{g}(\cdot|\hat{\boldsymbol{\lambda}})-\hat{g}(\cdot|\tilde{\boldsymbol{\lambda}})\right\rangle _{V}\wedge\left\Vert \hat{g}(\cdot|\hat{\boldsymbol{\lambda}})-\hat{g}(\cdot|\tilde{\boldsymbol{\lambda}})\right\Vert _{V}^{2}\le2^{2s+2}\delta^{2}+2\left|\left\langle \hat{g}(\cdot|\tilde{\boldsymbol{\lambda}})-\hat{g}(\cdot|\hat{\boldsymbol{\lambda}}),\hat{g}(\cdot|\tilde{\boldsymbol{\lambda}})-g^{*}\right\rangle _{V}\right|\right|T,\|\epsilon\|_{T}\le2\sigma\right)
\end{eqnarray*}


We can split each probability into the case where $2^{2s+2}\delta^{2}$
or $2\left|\left\langle \hat{g}(\cdot|\tilde{\boldsymbol{\lambda}})-\hat{g}(\cdot|\hat{\boldsymbol{\lambda}}),\hat{g}(\cdot|\tilde{\boldsymbol{\lambda}})-g^{*}\right\rangle _{V}\right|$
is bigger: 
\begin{eqnarray*}
 &  & Pr\left(\left.2^{2s}\delta^{2}\le2\left\langle \epsilon,\hat{g}(\cdot|\hat{\boldsymbol{\lambda}})-\hat{g}(\cdot|\tilde{\boldsymbol{\lambda}})\right\rangle _{V}\wedge\left\Vert \hat{g}(\cdot|\hat{\boldsymbol{\lambda}})-\hat{g}(\cdot|\tilde{\boldsymbol{\lambda}})\right\Vert _{V}^{2}\le2^{2s+2}\delta^{2}+2\left|\left\langle \hat{g}(\cdot|\tilde{\boldsymbol{\lambda}})-\hat{g}(\cdot|\hat{\boldsymbol{\lambda}}),\hat{g}(\cdot|\tilde{\boldsymbol{\lambda}})-g^{*}\right\rangle _{V}\right|\right|T,\|\epsilon\|_{T}\le2\sigma\right)\\
 & \le & Pr\left(\left.2^{2s}\delta^{2}\le2\left\langle \epsilon,\hat{g}(\cdot|\hat{\boldsymbol{\lambda}})-\hat{g}(\cdot|\tilde{\boldsymbol{\lambda}})\right\rangle _{V}\wedge\left\Vert \hat{g}(\cdot|\hat{\boldsymbol{\lambda}})-\hat{g}(\cdot|\tilde{\boldsymbol{\lambda}})\right\Vert _{V}^{2}\le4\left\Vert \hat{g}(\cdot|\tilde{\boldsymbol{\lambda}})-g^{*}\right\Vert _{V}\left\Vert \hat{g}(\cdot|\hat{\boldsymbol{\lambda}})-\hat{g}(\cdot|\tilde{\boldsymbol{\lambda}})\right\Vert _{V}\right|T,\|\epsilon\|_{T}\le2\sigma\right)\\
 &  & +Pr\left(\left.2^{2s}\delta^{2}\le2\left\langle \epsilon,\hat{g}(\cdot|\hat{\boldsymbol{\lambda}})-\hat{g}(\cdot|\tilde{\boldsymbol{\lambda}})\right\rangle _{V}\wedge\left\Vert \hat{g}(\cdot|\hat{\boldsymbol{\lambda}})-\hat{g}(\cdot|\tilde{\boldsymbol{\lambda}})\right\Vert _{V}^{2}\le2^{2s+3}\delta^{2}\right|T,\|\epsilon\|_{T}\le2\sigma\right)\\
 & \le & Pr\left(\left.\sup_{\left\Vert \hat{g}(\cdot|\hat{\boldsymbol{\lambda}})-\hat{g}(\cdot|\tilde{\boldsymbol{\lambda}})\right\Vert _{V}\le4\left\Vert \hat{g}(\cdot|\tilde{\boldsymbol{\lambda}})-g^{*}\right\Vert _{V}}2^{2s}\delta^{2}\le2\left\langle \epsilon,\hat{g}(\cdot|\hat{\boldsymbol{\lambda}})-\hat{g}(\cdot|\tilde{\boldsymbol{\lambda}})\right\rangle _{V}\right|T,\|\epsilon\|_{T}\le2\sigma\right)\\
 &  & +Pr\left(\left.\sup_{\left\Vert \hat{g}(\cdot|\hat{\boldsymbol{\lambda}})-\hat{g}(\cdot|\tilde{\boldsymbol{\lambda}})\right\Vert _{V}\le2^{s+3/2}\delta}2^{2s}\delta^{2}\le2\left\langle \epsilon,\hat{g}(\cdot|\hat{\boldsymbol{\lambda}})-\hat{g}(\cdot|\tilde{\boldsymbol{\lambda}})\right\rangle _{V}\right|T,\|\epsilon\|_{T}\le2\sigma\right)
\end{eqnarray*}


Hence 
\begin{eqnarray}
 &  & Pr\left(\left.\left\Vert \hat{g}(\cdot|\hat{\boldsymbol{\lambda}})-g^{*}\right\Vert _{V}^{2}-\left\Vert \hat{g}(\cdot|\tilde{\boldsymbol{\lambda}})-g^{*}\right\Vert _{V}^{2}\ge\delta^{2}\right|T,\|\epsilon\|_{T}\le2\sigma\right)\nonumber \\
 & \le & Pr\left(\left.\sup_{\left\Vert \hat{g}(\cdot|\hat{\boldsymbol{\lambda}})-\hat{g}(\cdot|\tilde{\boldsymbol{\lambda}})\right\Vert _{V}\le4\left\Vert \hat{g}(\cdot|\tilde{\boldsymbol{\lambda}})-g^{*}\right\Vert _{V}}\delta^{2}\le2\left\langle \epsilon,\hat{g}(\cdot|\hat{\boldsymbol{\lambda}})-\hat{g}(\cdot|\tilde{\boldsymbol{\lambda}})\right\rangle _{V}\right|T,\|\epsilon\|_{T}\le2\sigma\right)\\
 &  & +\sum_{s=0}^{\infty}Pr\left(\left.\sup_{\left\Vert \hat{g}(\cdot|\hat{\boldsymbol{\lambda}})-\hat{g}(\cdot|\tilde{\boldsymbol{\lambda}})\right\Vert _{V}\le2^{s+3/2}\delta}2^{2s}\delta^{2}\le2\left\langle \epsilon,\hat{g}(\cdot|\hat{\boldsymbol{\lambda}})-\hat{g}(\cdot|\tilde{\boldsymbol{\lambda}})\right\rangle _{V}\right|T,\|\epsilon\|_{T}\le2\sigma\right)
\end{eqnarray}


We have the entropy condition that 
\[
\sup_{T:\|\epsilon\|_{T}\le2\sigma}\int_{0}^{R}H{}^{1/2}(u,\mathcal{G}(T),\|\cdot\|_{V})du\le\psi_{\sigma}(R)
\]


By Vandegeer Corollary 8.3, for all $\delta>0$ with

\[
\sqrt{n_{V}}\delta^{2}\ge c\left(\psi_{\sigma}(4\left\Vert \hat{g}(\cdot|\tilde{\boldsymbol{\lambda}})-g^{*}\right\Vert _{V})\vee4\left\Vert \hat{g}(\cdot|\tilde{\boldsymbol{\lambda}})-g^{*}\right\Vert _{V}\right)
\]


we can bound (2) as follows

\[
Pr\left(\left.\sup_{\left\Vert \hat{g}(\cdot|\hat{\boldsymbol{\lambda}})-\hat{g}(\cdot|\tilde{\boldsymbol{\lambda}})\right\Vert _{V}\le4\left\Vert \hat{g}(\cdot|\tilde{\boldsymbol{\lambda}})-g^{*}\right\Vert _{V}}\delta^{2}\le2\left\langle \epsilon,\hat{g}(\cdot|\hat{\boldsymbol{\lambda}})-\hat{g}(\cdot|\tilde{\boldsymbol{\lambda}})\right\rangle _{V}\right|T,\|\epsilon\|_{T}\le2\sigma\right)\le c\exp\left(-\frac{n_{V}\delta^{4}}{4c^{2}\left(16\left\Vert \hat{g}(\cdot|\tilde{\boldsymbol{\lambda}})-g^{*}\right\Vert _{V}^{2}\right)}\right)
\]


We can also bound (3) using Vandegeer Corollary 8.3. First we check
the condition for Corollary 8.3 is satisfied. In particular, we need
to show that for all $s=0,1,2,...$ 
\begin{eqnarray*}
\sqrt{n_{V}}2^{2s+2}\delta^{2} & \ge & c\left(\psi_{\sigma}\left(2^{s+1}\delta\right)\vee\delta\right)
\end{eqnarray*}


This is true since we chose $\delta$ such that 
\begin{eqnarray*}
\sqrt{n_{V}}\delta^{2} & \ge & c\left(\psi_{\sigma}\left(\delta\right)\vee\delta\right)
\end{eqnarray*}


and we assumed that $\psi_{\sigma}(u)/u^{2}$ is nonincreasing for
all $u$.

So Corollary 8.3 states that for all $s=0,1,...$
\[
Pr\left(\left.\sup_{\lambda,\lambda':\left\Vert \hat{g}(\cdot|\lambda)-\hat{g}(\cdot|\lambda')\right\Vert _{V}\le2^{s+3/2}\delta}\left\langle \epsilon,\hat{g}(\cdot|\boldsymbol{\lambda})-\hat{g}(\cdot|\boldsymbol{\lambda}')\right\rangle _{V}\ge2^{2s-1}\delta^{2}\right|T,\|\epsilon\|_{T}\le2\sigma\right)\le c\exp\left(-n_{V}\frac{2^{4s-2}\delta^{4}}{4c^{2}2^{2s+3}\delta^{2}}\right)
\]


Putting this together, for all $\delta>r$ satisfying 
\[
\sqrt{n_{V}}\delta^{2}\ge a\left(\psi_{\sigma}\left(\delta\right)\vee\delta\vee\psi_{\sigma}\left(4\left\Vert \hat{g}(\cdot|\tilde{\boldsymbol{\lambda}})-g^{*}\right\Vert _{V}\right)\vee4\left\Vert \hat{g}(\cdot|\tilde{\boldsymbol{\lambda}})-g^{*}\right\Vert _{V}\right)
\]


we have
\begin{eqnarray*}
Pr\left(\left.\left\Vert \hat{g}(\cdot|\hat{\boldsymbol{\lambda}})-g^{*}\right\Vert _{V}^{2}-\left\Vert \hat{g}(\cdot|\tilde{\boldsymbol{\lambda}})-g^{*}\right\Vert _{V}^{2}\ge\delta^{2}\right|T,\|\epsilon\|_{T}\le2\sigma\right) & \le & c\exp\left(-\frac{n_{V}\delta^{4}}{c^{2}\left\Vert \hat{g}(\cdot|\tilde{\boldsymbol{\lambda}})-g^{*}\right\Vert _{V}^{2}}\right)+c\exp\left(-\frac{n_{V}\delta^{2}}{c^{2}}\right)
\end{eqnarray*}


Finally, we note that

\begin{eqnarray*}
Pr\left(\left\Vert \hat{g}(\cdot|\hat{\boldsymbol{\lambda}})-g^{*}\right\Vert _{V}^{2}-\left\Vert \hat{g}(\cdot|\tilde{\boldsymbol{\lambda}})-g^{*}\right\Vert _{V}^{2}\ge\delta^{2}\right) & \le & c\exp\left(-\frac{n_{V}\delta^{4}}{c^{2}\left\Vert \hat{g}(\cdot|\tilde{\boldsymbol{\lambda}})-g^{*}\right\Vert _{V}^{2}}\right)+c\exp\left(-\frac{n_{V}\delta^{2}}{c^{2}}\right)\\
 &  & +Pr\left(\|\epsilon\|_{T}^{2}\ge4\sigma^{2}\right)
\end{eqnarray*}


The last term can easily be bounded using Bernstein's inequality since
$\epsilon$ are uniformly sub-gaussian random variables
\[
Pr\left(\|\epsilon\|_{T}^{2}\ge4\sigma^{2}\right)\le c\exp\left(-\frac{n_{T}\sigma^{2}}{c^{2}}\right)
\]



\section{Theorem 1}

Let $\Lambda=[\lambda_{min},\lambda_{max}]^{J}$.

Suppose there is a constant $C>0$ such that for all $\boldsymbol{\lambda}_{1},\boldsymbol{\lambda}_{2}\in\Lambda$
\[
\sup_{T:\|\epsilon\|_{T}\le2\sigma}\left\Vert \hat{g}(\cdot|\boldsymbol{\lambda}^{(1)})-\hat{g}(\cdot|\boldsymbol{\lambda}^{(2)})\right\Vert _{V}\le C\|\boldsymbol{\lambda}_{1}-\boldsymbol{\lambda}_{2}\|
\]


Then there is a constant $c>0$ only depends on the characteristics
of the sub-gassian errors such that for any $\delta>0$ satisfying

\[
\delta^{2}\ge c\max\left\{ \frac{\alpha_{n}^{2}}{n_{V}},\frac{\alpha_{n}}{\sqrt{n_{V}}}\left\Vert \hat{g}(\cdot|\tilde{\boldsymbol{\lambda}})-g^{*}\right\Vert _{V}\right\} 
\]


where

\[
\alpha_{n}=\sqrt{J\left(1+\log\left(32Cn\left(\lambda_{max}-\lambda_{min}\right)\right)\right)}\vee1
\]


we have

\begin{eqnarray*}
Pr\left(\left\Vert \hat{g}(\cdot|\hat{\boldsymbol{\lambda}})-g^{*}\right\Vert _{V}^{2}-\left\Vert \hat{g}(\cdot|\tilde{\boldsymbol{\lambda}})-g^{*}\right\Vert _{V}^{2}\ge\delta^{2}\right) & \le & c\exp\left(-\frac{n_{V}\delta^{4}}{c^{2}\left\Vert \hat{g}(\cdot|\tilde{\boldsymbol{\lambda}})-g^{*}\right\Vert _{V}^{2}}\right)+c\exp\left(-\frac{n_{V}\delta^{2}}{c^{2}}\right)+c\exp\left(-\frac{n_{T}\sigma^{2}}{c^{2}}\right)
\end{eqnarray*}



\subsubsection*{Proof}

\textbf{1. Determine entropy bound and properties}

Under the given Lipschitz condition, a $\delta$-cover for $\Lambda$
is a $C\delta$-cover for $\mathcal{G}(T)$. We can therefore calculate
a covering number for $\mathcal{G}(T)$ wrt $\|\cdot\|_{V}$ by using
the covering number for $\Lambda$. 
\[
N\left(u,\mathcal{G}(T),\|\cdot\|_{V}\right)\le N\left(\frac{u}{C},\Lambda,\|\cdot\|_{2}\right)
\]


By Lemma Cube Covering Number (See Appendix below), we know that
\begin{eqnarray*}
N\left(u,\Lambda,\|\cdot\|_{2}\right) & \le & \frac{1}{C_{J}}\left(\frac{4\left(\lambda_{max}-\lambda_{min}\right)+2\frac{u}{C}}{\frac{u}{C}}\right)^{J}\\
 & = & \frac{1}{C_{J}}\left(\frac{4\left(\lambda_{max}-\lambda_{min}\right)C+2u}{u}\right)^{J}\\
 & \le & \frac{1}{C_{J}}\left(\frac{4C\left(\lambda_{max}-\lambda_{min}\right)+2u}{u}\right)^{J}
\end{eqnarray*}


Hence
\[
H(u,\mathcal{G}(T),\|\cdot\|_{V})\le\log\left[\frac{1}{C_{J}}\left(\frac{4C\left(\lambda_{max}-\lambda_{min}\right)+2u}{u}\right)^{J}\right]
\]


Then

\begin{eqnarray*}
\int_{0}^{R}H{}^{1/2}(u,\mathcal{G}(T),\|\cdot\|_{V})du & \le & \int_{0}^{R}\left[\log\frac{1}{C_{J}}+J\log\left(\frac{4C\left(\lambda_{max}-\lambda_{min}\right)+2u}{u}\right)\right]^{1/2}du\\
 & \le & \int_{0}^{R}\left[\log\frac{1}{C_{J}}+J\log4+J\log\left(\frac{8C\left(\lambda_{max}-\lambda_{min}\right)}{u}\right)\right]^{1/2}du\\
 & = & R\int_{0}^{1}\left[\log\frac{1}{C_{J}}+J\log4+J\log\left(\frac{8C\left(\lambda_{max}-\lambda_{min}\right)}{Rv}\right)\right]^{1/2}dv\\
 & \le & R\left[\int_{0}^{1}\left(\log\frac{1}{C_{J}}+J\log4+J\log\left(\frac{8C\left(\lambda_{max}-\lambda_{min}\right)}{R}\right)+J\log\frac{1}{v}\right)dv\right]^{1/2}\\
 & = & R\left[\log\frac{1}{C_{J}}+J(1+\log4)+J\log\left(8C\left(\lambda_{max}-\lambda_{min}\right)\right)+J\log\frac{1}{R}\right]^{1/2}
\end{eqnarray*}


The second bound is crazy loose but I think it is okay. It comes from
the fact that 
\[
\log\left(a+b\right)<\log\left(2a\right)+\log(2b)
\]


The third inequality follows from concavity of the square root.

We will consider the following bound for $\int_{0}^{R}H{}^{1/2}(u,\mathcal{G}(T),\|\cdot\|_{V})du$
for all $R\ge n^{-1}$

\[
\int_{0}^{R}H{}^{1/2}(u,\mathcal{G}(T),\|\cdot\|_{V})du\le\psi_{\sigma}(R)=R\left(J\left(1+\log\left(32C\left(\lambda_{max}-\lambda_{min}\right)n\right)\right)\right)^{1/2}
\]


Notice we've replaced the last term $\log\frac{1}{R}$ with $\log n$,
which is valid over the given range. We will see this is useful since
solving for $\delta$ is hard with the $\log\frac{1}{R}$ term.

Also, for simplicity, we dropped $\log\frac{1}{C_{J}}$ since $C_{J}>1$
for all $J=1,2,...$ 

\textbf{2. Apply Theorem 3}

Now we apply Theorem 3 to determine $\delta$ such that (1) holds.
Theorem 3 states that we need $\delta>n^{-1}$ to satisfy 
\[
\sqrt{n_{V}}\delta^{2}\ge c\left(\psi_{\sigma}\left(\delta\right)\vee\delta\vee\psi_{\sigma}\left(4\left\Vert \hat{g}(\cdot|\tilde{\boldsymbol{\lambda}})-g^{*}\right\Vert _{V}\right)\vee4\left\Vert \hat{g}(\cdot|\tilde{\boldsymbol{\lambda}})-g^{*}\right\Vert _{V}\right)
\]


where $a$ is a constant only dependent on characteristics of the
sub-gaussian random variables.

We will solve this by splitting into cases. 

\textbf{Case 1: }Suppose that 
\[
\psi_{\sigma}\left(\delta\right)\vee\delta\ge\psi_{\sigma}\left(4\left\Vert \hat{g}(\cdot|\tilde{\boldsymbol{\lambda}})-g^{*}\right\Vert _{V}\right)\vee4\left\Vert \hat{g}(\cdot|\tilde{\boldsymbol{\lambda}})-g^{*}\right\Vert _{V}
\]


In this case, we must have
\[
\sqrt{n_{V}}\delta^{2}\ge c\delta\left(\left(J\left(1+\log\left(32C\left(\lambda_{max}-\lambda_{min}\right)n\right)\right)\right)^{1/2}\vee1\right)
\]


So

\[
\delta\ge c\frac{1}{\sqrt{n_{V}}}\left(\left(J\left(1+\log\left(32C\left(\lambda_{max}-\lambda_{min}\right)n\right)\right)\right)^{1/2}\vee1\right)
\]


\textbf{Case 2:}

Suppose that 
\[
\psi_{\sigma}\left(\delta\right)\vee\delta\le\psi_{\sigma}\left(4\left\Vert \hat{g}(\cdot|\tilde{\boldsymbol{\lambda}})-g^{*}\right\Vert _{V}\right)\vee4\left\Vert \hat{g}(\cdot|\tilde{\boldsymbol{\lambda}})-g^{*}\right\Vert _{V}
\]


In this case, we must have

\[
\sqrt{n_{V}}\delta^{2}\ge4c\left\Vert \hat{g}\left(\cdot|\tilde{\boldsymbol{\lambda}}\right)-g^{*}\right\Vert _{V}\left(\left(J\left(1+\log\left(32C\left(\lambda_{max}-\lambda_{min}\right)n\right)\right)\right)^{1/2}\vee1\right)
\]


Putting these two inequalities together, we find that $\delta>0$
must satisfy
\[
\delta^{2}\ge c\max\left(\frac{\alpha_{n}^{2}}{n_{V}},\frac{\alpha_{n}}{\sqrt{n_{V}}}\left\Vert \hat{g}(\cdot|\tilde{\boldsymbol{\lambda}})-g^{*}\right\Vert _{V},\frac{1}{n^{2}}\right)
\]


where 
\[
\alpha_{n}=\left(J\left(1+\log\left(32C\left(\lambda_{max}-\lambda_{min}\right)n\right)\right)\right)^{1/2}
\]



\section{Lemma 1 with $\lambda$ changing with $n$}

We will express this using asymptotic notation.

Let $\Lambda=[n_{T}^{-t_{min}},n_{T}^{t_{max}}]^{J}$ where $t_{min},t_{max}>0$.

Suppose that if $\|\epsilon\|_{T}\le2\sigma$, there are constants
$C,\kappa$ such that for any $u>0$, we have for all $\lambda\in\Lambda$
\[
\left\Vert \hat{g}(\cdot|\boldsymbol{\lambda}^{(1)})-\hat{g}(\cdot|\boldsymbol{\lambda}^{(2)})\right\Vert _{V}\le Cn_{T}^{\kappa}\|\boldsymbol{\lambda}_{1}-\boldsymbol{\lambda}_{2}\|
\]


Then

\begin{eqnarray*}
\left\Vert \hat{g}(\cdot|\hat{\boldsymbol{\lambda}})-g^{*}\right\Vert _{V}^{2} & \le & \left\Vert \hat{g}(\cdot|\tilde{\boldsymbol{\lambda}})-g^{*}\right\Vert _{V}^{2}+O_{p}\left(\frac{J\left((\kappa+t_{max})\log n_{T}+\log\left(Cn\right)\right)}{n_{V}}\right)+O_{p}\left(\left[\frac{J\left((\kappa+t_{max})\log n_{T}+\log\left(Cn\right)\right)}{n_{V}}\right]^{1/2}\left\Vert \hat{g}(\cdot|\tilde{\boldsymbol{\lambda}})-g^{*}\right\Vert _{V}\right)
\end{eqnarray*}


(Note: $t_{min}$ doesn't appear in the formula, but this is because
it appears in the $\kappa$ term. $\kappa$ is usually a linear combination
of $t_{min}$ and $t_{max}$.)


\section{Understanding the behavior of the oracle error}

All of the oracle inequalities that we have derived use the oracle
error 
\[
\min_{\lambda\in\Lambda}\left\Vert \hat{g}(\cdot|\boldsymbol{\lambda})-g^{*}\right\Vert _{V}
\]
as an upper bound. Intuitively, one would think that the oracle error
is small, but we did not prove this earlier. In Mitchell's paper (CITE
in the real paper), he supposes the generalized loss is small 
\[
\min_{\lambda\in\Lambda}\left\Vert \hat{g}(\cdot|\boldsymbol{\lambda})-g^{*}\right\Vert ^{2}=\min_{\lambda\in\Lambda}\int\left(\hat{g}(x|\boldsymbol{\lambda})-g^{*}(x)\right)^{2}dx
\]


We will use the generalized loss as our start point to show that $\min_{\lambda\in\Lambda}\left\Vert \hat{g}(\cdot|\boldsymbol{\lambda})-g^{*}\right\Vert _{V}$
is small if $\min_{\lambda\in\Lambda}\left\Vert \hat{g}(\cdot|\boldsymbol{\lambda})-g^{*}\right\Vert ^{2}$
is small. To do this, we use Theorem 2.1 in Vandegeer (CITE in the
real paper, On the uniform convergence...). 

Let 
\[
\tilde{\boldsymbol{\lambda}}_{gen}=\arg\min_{\lambda\in\Lambda}\left\Vert \hat{g}(\cdot|\boldsymbol{\lambda})-g^{*}\right\Vert ^{2}
\]


Consider the function class composed of just the single function:
\[
\mathcal{G}(T)=\left\{ \hat{g}\left(\cdot|\tilde{\boldsymbol{\lambda}}_{gen}\right)-g^{*}\right\} 
\]


Since this class contains a single function, its entropy is zero.

Suppose that 
\[
K_{gen}=\|\hat{g}(\cdot|\tilde{\boldsymbol{\lambda}}_{gen})-g^{*}\|_{\infty}
\]


Then by Theorem 2.1, for any $t>0$, we can bound the conditional
probability (training set $T$ is given)
\[
Pr\left(\left.\frac{1}{C_{1}}\left|\left\Vert \hat{g}(\cdot|\tilde{\boldsymbol{\lambda}}_{gen})-g^{*}\right\Vert _{V}^{2}-\left\Vert \hat{g}(\cdot|\tilde{\boldsymbol{\lambda}}_{gen})-g^{*}\right\Vert ^{2}\right|\le\left\Vert \hat{g}(\cdot|\tilde{\boldsymbol{\lambda}}_{gen})-g^{*}\right\Vert K_{gen}\sqrt{\frac{t}{n_{V}}}+\frac{K_{gen}^{2}t}{n_{V}}\right|T\right)\ge1-\exp(-t)
\]


where $C_{1}$ is a constant given in the theorem. 


\subsubsection*{Extension}

Perhaps instead we want to start with knowing that the oracle training
loss is small. Then we need to go from the oracle training loss to
the generalized loss to the validation loss. All these jumps require
empirical process techniques. We can in fact use Theorem 2.1 to make
all these jumps.

Suppose we know that with high probability, for any training dataset,
there is an oracle penalty parameter vector $\tilde{\boldsymbol{\lambda}}_{T}$
such that 
\[
Pr\left(\min_{\lambda\in\Lambda}\left\Vert \hat{g}(\cdot|\boldsymbol{\lambda})-g^{*}\right\Vert _{T}^{2}\le Wn_{T}^{-2\omega}\right)\ge1-p(n_{T})
\]


where $p(n_{T})$ is some small probability tending to zero as $n_{T}\rightarrow\infty$
and constants $W,\omega>0$ only dependent on the model class $\mathcal{G}$

Suppose $\tilde{\boldsymbol{\lambda}}_{T}=\arg\min_{\lambda}\left\Vert \hat{g}(\cdot|\boldsymbol{\lambda})-g^{*}\right\Vert _{T}$.
Let 
\[
K_{\Lambda}=\sup_{\lambda\in\Lambda}\|\hat{g}(\cdot|\boldsymbol{\lambda})-g^{*}\|_{\infty}
\]


Then if $\left\Vert \hat{g}(\cdot|\tilde{\boldsymbol{\lambda}}_{T})-g^{*}\right\Vert _{T}^{2}\le Wn_{T}^{-2\omega}$,
according to Theorem 2.1 in Vandegeer , we have for all $\delta_{t}>0$
satisfying 
\[
\delta_{t}\ge\frac{2J_{\infty}\left(K_{\Lambda},\mathcal{G}\right)+K\sqrt{t}}{\sqrt{n_{T}}}+\frac{4J_{\infty}^{2}(K_{\Lambda},\mathcal{G})+K_{\Lambda}^{2}t}{n_{T}}
\]


the following inequality holds with probability at least $1-\exp(-t)$
\begin{eqnarray*}
\left\Vert \hat{g}(\cdot|\tilde{\boldsymbol{\lambda}}_{T})-g^{*}\right\Vert ^{2} & = & \left\Vert \hat{g}(\cdot|\tilde{\boldsymbol{\lambda}}_{T})-g^{*}\right\Vert ^{2}-\left\Vert \hat{g}(\cdot|\tilde{\boldsymbol{\lambda}}_{T})-g^{*}\right\Vert _{T}^{2}+\left\Vert \hat{g}(\cdot|\tilde{\boldsymbol{\lambda}}_{T})-g^{*}\right\Vert _{T}^{2}\\
 & \le & \delta_{t}\left\Vert \hat{g}(\cdot|\tilde{\boldsymbol{\lambda}}_{T})-g^{*}\right\Vert ^{2}+Wn_{T}^{-2\omega}
\end{eqnarray*}


which is equivalent to saying that for all $0<\delta_{t}<1/2$ satisfying
\[
\delta_{t}\ge\frac{2J_{\infty}\left(K,\mathcal{G}\right)+K_{\Lambda}\sqrt{t}}{\sqrt{n_{T}}}+\frac{4J_{\infty}^{2}(K_{\Lambda},\mathcal{G})+K_{\Lambda}^{2}t}{n}
\]


we have
\[
Pr\left(\left.\left\Vert \hat{g}(\cdot|\tilde{\boldsymbol{\lambda}}_{T})-g^{*}\right\Vert ^{2}\le2Wn_{T}^{-2\omega}\right|\left\Vert \hat{g}(\cdot|\tilde{\boldsymbol{\lambda}}_{T})-g^{*}\right\Vert _{T}^{2}\le Wn_{T}^{-2\omega}\right)\ge1-\exp(-t)
\]


Note that for the models of interest, $J_{\infty}\left(K_{\Lambda}^{2},\mathcal{G}\right)$
usually contains at most $\log n_{T}$ but not a polynomial term in
$n_{T}$.

These results can be used to extend Theorem 3 even further.


\section{Extended Theorem 3}

Let 
\[
K=\sup_{g\in\mathcal{G}}\|g-g^{*}\|_{\infty}
\]


Suppose that there are constants $W,\omega>0$ only dependent on the
model class $\mathcal{G}$ such that 
\[
Pr\left(\min_{\lambda\in\Lambda}\left\Vert \hat{g}(\cdot|\boldsymbol{\lambda})-g^{*}\right\Vert _{T}^{2}\le Wn_{T}^{-2\omega}\right)\ge1-p(n_{T})
\]


where $p(n_{T})$ tends to zero as $n_{T}\rightarrow\infty$.

Choose $t,\tilde{\delta_{t}}>0$ such that

\[
\tilde{\delta}_{t}^{2}\ge C_{1}\left(\sqrt{2W}Kn_{T}^{-\omega}n_{V}^{-1/2}\sqrt{t}+\frac{K^{2}t}{n_{V}}\right)+2Wn_{T}^{-2\omega}
\]


where $C_{1}>0$ is a constant given in Vandegeer Theorem 2.1.

Choose $t_{1}>0$ such that 
\[
\frac{1}{2}\ge\frac{2J_{\infty}\left(K,\mathcal{G}\right)+K\sqrt{t_{1}}}{\sqrt{n_{T}}}+\frac{4J_{\infty}^{2}(K,\mathcal{G})+K^{2}t_{1}}{n_{T}}
\]


Suppose there is an $r>0$ such that for all $R>r$, we have 
\[
\sup_{T:\|\epsilon\|_{T}\le2\sigma}\int_{0}^{R}H{}^{1/2}(u,\mathcal{G}(T),\|\cdot\|_{V})du\le\psi_{\sigma}(R)
\]


Then there is a constant $c>0$ only dependent on the sub-gaussian
parameters such that for all $\delta>r$ satisfying 
\[
\sqrt{n_{V}}\delta^{2}\ge c\left(\psi_{\sigma}(\delta)\vee\delta\vee\psi_{\sigma}(\tilde{\delta}_{t})\vee\tilde{\delta}_{t}\right)
\]


we have
\[
Pr\left(\left\Vert \hat{g}(\cdot|\hat{\boldsymbol{\lambda}})-g^{*}\right\Vert _{V}^{2}-\left[\min_{\lambda\in\Lambda}\left\Vert \hat{g}(\cdot|\boldsymbol{\lambda})-g^{*}\right\Vert _{V}^{2}\right]\ge\delta^{2}\right)\le c\exp\left(-\frac{n_{V}\delta^{4}}{c^{2}\tilde{\delta}_{t}}\right)+c\exp\left(-\frac{n_{V}\delta^{2}}{c^{2}}\right)+c\exp\left(-\frac{n_{T}\sigma^{2}}{c^{2}}\right)+\exp(-t_{1})+p(n_{T})+\exp(-t)
\]



\subsubsection*{Proof}

We break up the probability of interest into the following components

\begin{eqnarray*}
 &  & Pr\left(\left\Vert \hat{g}(\cdot|\hat{\boldsymbol{\lambda}})-g^{*}\right\Vert _{V}^{2}-\left[\min_{\lambda\in\Lambda}\left\Vert \hat{g}(\cdot|\boldsymbol{\lambda})-g^{*}\right\Vert _{V}^{2}\right]\ge\delta^{2}\right)\\
 & \le & Pr\left(\left\Vert \hat{g}(\cdot|\hat{\boldsymbol{\lambda}})-g^{*}\right\Vert _{V}^{2}-\left[\min_{\lambda\in\Lambda}\left\Vert \hat{g}(\cdot|\boldsymbol{\lambda})-g^{*}\right\Vert _{V}^{2}\right]\ge\delta^{2}\wedge\min_{\lambda\in\Lambda}\left\Vert \hat{g}(\cdot|\boldsymbol{\lambda})-g^{*}\right\Vert _{V}^{2}\le\tilde{\delta}^{2}\right)\\
 &  & +Pr\left(\min_{\lambda\in\Lambda}\left\Vert \hat{g}(\cdot|\boldsymbol{\lambda})-g^{*}\right\Vert _{V}^{2}\ge\tilde{\delta}_{t}^{2}\right)
\end{eqnarray*}


We bound the first probability term using Theorem 3: For our choice
of $\delta$, we have that
\begin{eqnarray*}
Pr\left(\left\Vert \hat{g}(\cdot|\hat{\boldsymbol{\lambda}})-g^{*}\right\Vert _{V}^{2}-\left[\min_{\lambda\in\Lambda}\left\Vert \hat{g}(\cdot|\boldsymbol{\lambda})-g^{*}\right\Vert _{V}^{2}\right]\ge\delta^{2}\wedge\min_{\lambda\in\Lambda}\left\Vert \hat{g}(\cdot|\boldsymbol{\lambda})-g^{*}\right\Vert _{V}^{2}\le\tilde{\delta}^{2}\right) & \le & c\exp\left(-\frac{n_{V}\delta^{4}}{c^{2}\tilde{\delta}_{t}}\right)+c\exp\left(-\frac{n_{V}\delta^{2}}{c^{2}}\right)+c\exp\left(-\frac{n_{T}\sigma^{2}}{c^{2}}\right)
\end{eqnarray*}


To bound the second probability term, we use the results just established

\begin{eqnarray*}
 &  & Pr\left(\min_{\lambda\in\Lambda}\left\Vert \hat{g}(\cdot|\boldsymbol{\lambda})-g^{*}\right\Vert _{V}^{2}\ge\tilde{\delta}_{t}^{2}\right)\\
 & \le & Pr\left(\left\Vert \hat{g}(\cdot|\tilde{\boldsymbol{\lambda}}_{gen})-g^{*}\right\Vert _{V}^{2}\ge\tilde{\delta}_{t}^{2}\right)\\
 & \le & Pr\left(\left|\left\Vert \hat{g}(\cdot|\tilde{\boldsymbol{\lambda}}_{gen})-g^{*}\right\Vert _{V}^{2}-\left\Vert \hat{g}(\cdot|\tilde{\boldsymbol{\lambda}}_{gen})-g^{*}\right\Vert ^{2}\right|+\left\Vert \hat{g}(\cdot|\tilde{\boldsymbol{\lambda}}_{gen})-g^{*}\right\Vert ^{2}\ge\tilde{\delta}_{t}^{2}\right)\\
 & \le & Pr\left(\left|\left\Vert \hat{g}(\cdot|\tilde{\boldsymbol{\lambda}}_{gen})-g^{*}\right\Vert _{V}^{2}-\left\Vert \hat{g}(\cdot|\tilde{\boldsymbol{\lambda}}_{gen})-g^{*}\right\Vert ^{2}\right|+\left\Vert \hat{g}(\cdot|\tilde{\boldsymbol{\lambda}}_{gen})-g^{*}\right\Vert ^{2}\ge\tilde{\delta}_{t}^{2}\wedge\frac{1}{C_{1}}\left|\left\Vert \hat{g}(\cdot|\tilde{\boldsymbol{\lambda}}_{gen})-g^{*}\right\Vert _{V}^{2}-\left\Vert \hat{g}(\cdot|\tilde{\boldsymbol{\lambda}}_{gen})-g^{*}\right\Vert ^{2}\right|\le\left\Vert \hat{g}(\cdot|\tilde{\boldsymbol{\lambda}}_{gen})-g^{*}\right\Vert K\sqrt{\frac{t}{n_{V}}}+\frac{K^{2}t}{n_{V}}\right)\\
 &  & +Pr\left(\frac{1}{C_{1}}\left|\left\Vert \hat{g}(\cdot|\tilde{\boldsymbol{\lambda}}_{gen})-g^{*}\right\Vert _{V}^{2}-\left\Vert \hat{g}(\cdot|\tilde{\boldsymbol{\lambda}}_{gen})-g^{*}\right\Vert ^{2}\right|\ge\left\Vert \hat{g}(\cdot|\tilde{\boldsymbol{\lambda}}_{gen})-g^{*}\right\Vert K\sqrt{\frac{t}{n_{V}}}+\frac{K^{2}t}{n_{V}}\right)\\
 & \le & Pr\left(C_{1}\left(\left\Vert \hat{g}(\cdot|\tilde{\boldsymbol{\lambda}}_{gen})-g^{*}\right\Vert K\sqrt{\frac{t}{n_{V}}}+\frac{K^{2}t}{n_{V}}\right)+\left\Vert \hat{g}(\cdot|\tilde{\boldsymbol{\lambda}}_{gen})-g^{*}\right\Vert ^{2}\ge\tilde{\delta}_{t}^{2}\right)\\
 &  & +\exp(-t)\\
 & \le & Pr\left(C_{1}\left(\left\Vert \hat{g}(\cdot|\tilde{\boldsymbol{\lambda}}_{T})-g^{*}\right\Vert K\sqrt{\frac{t}{n_{V}}}+\frac{K^{2}t}{n_{V}}\right)+\left\Vert \hat{g}(\cdot|\tilde{\boldsymbol{\lambda}}_{T})-g^{*}\right\Vert ^{2}\ge\tilde{\delta}_{t}^{2}\right)+\exp(-t)\\
 & \le & Pr\left(C_{1}\left(\left\Vert \hat{g}(\cdot|\tilde{\boldsymbol{\lambda}}_{T})-g^{*}\right\Vert K\sqrt{\frac{t}{n_{V}}}+\frac{K^{2}t}{n_{V}}\right)+\left\Vert \hat{g}(\cdot|\tilde{\boldsymbol{\lambda}}_{T})-g^{*}\right\Vert ^{2}\ge\tilde{\delta}_{t}^{2}\wedge\left\Vert \hat{g}(\cdot|\tilde{\boldsymbol{\lambda}}_{T})-g^{*}\right\Vert ^{2}\le2Wn_{T}^{-2\omega}\wedge\left\Vert \hat{g}(\cdot|\tilde{\boldsymbol{\lambda}}_{T})-g^{*}\right\Vert _{T}^{2}\le Wn_{T}^{-2\omega}\right)\\
 &  & +Pr\left(\left\Vert \hat{g}(\cdot|\tilde{\boldsymbol{\lambda}}_{T})-g^{*}\right\Vert ^{2}\ge2Wn_{T}^{-2\omega}\wedge\left\Vert \hat{g}(\cdot|\tilde{\boldsymbol{\lambda}}_{T})-g^{*}\right\Vert _{T}^{2}\le Wn_{T}^{-2\omega}\right)\\
 &  & +Pr\left(\left\Vert \hat{g}(\cdot|\tilde{\boldsymbol{\lambda}}_{T})-g^{*}\right\Vert _{T}^{2}\ge Wn_{T}^{-2\omega}\right)\\
 &  & +\exp(-t)\\
 & \le & 0+\exp(-t_{1})+p(n_{T})+\exp(-t)
\end{eqnarray*}


The last line follows as long as we choose $t$ and $\tilde{\delta}_{t}$
such that 
\[
\tilde{\delta}_{t}^{2}\ge C_{1}\left(\sqrt{2W}n_{T}^{-\omega}K\sqrt{\frac{t}{n_{V}}}+\frac{K^{2}t}{n_{V}}\right)+2Wn_{T}^{-2\omega}
\]


and $t_{1}>0$ such that 
\[
\frac{1}{2}\ge\frac{2J_{\infty}\left(K,\mathcal{G}\right)+K\sqrt{t_{1}}}{\sqrt{n_{T}}}+\frac{4J_{\infty}^{2}(K,\mathcal{G})+K^{2}t_{1}}{n_{T}}
\]



\section{Appendix}


\subsection*{Lemma Vandegeer (Based on Vandegeer Corollary 8.3)}

(This lemma is directly out of Vandegeer's Empirical Process book.
We apply this to the case where $\sigma=\infty$)

Let $Q_{m}$ be the empirical distributon of $m$ observations at
covariates $x_{i}$.

Suppose $\epsilon$ are $m$ independent sub-gaussian errors. Suppose
the model class $\mathcal{F}(T)$ has elements $\sup_{f\in\mathcal{F}_{n}(T)}\|f\|_{Q_{m}}\le R$
and satisfies
\[
\psi_{T}(R)\ge\int_{0}^{R}H^{1/2}(u,\mathcal{F}(T),\|\cdot\|_{Q_{m}})du
\]


There is $a$ dependent only on the sub-gaussian constants such that
for all $\delta>0$ such that
\[
\sqrt{m}\delta\ge a(\psi_{T}(R)\vee R)
\]


we have 
\[
Pr\left(\left.\sup_{f\in\mathcal{F}(T)}\left|\frac{1}{m}\sum_{i=1}^{m}\epsilon_{i}f(x_{i})\right|\ge\delta\right|T\right)\le a\exp\left(-\frac{m\delta^{2}}{4a^{2}R^{2}}\right)
\]



\subsection*{Lemma Cube Covering Number}

Suppose $\Lambda=[\lambda_{min},\lambda_{max}]^{J}$. Then the $\delta$-covering
number is bounded as follows 
\[
N(\delta,\Lambda,\|\cdot\|_{2})\le\frac{1}{C_{J}}\left(\frac{4(\lambda_{max}-\lambda_{min})+2\delta}{\delta}\right)^{J}
\]


where 
\[
C_{J}=\frac{\mbox{volume of ball of radius }\rho}{\rho^{J}}=\frac{\pi^{J/2}}{\Gamma(\frac{J}{2}+1)}
\]
.


\subsubsection*{Proof}

(Essentially the same proof as that for Lemma 2.5 in vandegeer)

Let $C=\{c_{j}\}_{j=1}^{N}\subset\Lambda$ be the largest set s.t.
two distinct points $c_{j_{1}},c_{j_{2}}$ are at least $\delta$
apart. Then balls with radius $\delta$ centered at $C$ cover $\Lambda$.
Hence 
\[
N(\delta,\Lambda,\|\cdot\|_{2})\le N
\]


If we instead consider the balls centered at $C$ but with radius
$\delta/4$, all of these smaller balls must be disjoint and are completely
contained in the box $\Lambda_{bigger}=[\lambda_{min}-\delta/4,\lambda_{max}+\delta/4]^{J}$.
So we know the aggregate volume of these smaller balls is less than
the volume of $\Lambda_{bigger}$.

Hence
\[
NC_{J}(\delta/4)^{J}\le(\lambda_{max}-\lambda_{min}+\delta/2)^{J}
\]



\subsection*{Lemma Sub-gaussian RV Bounds}

RVs $\epsilon_{1},...,\epsilon_{n}$ that are uniformly $b$-subgaussian
\[
\max_{i=1,...,n}\mathbb{E}e^{t\epsilon_{i}}\le e^{b^{2}t^{2}/2}
\]


then there is a constant $c$ only dependent on $b$ such that 
\[
Pr\left(\|\epsilon\|_{n}^{2}\ge\delta^{2}\right)\le c\exp\left(-\frac{n\delta^{2}}{c^{2}}\right)
\]



\subsubsection*{Proof}

We note that by Appendix A of http://www.stat.berkeley.edu/\textasciitilde{}mjwain/stat210b/Chap2\_TailBounds\_Jan22\_2015.pdf,
we have
\[
\mathbb{E}e^{t\epsilon_{i}}\le e^{b^{2}t^{2}/2}\mbox{ }\forall t\iff\mathbb{E}e^{t\epsilon_{i}^{2}/2b^{2}}\le\frac{1}{\sqrt{1-t}}\mbox{ }\forall t
\]


Then we can use a Berstein bound/Chernoff bound:

\begin{eqnarray*}
Pr\left(\frac{1}{n}\sum_{i=1}^{n}\epsilon_{i}^{2}\ge\delta^{2}\right) & = & Pr\left(\sum_{i=1}^{n}\epsilon_{i}^{2}\ge n\delta^{2}\right)\\
 & \le & Pr\left(\exp\left[\frac{t}{2b^{2}}\sum_{i=1}^{n}\epsilon_{i}^{2}\right]\ge\exp\left[\frac{t}{2b^{2}}n\delta^{2}\right]\right)\\
 & \le & \exp\left[-\frac{t}{2b^{2}}n\delta^{2}\right]\mathbb{E}\left[\frac{t}{2b^{2}}\sum_{i=1}^{n}\epsilon_{i}^{2}\right]\\
 & \le & \exp\left[-\frac{t}{2b^{2}}n\delta^{2}\right]\frac{1}{\sqrt{1-t}}
\end{eqnarray*}


We can choose any $t>0$. Suppose $t=1/2$. Then
\[
Pr\left(\frac{1}{n}\sum_{i=1}^{n}\epsilon_{i}^{2}\ge\delta^{2}\right)\le2\exp\left[-\frac{n\delta^{2}}{4b^{2}}\right]
\]

\end{document}
