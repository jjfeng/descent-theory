\documentclass[]{article}
\usepackage{amsmath}
\usepackage{color}

\addtolength{\oddsidemargin}{-.5in}%
\addtolength{\evensidemargin}{-.5in}%
\addtolength{\textwidth}{1in}%
\addtolength{\textheight}{1.3in}%
\addtolength{\topmargin}{-.8in}%


\newcommand{\overall}[1]{\textcolor{blue}{#1}}

\newcommand{\point}[1]{\item \textcolor{blue}{#1}}
\newcommand{\reply}{\item[]\ }

%opening
\title{Response to Reviewer 1}

\begin{document}
	
	\maketitle
	
	We appreciate the helpful feedback from the reviewer. We have addressed your questions and comments. Below we give a point-by-point response to each of the questions:
	
	\subsubsection*{Specific Suggestions/comments}
	
	\begin{enumerate}
		\point{
			The key assumption is the C-Lipschitz condition of the penalized criterion function on the tuning parameters.  This is a high level condition and the authors try to illustrate it using some low-level sufficient conditions.
			In the current version, it is not clear whether these low-level conditions can be verified in specific examples, and they are sufficient for the main results of the paper.
			For example, in Lemma 2 of the paper, the authors introduced the subset of the tuning parameters such that the C-Lipschitz condition holds in the case that the penalty function is not smooth.
			Does this mean that one has to replace the original set by this subset in practice since it is not clear whether the C-Lipschitz condition holds outside this subset?
			Is this subset known or it has to be estimated?
		}
	
		\point{
			Since the authors consider the smooth and non-smooth penalty functions in the paper. It would be good if the can use two specific examples to illustrate the low-level conditions in Lemma 1, Lemma 2 and Lemma 3. For example, they may consider the penalty function using the Sobolev norm for smooth case, and the Lasso penalty for the non-smooth case.
		}
		
		\point{
		The notations of the paper are slightly messy. In Theorem 2 and Theorem 3, are the expectation operators taken under the empirical measure? In displays (4.26) and (4.32), does “I” stand for the identity matrix? Moreover, what does the inequality symbol in these two displays mean?
		}
		
		\reply We apologize that the notation in the paper were messy. We have updated much of the notation to improve the clarity of the equations.
	\end{enumerate} 
	
\end{document}
