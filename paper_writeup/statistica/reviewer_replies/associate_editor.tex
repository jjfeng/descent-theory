\documentclass[]{article}
\usepackage{amsmath}
\usepackage{color}

\addtolength{\oddsidemargin}{-.5in}%
\addtolength{\evensidemargin}{-.5in}%
\addtolength{\textwidth}{1in}%
\addtolength{\textheight}{1.3in}%
\addtolength{\topmargin}{-.8in}%


\newcommand{\overall}[1]{\textcolor{blue}{#1}}

\newcommand{\point}[1]{\item \textcolor{blue}{#1}}
\newcommand{\reply}{\item[]\ }

%opening
\title{Response to Associate Editor}

\begin{document}
	
	\maketitle
	
	
	Response from the associate editor:
	
	\overall{
		Two referees and I have read your paper. The consensus on the work is positive: we all thought that the paper was interesting and yielded useful results. The work does, however, need substantial revision. The two referees have provided detailed reports that clearly outline the issues that need to be addressed. Primary among them is the point raised by the first referee expressing doubt that the (low-level) conditions that underpin the results would be straightforward to establish in specific example. This question, in particular, must be addressed in the revision of your paper. I would ask also that you address each of the individual points raised by each of the referees, and clearly signal the changes you have made in response to those points in the revised paper as well as providing a detailed rejoinder to the referees' queries and comments.
	}

	We appreciate the helpful feedback from the associate editor.
	We have substantially updated the paper per the reviewer comments.
	In particular, we added four in-depth examples to illustrate that the conditions in our lemmas is satisfied.
	For Lemma 1 which applies to smooth penalty functions, we analyze a problem with multiple ridge penalties.
	In addition, we consider a generalized additive model which involves multiple Sobolev penalties.
	We transform the problem to its finite-dimensional equivalent, show that the Lipschitz condition is satisfied, and establish the oracle inequality for the training/validation split setting.
	For Lemma 2 which applies to nonsmooth penalty functions, we analyze the elastic net in both the training/validation split and cross-validation settings.
	
	Per Reviewer 1's comments, we also updated Section 4.1.2 since Lemma 2 in the original submission was limited in its applicability.
	We previously established that the fitted models are only Lipschitz over some subset of penalty parameters that are well-behaved.
	We now establish that the fitted models are Lipschitz over almost every pair of penalty parameters under some new assumptions.
	We believe these assumptions are still reasonable (we are able to apply Lemma 2 to analyze the elastic net, for example).

	In addition, upon taking a second pass at this paper, we found that Theorem 2 in the paper can be extended to more general settings.
	We have loosened the assumptions needed in Theorem 2 so that we can consider problems with unbounded covariates $X$ and noise $\epsilon$.
	This required extending the results in Lecue and Mitchell (2012).
	Thus Theorem 2 and 4 in the latest revision are an extension of the results in Lecue and Mitchell (2012).

	Finally, we have made a significant effort in updating the notation in the paper. We thank the reviewers for pointing out the typos.
\end{document}
