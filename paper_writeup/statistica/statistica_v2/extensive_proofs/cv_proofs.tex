%% LyX 2.1.3 created this file.  For more info, see http://www.lyx.org/.
%% Do not edit unless you really know what you are doing.
\documentclass[english]{article}
\usepackage[T1]{fontenc}
\usepackage[latin9]{inputenc}
\usepackage{geometry}
\geometry{verbose,tmargin=3cm,bmargin=3cm,lmargin=3cm,rmargin=3cm}
\usepackage{amsthm}
\usepackage{amsmath}
\usepackage{amssymb}
\usepackage{esint}

\makeatletter
%%%%%%%%%%%%%%%%%%%%%%%%%%%%%% Textclass specific LaTeX commands.
\theoremstyle{plain}
\newtheorem{thm}{\protect\theoremname}
  \theoremstyle{plain}
  \newtheorem{lem}[thm]{\protect\lemmaname}
  \theoremstyle{plain}
  \newtheorem{cor}[thm]{\protect\corollaryname}

\makeatother

\usepackage{babel}
  \providecommand{\corollaryname}{Corollary}
  \providecommand{\lemmaname}{Lemma}
\providecommand{\theoremname}{Theorem}

\begin{document}
In this paper, we consider the least squares loss. Thus we can specialize
the results in Lecue and Mitchell (LM) to the case where $f\mapsto R(f)$
is convex. Thus we will be establishing oracle inequalities on the
averaged version of the modified CV. We denote the averaged version
of the modified CV as $\bar{g}(\hat{\boldsymbol{\lambda}}|D^{(n)})$.

First we directly apply Lemma 3.1 from LM to our least squares problem.
\begin{lem}
(Lemma 3.1 from LM, a shifted basic inequality)

For any constant $a\ge0$, we have
\begin{eqnarray*}
 &  & \mathbb{P}_{D^{(n)}}\left(\|\bar{g}(\hat{\boldsymbol{\lambda}}|D^{(n)})(x)-g^{*}(x)\|_{L_{2}}^{2}\right)\\
 & \le & (1+a)\inf_{\lambda\in\Lambda}\left[\mathbb{P}_{D^{(n_{T})}}\left(\|\hat{g}(\boldsymbol{\lambda}|D^{(n_{T})})(x)-g^{*}(x)\|_{L_{2}}^{2}\right)\right]\\
 &  & +\mathbb{P}_{D^{(n)}}\left[\sup_{\lambda\in\Lambda}\left(\mathbb{P}-(1+a)P_{n_{V}}\right)\left(\left(y-\hat{g}(\boldsymbol{\lambda}|D^{(n_{T})})(x)\right)^{2}-\left(y-g^{*}(x)\right)^{2}\right)\right]
\end{eqnarray*}


where $P_{n_{V}}=\frac{1}{n_{v}}\sum_{Z_{i}\in D^{(n_{V})}}\delta_{Z_{i}}$
is the empirical average.\end{lem}
\begin{proof}
By Lemma 3.1
\begin{eqnarray*}
 &  & \mathbb{P}_{D^{(n)}}\left(\|y-\bar{g}(\hat{\boldsymbol{\lambda}}|D^{(n)})(x)\|_{L_{2}}^{2}-\|y-g^{*}(x)\|_{L_{2}}^{2}\right)\\
 & \le & (1+a)\inf_{\lambda\in\Lambda}\left[\mathbb{P}_{D^{(n_{T})}}\left(\|y-\hat{g}(\boldsymbol{\lambda}|D^{(n_{T})})(x)\|_{L_{2}}^{2}-\|y-g^{*}(x)\|_{L_{2}}^{2}\right)\right]\\
 &  & +\mathbb{P}_{D^{(n)}}\left[\sup_{\lambda\in\Lambda}\left(\mathbb{P}-(1+a)P_{n_{V}}\right)\left(\left(y-\hat{g}(\boldsymbol{\lambda}|D^{(n_{T})})(x)\right)^{2}-\left(y-g^{*}(x)\right)^{2}\right)\right].
\end{eqnarray*}


Note that for any $m\in\mathbb{N}$, 
\[
\mathbb{P}_{D^{(m)}}\left(\|y-\bar{g}(\hat{\boldsymbol{\lambda}}|D^{(m)})(x)\|_{L_{2}}^{2}-\|y-g^{*}(x)\|_{L_{2}}^{2}\right)=\mathbb{P}_{D^{(m)}}\left(\|\bar{g}(\hat{\boldsymbol{\lambda}}|D^{(m)})(x)-g^{*}(x)\|_{L_{2}}^{2}\right).
\]

\end{proof}
Next, we restate Lemma 3.4 to clarify the dependencies and use notation
more amenable for establishing our desired results.
\begin{lem}
(Lemma 3.4 from LM)

Let $\mathcal{Q}(D^{(m)})\equiv\left\{ Q(\lambda|D^{(m)}):\lambda\in\Lambda\right\} $
and $\mathcal{Q}\equiv\cup_{m\in\mathbb{N}}\cup_{D^{(m)}}\mathcal{Q}(D^{(m)})$.

Suppose there exists $C_{1}>0$ and increasing function $G(\cdot)$
such that 
\[
\forall Q\in\mathcal{Q},\quad\|Q(Z)\|_{L_{2}}\le G\left(\mathbb{P}Q(Z)\right)\text{ and }\|Q(Z)\|_{L_{\psi_{1}}}\le C_{1}.
\]


Let $n_{T},n_{V}\in\mathbb{N}$ and fix a dataset $D^{(n_{T})}$.
Suppose there exist functions $D^{(n_{T})}\mapsto h(D^{(n_{T})})\in\mathbb{R}^{+}$
and $\left\{ \epsilon\mapsto J_{\delta}(\epsilon):\delta>0\right\} $
and a constant $\epsilon_{\min}>0$ such that for any dataset $D^{(n_{T})}$,
\[
h(D^{(n_{T})})\le\delta\implies\frac{\log n_{V}}{\sqrt{n_{V}}}\gamma_{1}\left(\mathcal{Q}_{\epsilon}^{L_{2}}(D^{(n_{T})}),\|\cdot\|_{L_{\psi_{1}}}\right)+\gamma_{2}\left(\mathcal{Q}_{\epsilon}^{L_{2}}(D^{(n_{T})}),\|\cdot\|_{L_{2}}\right)\le J_{\delta}(\epsilon)\forall\epsilon\ge\epsilon_{\min}
\]
where $\mathcal{Q}_{\epsilon}^{L_{2}}(D^{(n_{T})})\equiv\left\{ Q\in\mathcal{Q}(D^{(n_{T})}):\|Q(Z)\|_{L_{2}}\le G(\epsilon)\right\} $.

Then there exists some absolute constant $L,c>0$ such that for all
$\epsilon\ge\epsilon_{\min}$ and for all $u\ge1$,
\[
\Pr\left(\sup_{Q\in\mathcal{Q}(D^{(n_{T})})}\left(\left(\mathbb{P}-P_{n_{V}}\right)Q\right)_{+}\le uL\frac{J_{\delta}(\epsilon)}{\sqrt{n_{V}}}\mid h\left(D^{(n_{T})}\right)\le\delta\right)\ge1-L\exp(-cu).
\]

\end{lem}
Next, we restate Lemma 3.2 in a similar fashion to clarify the dependencies.
\begin{lem}
(Lemma 3.2 from LM)

Let $a>0$. Let $\mathcal{Q}(D^{(m)})\equiv\left\{ Q(\lambda|D^{(m)}):\lambda\in\Lambda\right\} $
be a set of measurable functions.

Let random variable $Z$ satisfy for all $m\in\mathbb{N}$, any dataset
$D^{(m)}$, $\mathbb{P}Q(Z)\ge0\forall Q\in\mathcal{Q}\left(D^{(m)}\right)$.

Suppose for any $n_{T},n_{V}\in\mathbb{N}$ and dataset $D^{(n_{T})}$
there exists some absolute constant $L,c>0$ such that for all $\epsilon\ge\epsilon_{\min}$
and for all $u\ge1$,

\[
\Pr\left(\sup_{Q\in\mathcal{Q}(D^{(n_{T})})}\left(\left(\mathbb{P}-P_{n_{V}}\right)Q\right)_{+}\le uL\frac{J_{\delta}(\epsilon)}{\sqrt{n_{V}}}\mid h\left(D^{(n_{T})}\right)\le\delta\right)\ge1-L\exp(-cu).
\]


Suppose every function in $\left\{ J_{\delta}:\delta>0\right\} $
is strictly increasing and its inverse is strictly convex.

Let $\psi_{\delta}$be the convex conjugate of $J_{\delta}^{-1}$,
e.g. $\psi_{\delta}(u)=\sup_{v>0}uv-J_{\delta}^{-1}(v)\forall u>0$.
Assume there is a $r\ge1$ such that $x>0\mapsto\psi(x)/x^{r}$ decreases
and define for $q>1$ and $u\ge1$,
\[
\epsilon_{q,\delta}(u)=\psi_{\delta}\left(\frac{2q^{r+1}(1+a)u}{a\sqrt{n_{V}}}\right)\vee\epsilon_{\min}.
\]


Then there exists a constant $L_{1}$ that only depends on $L$ such
that for every $u\ge1$,
\[
\Pr\left(\sup_{Q\in\mathcal{Q}(D^{(n_{T})})}\left(\left(\mathbb{P}-(1+a)P_{n_{V}}\right)Q\right)_{+}\le\frac{a\epsilon_{q,\delta}(u/q)}{q}\mid h\left(D^{(n_{T})}\right)\le\delta\right)\ge1-L_{1}\exp(-cu).
\]


Moreover, assume that $\psi_{\delta}$ increases such that $\psi_{\delta}(\infty)=\infty$.
Then there exists a constant $c_{1}$that depends only on $L$ and
$c$ such that
\[
\mathbb{P}\left[\sup_{Q\in\mathcal{Q}(D^{(n_{T})})}\left(\left(\mathbb{P}-(1+a)P_{n_{V}}\right)Q\right)_{+}\mid h\left(D^{(n_{T})}\right)\le\delta\right]\le\frac{ac_{1}\epsilon_{q,\delta}(1/q)}{q}.
\]

\end{lem}
Finally, we are ready to state our extension to LM's results. It is
the result of a simple chaining argument.
\begin{lem}
Let $a>0$.

There exist a constant $c_{1}$such that for any $n_{T},n_{V},\delta$
and $q>1$, we have 
\[
\mathbb{P}\left[\sup_{Q\in\mathcal{Q}(D^{(n_{T})})}\left(\left(\mathbb{P}-(1+a)P_{n_{V}}\right)Q\right)_{+}\mid h\left(D^{(n_{T})}\right)\le\delta\right]\le\frac{ac_{1}\epsilon_{q,\delta}(1/q)}{q}.
\]


Then for any $\sigma$,
\[
\mathbb{P}\left[\sup_{Q\in\mathcal{Q}(D^{(n_{T})})}\left(\left(\mathbb{P}-(1+a)P_{n_{V}}\right)Q\right)_{+}\right]\le\frac{ac\epsilon_{q,2\sigma}(1/q)}{q}+\sum_{k=1}^{\infty}\Pr\left(h\left(D^{(n_{T})}\right)\ge2^{k}\sigma\right)\frac{ac\epsilon_{q,2^{k}\sigma}(1/q)}{q}.
\]
\end{lem}
\begin{proof}
By a simple chaining argument, we have
\begin{eqnarray*}
 &  & \mathbb{P}\left[\sup_{Q\in\mathcal{Q}(D^{(n_{T})})}\left(\left(\mathbb{P}-(1+a)P_{n_{V}}\right)Q\right)_{+}\right]\\
 & \le & \Pr\left(h\left(D^{(n_{T})}\right)\le2\sigma\right)\mathbb{P}\left[\sup_{Q\in\mathcal{Q}(D^{(n_{T})})}\left(\left(\mathbb{P}-(1+a)P_{n_{V}}\right)Q\right)_{+}\mid h\left(D^{(n_{T})}\right)\le2\sigma\right]\\
 &  & +\sum_{k=1}^{\infty}\Pr\left(2^{k}\sigma\le h\left(D^{(n_{T})}\right)\le2^{k+1}\sigma\right)\mathbb{P}\left[\sup_{Q\in\mathcal{Q}(D^{(n_{T})})}\left(\left(\mathbb{P}-(1+a)P_{n_{V}}\right)Q\right)_{+}\mid h\left(D^{(n_{T})}\right)\le2^{k+1}\sigma\right]\\
 & \le & \frac{ac\epsilon_{q,2\sigma}(1/q)}{q}+\sum_{k=1}^{\infty}\Pr\left(h\left(D^{(n_{T})}\right)\ge2^{k}\sigma\right)\frac{ac\epsilon_{q,2^{k}\sigma}(1/q)}{q}.
\end{eqnarray*}

\end{proof}
To see why the lemma above is useful, we will now walk through the
case where the fitted functions are Lipschitz in the hyper-parameters.
We will see that the chaining argument shows that all we need to do
is to control the following sum 
\[
\sum_{k=1}^{\infty}k\Pr\left(h\left(D^{(n_{T})}\right)\ge2^{k}\right).
\]

\begin{lem}
Covering number bounds with Lipschitz functions
\end{lem}
Suppose norm $\|\cdot\|$ satisfies 
\begin{align}
\|XY\|\le\|X\|_{*}\|Y\|_{*}\label{eq:cauchy-like}
\end{align}
for some norm $\|\cdot\|_{*}$ for any random variables $X,Y$ and
$\sup_{g\in\mathcal{G}}\|g(x)\|_{*},\|\epsilon\|_{*}<\infty$. Moreover,
suppose the Lipschitz assumption that for any $m\in\mathbb{N}$ and
any dataset $D^{(m)}$, we have 
\begin{align}
\left|\hat{g}^{(n_{T})}(\boldsymbol{\lambda}^{(1)}|D^{(n_{T})})(x)-\hat{g}^{(n_{T})}(\boldsymbol{\lambda}^{(1)}|D^{(n_{T})})(x)\right|\le G(x)\|\boldsymbol{\lambda}^{(1)}-\boldsymbol{\lambda}^{(2)}\|_{2}
\end{align}
We have 
\begin{align}
N(\mathcal{Q},u,\|\cdot\|)\le N\left(\Lambda,\frac{u}{\left(2\|\epsilon\|_{*}+6\sup_{g\in\mathcal{G}}\left\Vert g(x)\right\Vert _{*}\right)\|G(x)\|_{*}},\|\cdot\|_{2}\right)
\end{align}

\begin{proof}
Via simple rearrangements of expressions, triangle inequality, and
the Cauchy-schwarz-like assumption above, we have
\begin{align}
 & \left\Vert \left(y-\hat{g}^{(n_{T})}(\boldsymbol{\lambda}^{(1)}|D^{(n_{T})})(x)\right)^{2}-\left(y-g^{*}(x)\right)^{2}-\left(y-\hat{g}^{(n_{T})}(\boldsymbol{\lambda}^{(2)}|D^{(n_{T})})(x)\right)^{2}-\left(y-g^{*}(x)\right)^{2}\right\Vert \\
 & =\left\Vert \left(y-\hat{g}^{(n_{T})}(\boldsymbol{\lambda}^{(2)}|D^{(n_{T})})(x)\right)\left(\hat{g}^{(n_{T})}(\boldsymbol{\lambda}^{(2)}|D^{(n_{T})})(x)-\hat{g}^{(n_{T})}(\boldsymbol{\lambda}^{(1)}|D^{(n_{T})})(x)\right)-\left(\hat{g}^{(n_{T})}(\boldsymbol{\lambda}^{(2)}|D^{(n_{T})})(x)-\hat{g}^{(n_{T})}(\boldsymbol{\lambda}^{(1)}|D^{(n_{T})})(x)\right)^{2}\right\Vert \\
%
 & \le\left(2\|\epsilon\|_{*}+\left\Vert g^{*}(x)-\hat{g}(\boldsymbol{\lambda}^{(2)}|D^{(m)})(x)\right\Vert _{*}\right)\left\Vert \hat{g}^{(n_{T})}(\boldsymbol{\lambda}^{(2)}|D^{(n_{T})})(x)-\hat{g}^{(n_{T})}(\boldsymbol{\lambda}^{(1)}|D^{(n_{T})})(x)\right\Vert _{*}\\
 & \le\left(2\|\epsilon\|_{*}+6\sup_{g\in\mathcal{G}}\left\Vert g(x)\right\Vert _{*}\right)\left\Vert \hat{g}^{(n_{T})}(\boldsymbol{\lambda}^{(2)}|D^{(n_{T})})(x)-\hat{g}^{(n_{T})}(\boldsymbol{\lambda}^{(1)}|D^{(n_{T})})(x)\right\Vert _{*}
\end{align}
where we applied (\ref{eq:cauchy-like}) in the first inequality.

Applying the Lipschitz assumption, we have 
\begin{align}
\begin{split} & \left\Vert \left(y-\hat{g}^{(n_{T})}(\boldsymbol{\lambda}^{(1)}|D^{(n_{T})})(x)\right)^{2}-\left(y-g^{*}(x)\right)^{2}-\left(y-\hat{g}^{(n_{T})}(\boldsymbol{\lambda}^{(2)}|D^{(n_{T})})(x)\right)^{2}-\left(y-g^{*}(x)\right)^{2}\right\Vert \\
 & \le\left(2\|\epsilon\|_{*}+6\sup_{g\in\mathcal{G}}\left\Vert g(x)\right\Vert _{*}\right)\|G(x)\|_{*}\|\boldsymbol{\lambda}^{(1)}-\boldsymbol{\lambda}^{(2)}\|_{2}.
\end{split}
\end{align}

\end{proof}
For this to be useful for bounding the $\gamma$-function from Talagrand,
we need to check that a Cauchy-Schwarz-like inequality also holds
for the $\psi_{1}$-norm.
\begin{lem}
Orlicz norm Cauchy-Schwarz-like

For any random variables $X,Y$, we have

\[
\|XY\|_{L_{\psi_{1}}}\le\|X\|_{L_{\psi_{2}}}\|Y\|_{L_{\psi_{2}}}
\]
\end{lem}
\begin{proof}
By the definition of the Orlicz norm, it suffices to show that 
\[
\mathbb{P}\left[\psi_{1}\left(\frac{|XY|}{\|X\|_{L_{\psi_{2}}}\|Y\|_{L_{\psi_{2}}}}\right)\right]\le1.
\]


By Taylor expansion:
\begin{eqnarray*}
\mathbb{P}\left[\exp\left(\frac{|XY|}{\|X\|_{L_{\psi_{2}}}\|Y\|_{L_{\psi_{2}}}}\right)-1\right] & = & \mathbb{P}\left[\sum_{k=1}^{\infty}\left(\frac{|XY|}{\|X\|_{L_{\psi_{2}}}\|Y\|_{L_{\psi_{2}}}}\right)^{k}\frac{1}{k!}\right]\\
 & = & \mathbb{P}\left[\sum_{k=1}^{\infty}\left(\frac{|X|}{\|X\|_{L_{\psi_{2}}}}\right)^{k}\sqrt{\frac{1}{k!}}\left(\frac{|Y|}{\|Y\|_{L_{\psi_{2}}}}\right)^{k}\sqrt{\frac{1}{k!}}\right]\\
 & \le & \mathbb{P}\left[\sqrt{\sum_{k=1}^{\infty}\left(\left[\frac{|X|}{\|X\|_{L_{\psi_{2}}}}\right]^{2}\right)^{k}\frac{1}{k!}}\sqrt{\sum_{k=1}^{\infty}\left(\left[\frac{|Y|}{\|Y\|_{L_{\psi_{2}}}}\right]^{2}\right)^{k}\frac{1}{k!}}\right]\\
 & = & \mathbb{P}\left[\sqrt{\exp\left(\left[\frac{|X|}{\|X\|_{L_{\psi_{2}}}}\right]^{2}\right)-1}\sqrt{\exp\left(\left[\frac{|Y|}{\|Y\|_{L_{\psi_{2}}}}\right]^{2}\right)-1}\right]\\
 & = & \mathbb{P}\left[\sqrt{\psi_{2}\left(\frac{|X|}{\|X\|_{L_{\psi_{2}}}}\right)}\sqrt{\psi_{2}\left(\frac{|Y|}{\|Y\|_{L_{\psi_{2}}}}\right)}\right]\\
 & \le & \sqrt{\mathbb{P}\left[\psi_{2}\left(\frac{|X|}{\|X\|_{L_{\psi_{2}}}}\right)\right]\mathbb{P}\left[\psi_{2}\left(\frac{|Y|}{\|Y\|_{L_{\psi_{2}}}}\right)\right]}\\
 & = & 1
\end{eqnarray*}


where the first inequality followed by Holder's for infinite sequences
and the last inequality follows by Holders/Cauchy-Schwarz again.
\end{proof}
Now let us prove a bound on the covering number of the box $\Lambda$
\begin{lem}
Let $\Lambda=[\lambda_{\min},\lambda_{\max}]^{J}$ where $\lambda_{\min}\le\lambda_{\max}$.
Then 
\begin{align}
N\left(u,\Lambda,\|\cdot\|_{2}\right)\le\left(\frac{4\left(\lambda_{max}-\lambda_{min}\right)+2u}{u}\right)^{J}.
\end{align}
\end{lem}
\begin{proof}
Use a slight variation of the proof for Lemma 2.5 in Van de geer (2000). 
\end{proof}
Now we are ready to use the bounds on covering numbers and the Cauchy-Schwarz-like
inequality to bound the $\gamma$-function from Talagrand. Recall
from LM and Talagrand that (for an absolute constant $c$?) 
\[
\gamma_{\alpha}(T,D)\le c\int_{0}^{\text{Diam}(T,d)}\left(\log N(T,\epsilon,d)\right)^{1/\alpha}d\epsilon.
\]

\begin{lem}
$\gamma$-function from Talagrand for Lipschitz functions

Suppose there is a function $G(x|D^{(n_{T})})$ such that 
\[
\left|\hat{g}(\boldsymbol{\lambda}^{(1)}|D^{(n_{T})})(x)-\hat{g}(\boldsymbol{\lambda}^{(2)}|D^{(n_{T})})(x)\right|\le G(x|D^{(n_{T})})\|\boldsymbol{\lambda}^{(1)}-\boldsymbol{\lambda}^{(2)}\|_{2}.
\]


Let $\epsilon>0$. Let $\mathcal{Q}_{\epsilon}^{L_{2}}(D^{(n_{T})})\equiv\left\{ Q\in\mathcal{Q}(D^{(n_{T})}):\|Q(Z)\|_{L_{2}}\le\sqrt{\epsilon}\right\} $.
Then
\[
\gamma_{2}\left(\mathcal{Q}_{\epsilon}^{L_{2}}(D^{(n_{T})}),\|\cdot\|_{L_{2}}\right)\le2c\sqrt{\epsilon J}\left[\sqrt{\log\left(\frac{4\Delta_{\Lambda}C_{2}\|G(x|D^{(n_{T})})\|_{L_{2}}+4\sqrt{\epsilon}}{2\sqrt{\epsilon}}\right)}+\frac{\sqrt{\pi}}{2}\right]
\]
where $C_{2}\ge2\|\epsilon\|_{L_{2}}+6\sup_{g\in\mathcal{G}}\left\Vert g(x)\right\Vert _{L_{2}}$
and

\[
\gamma_{1}\left(\mathcal{Q}_{\epsilon}^{L_{2}}(D^{(n_{T})}),\|\cdot\|_{L_{\psi_{1}}}\right)\le cJD_{\psi_{1}}\left[\log\left(\frac{4\Delta_{\Lambda}C_{\psi_{2}}\|G(x|D^{(n_{T})})\|_{L_{\psi_{2}}}+2D_{\psi_{1}}}{D_{\psi_{1}}}\right)+1\right]
\]
where $C_{\psi_{2}}\ge2\|\epsilon\|_{L_{\psi_{2}}}+6\sup_{g\in\mathcal{G}}\left\Vert g(x)\right\Vert _{L_{\psi_{2}}}$
and $D_{\psi_{1}}\ge2\sup_{Q\in\mathcal{Q}}\|Q\|_{L_{\psi_{1}}}$.\end{lem}
\begin{proof}
First, we bound the diameters of the function space. 
\[
\text{Diam}\left(\mathcal{Q}_{\epsilon}^{L_{2}}(D^{(n_{T})}),\|\cdot\|_{L_{2}}\right)=\sup_{Q_{1},Q_{2}\in\mathcal{Q}_{\epsilon}^{L_{2}}(D^{(n_{T})})}\|Q_{1}-Q_{2}\|_{L_{2}}\le2\sup_{Q_{1}\in\mathcal{Q}_{\epsilon}^{L_{2}}(D^{(n_{T})})}\|Q_{1}\|_{L_{2}}=2\sqrt{\epsilon}
\]


and
\[
\text{Diam}\left(\mathcal{Q}_{\epsilon}^{L_{2}}(D^{(n_{T})}),\|\cdot\|_{L_{\psi_{1}}}\right)\le2\sup_{Q\in\mathcal{Q}}\|Q\|_{L_{\psi_{1}}}.
\]


Let $\Delta_{\Lambda}=\lambda_{max}-\lambda_{min}$. For notational
convenience, let $C_{2}\ge2\|\epsilon\|_{L_{2}}+6\sup_{g\in\mathcal{G}}\left\Vert g(x)\right\Vert _{L_{2}}$.
We plug in the results from the covering number lemmas above:
\begin{eqnarray*}
\gamma_{2}\left(\mathcal{Q}_{\epsilon}^{L_{2}}(D^{(n_{T})}),\|\cdot\|_{L_{2}}\right) & \le & c\int_{0}^{2\sqrt{\epsilon}}\sqrt{\log N\left(\mathcal{Q}_{\epsilon}^{L_{2}}(D^{(n_{T})}),u,\|\cdot\|_{L_{2}}\right)}du\\
 & \le & c\int_{0}^{2\sqrt{\epsilon}}\sqrt{\log N\left(\Lambda,\frac{u}{C_{2}\|G(x|D^{(n_{T})})\|_{L_{2}}},\|\cdot\|_{2}\right)}du\\
 & \le & c\int_{0}^{2\sqrt{\epsilon}}\sqrt{J\log\left(\frac{4\Delta_{\Lambda}C_{2}\|G(x|D^{(n_{T})})\|_{L_{2}}+2u}{u}\right)}du\\
 & \le & c\sqrt{J}\int_{0}^{2\sqrt{\epsilon}}\sqrt{\log\left(\frac{4\Delta_{\Lambda}C_{2}\|G(x|D^{(n_{T})})\|_{L_{2}}+4\sqrt{\epsilon}}{u}\right)}du\\
 & \le & 2c\sqrt{\epsilon J}\int_{0}^{1}\sqrt{\log\left(\frac{4\Delta_{\Lambda}C_{2}\|G(x|D^{(n_{T})})\|_{L_{2}}+4\sqrt{\epsilon}}{2\sqrt{\epsilon}v}\right)}dv
\end{eqnarray*}


where the second to last line is bcause $u\le2\sqrt{\epsilon}$ and
the $\log$ function is monotonic and the last line follows from a
change of variables. Now because $\sqrt{a+b}\le\sqrt{a}+\sqrt{b}$
for $a,b>0$, we have
\begin{eqnarray*}
\gamma_{2}\left(\mathcal{Q}_{\epsilon}^{L_{2}}(D^{(n_{T})}),\|\cdot\|_{L_{2}}\right) & \le & 2c\sqrt{\epsilon J}\int_{0}^{1}\sqrt{\log\left(\frac{4\Delta_{\Lambda}C_{2}\|G(x|D^{(n_{T})})\|_{L_{2}}+4\sqrt{\epsilon}}{2\sqrt{\epsilon}}\right)}+\sqrt{\log\frac{1}{v}}dv\\
 & \le & 2c\sqrt{\epsilon J}\left[\sqrt{\log\left(\frac{4\Delta_{\Lambda}C_{2}\|G(x|D^{(n_{T})})\|_{L_{2}}+4\sqrt{\epsilon}}{2\sqrt{\epsilon}}\right)}+\frac{\sqrt{\pi}}{2}\right].
\end{eqnarray*}


Now let's bound the other $\gamma$-function. Let $C_{\psi_{2}}\ge2\|\epsilon\|_{L_{\psi_{2}}}+6\sup_{g\in\mathcal{G}}\left\Vert g(x)\right\Vert _{L_{\psi_{2}}}$
and $D_{\psi_{1}}\ge2\sup_{Q\in\mathcal{Q}}\|Q\|_{L_{\psi_{1}}}$.
Then
\begin{eqnarray*}
\gamma_{1}\left(\mathcal{Q}_{\epsilon}^{L_{2}}(D^{(n_{T})}),\|\cdot\|_{L_{\psi_{1}}}\right) & \le & c\int_{0}^{D_{\psi_{1}}}\log N\left(\mathcal{Q}_{\epsilon}^{L_{2}}(D^{(n_{T})}),u,\|\cdot\|_{L_{\psi_{1}}}\right)du\\
 & \le & c\int_{0}^{D_{\psi_{1}}}\log N\left(\Lambda,\frac{u}{C_{\psi_{2}}\|G(x|D^{(n_{T})})\|_{L_{\psi_{2}}}},\|\cdot\|_{2}\right)du\\
 & \le & c\int_{0}^{D_{\psi_{1}}}J\log\left(\frac{4\Delta_{\Lambda}C_{\psi_{2}}\|G(x|D^{(n_{T})})\|_{L_{\psi_{2}}}+2u}{u}\right)du\\
 & \le & cJ\int_{0}^{D_{\psi_{1}}}\log\left(\frac{4\Delta_{\Lambda}C_{\psi_{2}}\|G(x|D^{(n_{T})})\|_{L_{\psi_{2}}}+2D_{\psi_{1}}}{u}\right)du\\
 & \le & cJD_{\psi_{1}}\int_{0}^{1}\log\left(\frac{4\Delta_{\Lambda}C_{\psi_{2}}\|G(x|D^{(n_{T})})\|_{L_{\psi_{2}}}+2D_{\psi_{1}}}{D_{\psi_{1}}v}\right)dv\\
 & = & cJD_{\psi_{1}}\int_{0}^{1}\log\left(\frac{4\Delta_{\Lambda}C_{\psi_{2}}\|G(x|D^{(n_{T})})\|_{L_{\psi_{2}}}+2D_{\psi_{1}}}{D_{\psi_{1}}}\right)+\log\frac{1}{v}dv\\
 & = & cJD_{\psi_{1}}\left[\log\left(\frac{4\Delta_{\Lambda}C_{\psi_{2}}\|G(x|D^{(n_{T})})\|_{L_{\psi_{2}}}+2D_{\psi_{1}}}{D_{\psi_{1}}}\right)+1\right]
\end{eqnarray*}
by the same set of arguments as above.

Now we combine these bounds on the $\gamma$-functions and Lemma 2
(also Lemma 3.4 in LM) to get the following corollary.\end{proof}
\begin{cor}
Let $h(D^{(n_{T})})=c\|G(x|D^{(n_{T})})\|_{L_{\psi_{2}}}$ for an
absolute constant $c$ and 
\[
J_{\delta}(\epsilon)=\frac{\log n_{V}}{\sqrt{n_{V}}}cJD_{\psi_{1}}\left[\log\left(\frac{4\Delta_{\Lambda}C_{\psi_{2}}\delta/c+2D_{\psi_{1}}}{D_{\psi_{1}}}\right)+1\right]+\sqrt{\epsilon}2c\sqrt{J}\left[\sqrt{\log\left(2\Delta_{\Lambda}C_{2}\delta n_{V}+2\right)}+\frac{\sqrt{\pi}}{2}\right].
\]
Then for all $D^{(n_{T})}$ such that $h(D^{(n_{T})})=c\|G(x|D^{(n_{T})})\|_{L_{\psi_{2}}}\le\delta$,
we have $\forall\epsilon\ge1/n_{V}$
\[
\frac{\log n_{V}}{\sqrt{n_{V}}}\gamma_{1}\left(\mathcal{Q}_{\epsilon}^{L_{2}}(D^{(n_{T})}),\|\cdot\|_{L_{\psi_{1}}}\right)+\gamma_{2}\left(\mathcal{Q}_{\epsilon}^{L_{2}}(D^{(n_{T})}),\|\cdot\|_{L_{2}}\right)\le J_{\delta}(\epsilon).
\]
\end{cor}
\begin{thm}
Suppose that $\|\epsilon\|_{L_{\psi_{2}}},\sup_{g\in\mathcal{G}}\|g\|_{L_{\psi_{2}}}\le K$. 

Suppose that 
\begin{align}
\left|\hat{g}^{(n_{T})}(\boldsymbol{\lambda}^{(1)}|D^{(n_{T})})(x)-\hat{g}^{(n_{T})}(\boldsymbol{\lambda}^{(1)}|D^{(n_{T})})(x)\right|\le G(x|D^{(n_{T})})\|\boldsymbol{\lambda}^{(1)}-\boldsymbol{\lambda}^{(2)}\|_{2}.
\end{align}


Let $\tilde{h}$ be the function 
\[
\tilde{h}(n_{T})\ge\sum_{k=1}^{\infty}k\Pr\left(c\|G(x|D^{(n_{T})})\|_{L_{\psi_{2}}}\ge2^{k}\sigma\right)
\]


where $c$ is an absolute constant.

(The bound $\tilde{h}(n_{T})\ge\int\left(\log_{2}\left(c\|G(x|D^{(n_{T})})\|_{L_{\psi_{2}}}/\sigma\right)\right)^{2}d\mu\left(D^{(n_{T})}\right)$
also works.)

Then

\begin{eqnarray*}
 &  & \mathbb{P}_{D^{(n)}}\left(\|\bar{g}(\hat{\boldsymbol{\lambda}}|D^{(n)})(x)-g^{*}(x)\|_{L_{2}}^{2}\right)\\
 & \le & (1+a)\inf_{\lambda\in\Lambda}\left[\mathbb{P}_{D^{(n_{T})}}\left(\|\hat{g}(\boldsymbol{\lambda}|D^{(n_{T})})(x)-g^{*}(x)\|_{L_{2}}^{2}\right)\right]\\
 &  & +ac\left(\frac{\log n_{V}}{n_{V}}c_{0}JK\left[\log\left(c_{1}\Delta_{\Lambda}\sigma+1\right)+1\right]\frac{(1+a)}{a}+c_{2}\epsilon\frac{J}{n_{V}}\left[\sqrt{\log\left(c_{3}\Delta_{\Lambda}K\sigma n_{V}+1\right)}+1\right]^{2}\left(\frac{1+a}{a}\right)^{2}\right)\left[1+\tilde{h}(n_{T})\right]
\end{eqnarray*}
\end{thm}
\begin{proof}
By Lemma 1, 
\begin{eqnarray*}
 &  & \mathbb{P}_{D^{(n)}}\left(\|\bar{g}(\hat{\boldsymbol{\lambda}}|D^{(n)})(x)-g^{*}(x)\|_{L_{2}}^{2}\right)\\
 & \le & (1+a)\inf_{\lambda\in\Lambda}\left[\mathbb{P}_{D^{(n_{T})}}\left(\|\hat{g}(\boldsymbol{\lambda}|D^{(n_{T})})(x)-g^{*}(x)\|_{L_{2}}^{2}\right)\right]\\
 &  & +\mathbb{P}_{D^{(n)}}\left[\sup_{\lambda\in\Lambda}\left(\mathbb{P}-(1+a)P_{n_{V}}\right)\left(\left(y-\hat{g}(\boldsymbol{\lambda}|D^{(n_{T})})(x)\right)^{2}-\left(y-g^{*}(x)\right)^{2}\right)\right].
\end{eqnarray*}


We need to bound the second term on the RHS. We would like to use
Lemma 4.

We need the following two assumptions to hold for all $m\in\mathbb{N}$
and any dataset $D^{(m)}$:

1. $\left\Vert \left(y-\hat{g}^{(m)}(\boldsymbol{\lambda}|D^{(m)})(x)\right)^{2}-\left(y-g^{*}(x)\right)^{2}\right\Vert _{L_{\psi_{1}}}\le K_{1}$

2. $\left\Vert \left(y-\hat{g}^{(m)}(\boldsymbol{\lambda}|D^{(m)})(x)\right)^{2}-\left(y-g^{*}(x)\right)^{2}\right\Vert _{L_{2}}\le K_{2}\left\Vert g^{*}(x)-\hat{g}(\boldsymbol{\lambda}|D^{(m)})(x)\right\Vert _{L_{2}}$

We check that the assumption 1 holds. 
\begin{align}
 & \left\Vert \left(y-\hat{g}^{(m)}(\boldsymbol{\lambda}|D^{(m)})(x)\right)^{2}-\left(y-g^{*}(x)\right)^{2}\right\Vert _{L_{\psi_{1}}}\\
 & =\left\Vert 2\epsilon\left(g^{*}(x)-\hat{g}(\boldsymbol{\lambda}|D^{(m)})(x)\right)+\left(g^{*}(x)-\hat{g}(\lambda|D^{(m)})(x)\right)^{2}\right\Vert _{L_{\psi_{1}}}\\
 & \le3\left\Vert \epsilon^{2}\vee\left(g^{*}(x)-\hat{g}(\boldsymbol{(}\lambda)|D^{(m)})(x)\right)^{2}\right\Vert _{L_{\psi_{1}}}\label{eq:sum2}
\end{align}
The last line follows by considering the case where $|\epsilon|>|g^{*}(x)-\hat{g}(\boldsymbol{(}\lambda)|D^{(m)})(x)|$
and when the inequality is in the other direction. By triangle inequality,
we then have 
\begin{align}
 & \left\Vert \left(y-\hat{g}^{(m)}(\boldsymbol{\lambda}|D^{(m)})(x)\right)^{2}-\left(y-g^{*}(x)\right)^{2}\right\Vert _{L_{\psi_{1}}}\\
 & \le3\|\epsilon^{2}\|_{L_{\psi_{1}}}+\|\left(g^{*}(x)-\hat{g}(\boldsymbol{(}\lambda)|D^{(m)})(x)\right)^{2}\|_{L_{\psi_{1}}}\\
 & =3\|\epsilon\|_{L_{\psi_{2}}}^{2}+\|g^{*}(x)-\hat{g}(\boldsymbol{(}\lambda)|D^{(m)})(x)\|_{L_{\psi_{2}}}^{2}
\end{align}
Since we assumed that $\|\epsilon\|_{L_{\psi_{2}}}$ and $\|G\|_{L_{\psi_{2}}}$
are finite, then clearly assumption 1 holds.

Next, we check that assumption 2 holds. 
\begin{align}
 & \left\Vert \left(y-\hat{g}^{(m)}(\boldsymbol{\lambda}|D^{(m)})(x)\right)^{2}-\left(y-g^{*}(x)\right)^{2}\right\Vert _{L_{2}}\\
 & =\left\Vert 2\epsilon\left(g^{*}(x)-\hat{g}(\boldsymbol{\lambda}|D^{(m)})(x)\right)+\left(g^{*}(x)-\hat{g}(\boldsymbol{\lambda}|D^{(m)})(x)\right)^{2}\right\Vert _{L_{2}}\\
 & \le\left(2\|\epsilon\|_{L_{2}}+\|g^{*}(x)-\hat{g}(\boldsymbol{\lambda}|D^{(m)})(x)\|_{L_{2}}\right)\left\Vert g^{*}(x)-\hat{g}(\boldsymbol{\lambda}|D^{(m)})(x)\right\Vert _{L_{2}}
\end{align}
Since $\|f\|_{L_{2}}\le c\|f\|_{L_{\psi_{2}}}$, then clearly assumption
2 holds. 

Plug in $C_{2}=c_{0}K,C_{\psi_{2}}=c_{1}K,D_{\psi_{1}}=c_{2}K$ for
absolute constants $c_{0},c_{1},c_{2}$ into $J_{\delta}(\epsilon)$
defined in Corollary 9. For other absolute constants, 
\[
J_{\delta}(\epsilon)=\frac{\log n_{V}}{\sqrt{n_{V}}}c_{0}JK\left[\log\left(c_{1}\Delta_{\Lambda}\delta+1\right)+1\right]+c_{2}\sqrt{\epsilon}\sqrt{J}\left[\sqrt{\log\left(c_{3}\Delta_{\Lambda}K\delta n_{V}+1\right)}+1\right].
\]


Now we calculate the convex conjugate of $J_{\delta}^{-1}$, where
$J_{\delta}$ is defined in Corollary 9. For notational ease, notice
that $J_{\delta}$ is of the form $J_{\delta}(t)=c_{1,\delta}+c_{2,\delta}\sqrt{\epsilon}$
for $c_{1,\delta},c_{2,\delta}>0$. So $J_{\delta}$ is strictly increasing
and $J_{\delta}^{-1}(v)=\left(\frac{v-c_{1,\delta}}{c_{2,\delta}}\right)^{2}$
is strictly convex. Then $\psi_{\delta}(u)=\sup_{v>0}uv-\left(\frac{v-c_{1,\delta}}{c_{2,\delta}}\right)^{2}=\frac{c_{1,\delta}u}{2}+\frac{c_{2,\delta}^{2}u^{2}}{4}$
for $u>0$. Note that $\psi_{\delta}(\infty)=\infty$ and $\psi_{\delta}(x)/x^{r}$
decreases for $r\ge2$.

Plugging the definition of $J_{\delta}(\epsilon)$ back in, the function
$\epsilon_{q,\delta}$ for $q>1$ and $u\ge1$ is
\begin{eqnarray*}
\epsilon_{q,\delta}(u) & = & \psi_{\delta}\left(\frac{2q^{r+1}(1+a)u}{a\sqrt{n_{V}}}\right)\vee\frac{1}{n_{V}}\\
 & = & \left[\frac{\log n_{V}}{n_{V}}c_{0}JK\left[\log\left(c_{1}\Delta_{\Lambda}\delta+1\right)+1\right]\frac{q^{r+1}(1+a)u}{a}+c_{2}\epsilon\frac{J}{n_{V}}\left[\sqrt{\log\left(c_{3}\Delta_{\Lambda}K\delta n_{V}+1\right)}+1\right]^{2}\left(\frac{q^{r+1}(1+a)u}{a}\right)^{2}\right]\vee\frac{1}{n_{V}}.
\end{eqnarray*}


(more new absolute constants above)

In particular, if we plug in $\delta=2^{k}\sigma$ for any $\sigma>0$
and $k\ge1$, we have (for a new set of absolute constants)
\[
\epsilon_{q,2^{k}\sigma}(u)\le\left[k\left(\frac{\log n_{V}}{n_{V}}c_{0}JK\left[\log\left(c_{1}\Delta_{\Lambda}\sigma+1\right)+1\right]\frac{q^{r+1}(1+a)u}{a}+c_{2}\epsilon\frac{J}{n_{V}}\left[\sqrt{\log\left(c_{3}\Delta_{\Lambda}K\sigma n_{V}+1\right)}+1\right]^{2}\left(\frac{q^{r+1}(1+a)u}{a}\right)^{2}\right)\right]\vee\frac{1}{n_{V}}.
\]


Plugging this into Lemma 4 with $Q(\lambda|D^{(n_{T})})=\left(y-\hat{g}^{(n_{T})}(\boldsymbol{\lambda}|D^{(n_{T})})(x)\right)^{2}-\left(y-g^{*}(x)\right)^{2}$
, the function class defined appropriately, and $h(D^{(n_{T})})=c\|G(x|D^{(n_{T})})\|_{L_{\psi_{2}}}$
we have 
\begin{eqnarray*}
 &  & \mathbb{P}\left[\sup_{Q\in\mathcal{Q}(D^{(n_{T})})}\left(\left(\mathbb{P}-(1+a)P_{n_{V}}\right)\left[\left(y-\hat{g}^{(n_{T})}(\boldsymbol{\lambda}|D^{(n_{T})})(x)\right)^{2}-\left(y-g^{*}(x)\right)^{2}\right]\right)_{+}\right]\\
 & \le & \frac{ac}{q}\left(\frac{\log n_{V}}{n_{V}}c_{0}JK\left[\log\left(c_{1}\Delta_{\Lambda}\sigma+1\right)+1\right]\frac{q^{r}(1+a)}{a}+c_{2}\epsilon\frac{J}{n_{V}}\left[\sqrt{\log\left(c_{3}\Delta_{\Lambda}K\sigma n_{V}+1\right)}+1\right]^{2}\left(\frac{q^{r}(1+a)}{a}\right)^{2}\right)\times\\
 &  & \left(1+\sum_{k=1}^{\infty}k\Pr\left(c\|G(x|D^{(n_{T})})\|_{L_{\psi_{2}}}\ge2^{k}\sigma\right)\right)
\end{eqnarray*}


Take $q\rightarrow1$ and using the assumption that $\tilde{h}(n_{T})\ge c\sum_{k=1}^{\infty}k\Pr\left(c\|G(x|D^{(n_{T})})\|_{L_{\psi_{2}}}\ge2^{k}\sigma\right)$
, we get
\begin{eqnarray*}
 &  & \mathbb{P}\left[\sup_{Q\in\mathcal{Q}(D^{(n_{T})})}\left(\left(\mathbb{P}-(1+a)P_{n_{V}}\right)Q\right)_{+}\right]\\
 & \le & ac\left(\frac{\log n_{V}}{n_{V}}c_{0}JK\left[\log\left(c_{1}\Delta_{\Lambda}\sigma+1\right)+1\right]\frac{(1+a)}{a}+c_{2}\epsilon\frac{J}{n_{V}}\left[\sqrt{\log\left(c_{3}\Delta_{\Lambda}K\sigma n_{V}+1\right)}+1\right]^{2}\left(\frac{1+a}{a}\right)^{2}\right)\left(1+\tilde{h}(n_{T})\right)
\end{eqnarray*}


The integral bound provided in the theorem might be easier to reason
with. To derive it, note that for any positive function $h$
\begin{eqnarray*}
\sum_{k=1}^{\infty}k\Pr\left(h\left(D^{(n_{T})}\right)\ge2^{k}\sigma\right) & = & \sum_{k=1}^{\infty}k\int1\left\{ h\left(D^{(n_{T})}\right)\ge2^{k}\sigma\right\} d\mu\left(D^{(n_{T})}\right)\\
 & = & \int1\left\{ \log_{2}\left(h\left(D^{(n_{T})}\right)/\sigma\right)\ge1\right\} \sum_{k=1}^{\log_{2}\left(h\left(D^{(n_{T})}\right)/\sigma\right)}kd\mu\left(D^{(n_{T})}\right)\\
 & = & \int1\left\{ \log_{2}\left(h\left(D^{(n_{T})}\right)/\sigma\right)\ge1\right\} \frac{1}{2}\left(\log_{2}\left(h\left(D^{(n_{T})}\right)/\sigma\right)\right)\left(\log_{2}\left(h\left(D^{(n_{T})}\right)/\sigma\right)+1\right)d\mu\left(D^{(n_{T})}\right)\\
 & \le & \int\left(\log_{2}\left(h\left(D^{(n_{T})}\right)/\sigma\right)\right)^{2}d\mu\left(D^{(n_{T})}\right).
\end{eqnarray*}

\end{proof}
The bound 
\[
\tilde{h}(n_{T})\ge\sum_{k=1}^{\infty}k\Pr\left(c\|G(x|D^{(n_{T})})\|_{L_{\psi_{2}}}\ge2^{k}\sigma\right)\approx\int\left(\log_{2}\left(c\|G(x|D^{(n_{T})})\|_{L_{\psi_{2}}}/\sigma\right)\right)^{2}d\mu\left(D^{(n_{T})}\right)
\]
 can be interpretted as a way of controlling the tail behavior of
the Lipschitz factor $G(x|D^{(n_{T})})$. Since we only need to control
the $\log$ of the $\psi_{2}$-norm, the tail behavior of the Lipschitz
factor can actually be quite crazy. As we will see, the Lipschitz
factor will turn out to be $\|G(x|D^{(n_{T})})\|_{L_{\psi_{2}}}=n_{T}^{\kappa}\left(P_{n_{T}}\epsilon^{2}\right)\|\|x\|_{2}\|_{L_{\psi_{2}}}$
for some constant $\kappa$. So $\tilde{h}(n_{T})$ is growing at
some $\log n_{T}$ rate.
\end{document}
