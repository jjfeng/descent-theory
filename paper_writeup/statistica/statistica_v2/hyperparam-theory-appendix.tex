\documentclass[10pt]{book}
\usepackage[sectionbib]{natbib}
\usepackage{array,epsfig,fancyheadings,rotating}
%\usepackage[dvipdfm]{hyperref}
%%%%%%%%%%%%%%%%%%%%%%%%%%%%%%%%%%%%%%%%%%%%%%%%%%%%%%%%%%%%%%%%%%%%%%%%%%%%%%%%%%%%%%%%%%%%%%%%%%%%%%%%%%%%%%%%%%%%%%%%%%%%

\textwidth=36pc
\textheight=49pc
\oddsidemargin=1pc
\evensidemargin=1pc
\headsep=15pt
\topmargin=.6cm
\parindent=1.7pc
\parskip=0pt

\usepackage{multirow}
\usepackage{amsmath}
\usepackage{amssymb}
\usepackage{amsfonts}
\usepackage{amsthm}
\usepackage{todonotes}
\usepackage{mathtools}
\usepackage{color}
\usepackage{xr}
\externaldocument{hyperparam-theory}

\setcounter{page}{1}
\newtheorem{theorem}{Theorem}
\newtheorem{lemma}{Lemma}
\newtheorem{corollary}{Corollary}
\newtheorem{proposition}{Proposition}
\theoremstyle{definition}
\newtheorem{definition}{Definition}

\newtheorem{example}{Example}
\newtheorem{remark}{Remark}

\DeclareMathOperator{\Diam}{Diam}
\newcommand{\textred}[1]{\textcolor{red}{#1}}


\setcounter{theorem}{2}
\setcounter{lemma}{3}
\setcounter{definition}{3}
\DeclareMathOperator*{\argmin}{arg\,min}
\DeclareMathOperator*{\conv}{Conv}
\pagestyle{fancy}
\newtheorem{assump}{Assumption}
\setcounter{assump}{2}
\newtheorem{condition}{Condition}
\DeclareMathOperator{\diag}{diag}
\DeclareMathOperator{\spann}{span}

%%%%%%%%%%%%%%%%%%%%%%%%%%%%%%%%%%%%%%%%%%%%%%%%%%%%%%%%%%%%%%%%%%%%%%%%%%%%%%%%%%%%%%%%%%%%%%%%%%%%%%%%%%%%%%%%%%%%%%%%%%%%
\pagestyle{fancy}
\def\n{\noindent}
\lhead[\fancyplain{} \leftmark]{}
\chead[]{}
\rhead[]{\fancyplain{}\rightmark}
\cfoot{}
%\headrulewidth=0pt

%%%%%%%%%%%%%%%%%%%%%%%%%%%%%%%%%%%%%%%%%%%%%%%%%%%%%%%%%%%%%%%%%%%%%%%%%%%%%%%%%%%%%%%%%%%%%%%%%%%%%%%%%%%%%%%%%%%%%%%%%%%%
%%%%%%%%%%%%%%%%%%%%%%%%%%%%%%%%%%%%%%%%%%%%%%%%%%%%%%%%%%%%%%%%%%%%%%%%%%%%%%%%%%%%%%%%%%%%%%%%%%%%%%%%%%%%%%%%%%%%%%%%%%%%

\begin{document}
%%%%%%%%%%%%%%%%%%%%%%%%%%%%%%%%%%%%%%%%%%%%%%%%%%%%%%%%%%%%%%%%%%%%%%%%%%%%%%%%%%%%%%%%%%%%%%%%%%%%%%%%%%%%%%%%%%%%%%%%%%%%
%%%%%%%%%%%%%%%%%%%%%%%%%%%%%%%%%%%%%%%%%%%%%%%%%%%%%%%%%%%%%%%%%%%%%%%%%%%%%%%%%%%%%%%%%%%%%%%%%%%%%%%%%%%%%%%%%%%%%%%%%%%%

\renewcommand{\baselinestretch}{2}

\markright{ \hbox{\footnotesize\rm Statistica Sinica: Supplement
%{\footnotesize\bf 24} (201?), 000-000
}\hfill\\[-13pt]
\hbox{\footnotesize\rm
%\href{http://dx.doi.org/10.5705/ss.20??.???}{doi:http://dx.doi.org/10.5705/ss.20??.???}
}\hfill }

\markboth{\hfill{\footnotesize\rm Jean Feng and Noah Simon} \hfill}
{\hfill {\footnotesize\rm Hyper-parameter selection via split-sample validation} \hfill}

\renewcommand{\thefootnote}{}
$\ $\par \fontsize{12}{14pt plus.8pt minus .6pt}\selectfont

%%%%%%%%%%%%%%%%%%%%%%%%%%%%%%%%%%%%%%%%%%%%%%%%%%%%%%%%%%%%%%%%%%%%%%%%%%%%%%%%%%%%%%%%%%%%%%%%%%%%%%%%%%%%%%%%%%%%%%%%%%%%

\centerline{\large\bf An analysis of the cost of hyper-parameter selection via split-sample}
\vspace{2pt}
\centerline{\large\bf validation, with applications to penalized regression}
\vspace{.25cm}
\centerline{Jean Feng and Noah Simon}
\vspace{.4cm}
\centerline{\it Department of Biostatistics, University of Washington}
\vspace{.55cm}
\centerline{\bf Supplementary Material}
\fontsize{9}{11.5pt plus.8pt minus .6pt}\selectfont
\noindent
\par

\setcounter{section}{0}
\setcounter{equation}{0}
\def\theequation{S\arabic{section}.\arabic{equation}}
\def\thesection{S\arabic{section}}

\fontsize{12}{14pt plus.8pt minus .6pt}\selectfont

\section{Appendix}\label{sec:proofs}

We will use the following notation: for functions $f$ and $g$ and a dataset $D$ with $m$ samples, we denote the inner product of $f$ and $g$ at covariates $D$ as $\langle f,g \rangle_{D} = \frac{1}{m} \sum_{(x_i, y_i) \in D} f(x_i, y_i) g(x_i, y_i) $.

\subsection{A single training/validation split}
\label{appendix:train_val}

Theorem \ref{thrm:train_val} is a special case of Theorem \ref{thrm:train_val_complicated}, which applies to general model-estimation procedures. The proof is based on the so-called ``basic inequality'' below.

\begin{lemma}
	For any $\tilde{\boldsymbol{\lambda}} \in \tilde{\Lambda}$, we have
	\begin{equation}
	\label{thrm:basic_ineq}
	\left \| g^* - \hat{g}^{(n_T)}(\hat{\boldsymbol{\lambda}}|T) \right \|^2_V 
	- \left \| g^* - \hat{g}^{(n_T)}(\tilde{\boldsymbol{\lambda}}|T) \right \|^2_V
	\le 
	2 \left \langle \epsilon, \hat{g}^{(n_T)}(\tilde{\boldsymbol{\lambda}}|T) - \hat{g}^{(n_T)}(\hat{\boldsymbol{\lambda}}|T) \right \rangle_V
	\end{equation}
\end{lemma}

\begin{proof}
	The desired result can be attained by rearranging the definition of $\hat{\boldsymbol{\lambda}}$
	\begin{equation}
	\left \| y - \hat{g}^{(n_T)}(\hat{\boldsymbol{\lambda}}|T) \right \|^2_V \le
	\min_{\tilde{\boldsymbol{\lambda}} \in \tilde{\Lambda}} \left \| y - \hat{g}^{(n_T)}(\tilde{\boldsymbol{\lambda}}|T) \right \|^2_V.
	\end{equation}
\end{proof}

We are therefore interested in bounding the empirical process term in \eqref{thrm:basic_ineq}. A common approach is to use a measure of complexity of the function class. For a single training/validation split, where we treat the training set as fixed, we only need to consider the complexity of the fitted models from the model-selection procedure
\begin{equation}
\mathcal{G}(T)=\left\{ \hat{g}^{(n_T)}(\boldsymbol{\lambda}|T) : \boldsymbol{\lambda} \in \Lambda \right\}.
\end{equation}
This model class can be considerably less complex compared to the original function class $\mathcal{G}$, such as the special case in Theorem \ref{thrm:train_val} where we suppose $\mathcal{G}(T)$ is Lipschitz. For this proof, we will use metric entropy as a measure of model class complexity. We recall its definition below.
\begin{definition}
	Let $\mathcal{F}$ be a function class. Let the covering number $N(u, \mathcal{F}, \| \cdot \|)$ be the smallest set of $u$-covers of $\mathcal{F}$ with respect to the norm $\| \cdot \|$. The metric entropy of $\mathcal{F}$ is defined as the log of the covering number:
	\begin{equation}
	H (u, \mathcal{F}, \| \cdot \| ) = \log N(u, \mathcal{F}, \| \cdot \|).
	\end{equation}
\end{definition}

We will bound the empirical process term using the following Lemma, which is a simplification of Corollary 8.3 in \citet{van2000empirical}.

\begin{lemma}
	\label{lemma:cor83}
	Suppose $D^{(m)} = \{x_1,...,x_m\}$ are fixed and $\epsilon_1,...,\epsilon_m$ are independent random variables with mean zero and uniformly sub-gaussian with parameters $b$ and $B$. Suppose
	the model class $\mathcal{F}$ satisfies $\sup_{f\in\mathcal{F}}\|f\|_{D^{(m)}}\le R$
	and
	\[
	\int_{0}^{R}H^{1/2}(u,\mathcal{F},\|\cdot\|_{D^{(m)}})du \le \mathcal{J} (R).
	\]
	
	
	There is a constant $a > 0$ dependent only on $b$ and $B$ such that
	for all $\delta>0$ satisfying
	\[
	\sqrt{m}\delta\ge a(\mathcal{J} (R)\vee R),
	\]
	we have 
	\[
	Pr\left(\sup_{f\in\mathcal{F}}\left|\frac{1}{m}\sum_{i=1}^{m}\epsilon_{i}f(x_{i})\right|\ge\delta\right)
	\le 
	a\exp\left(-\frac{m\delta^{2}}{4a^{2}R^{2}}\right).
	\]
	
\end{lemma}

We are now ready to prove the oracle inequality. It uses a standard peeling argument.

\begin{theorem}
	\label{thrm:train_val_complicated}
	Consider a set of hyper-parameters $\Lambda$.
	Let training data $T$ be fixed, as well as the covariates of the validation set $X_V$.
	Let the oracle risk be denoted
	\begin{equation}
	\tilde{R}(X_V|T) = \argmin_{\lambda \in \Lambda} \left \| g^*-\hat{g}^{(n_T)}( \boldsymbol{\lambda} | T) \right \|_{V}^{2}.
	\label{eq:tilde_lambda_def}
	\end{equation}
	
	Suppose independent random variables $\epsilon_i$ for validation set $V$ have expectation zero and are uniformly sub-Gaussian with parameter $b$ and $B$.
	Suppose there is a function $\mathcal{J} (\cdot | T):\mathbb{R}\mapsto\mathbb{R}$ and constant $r > 0$ such that
	\begin{equation}
	\label{eq:dudley_bound}
	\int_{0}^{R}H^{1/2}(u,\mathcal{G}(T),\|\cdot\|_{V})du\le \mathcal{J} (R| T) \quad \forall R>r
	\end{equation}
	Also, suppose $\mathcal{J} \left(u | T \right)/u^{2}$ is non-increasing in $u$ for all $u > r$.

	Then there is a constant $c>0$ only depending on $b$ and $B$ such that for all $\delta$ satisfying
	\begin{equation}
	\label{eq:train_val_delta_condn}
	\sqrt{n_V}\delta^{2}
	\ge
	c \left ( 
	\mathcal{J}(\delta| T)
	\vee 
	\delta
	\vee
	\mathcal{J} \left (\tilde{R}(X_V|T)\middle | T
	\right ) 
	\vee
	4 \tilde{R}(X_V|T) \right ),
	\end{equation}
	we have
	\begin{align}
		Pr\left(
		\left\Vert g^* - \hat{g}^{(n_T)}( \hat{\boldsymbol{\lambda}} | T) \right\Vert _{V}^2 -
		\tilde{R}(X_V|T)
		\ge\delta^2
		\middle | 
		T, X_V
		\right )
		&\le c\exp\left(-\frac{n_{V}\delta^{4}}{
			c^{2}
			\tilde{R}(X_V|T)
		}\right) 
		+c\exp\left(-\frac{n_{V}\delta^{2}}{c^{2}}\right).
	\end{align}
\end{theorem}

\begin{proof}
	Consider any $\tilde{\boldsymbol{\lambda}} \in \tilde{\Lambda}$.
	We will use the simplified notation $\hat{g}(\hat{\boldsymbol{\lambda}}) \coloneqq \hat{g}^{(n_T)}(\hat{\boldsymbol{\lambda}} | T)$ and $\hat{g}(\tilde{\boldsymbol{\lambda}}) \coloneqq \hat{g}^{(n_T)}(\tilde{\boldsymbol{\lambda}} | T)$. In addition, the following probabilities are all conditional on $X_V$ and $T$ but we leave them out for readability.
	\begin{align}
	& \Pr\left(
	\left\Vert \hat{g}(\hat{\boldsymbol{\lambda}})-g^{*}\right\Vert _{V}^{2}
	- \tilde{R}(X_V|T)
	\ge \delta^2
	\right) \label{eq:train_val_prob}\\
	& = \sum_{s=0}^{\infty}
	\Pr\left(
	2^{2s}\delta^{2}
	\le \left\Vert \hat{g}(\hat{\boldsymbol{\lambda}})-g^{*}\right\Vert _{V}^{2}
	-\tilde{R}(X_V|T)
	\le 2^{2s+2}\delta^{2}\right) 
	\label{eq:peeled} \\
	&\le \sum_{s=0}^{\infty}
	\Pr\left(
	2^{2s}\delta^{2}
	\le 2\left\langle \epsilon,\hat{g}(\hat{\boldsymbol{\lambda}})-\hat{g}(\tilde{\boldsymbol{\lambda}})\right\rangle _{V}\right. \label{eq:peel_ineq}\\
	& \qquad  \left.\wedge \left\Vert \hat{g}(\hat{\boldsymbol{\lambda}})-\hat{g}(\tilde{\boldsymbol{\lambda}})\right\Vert_{V}^{2}\le2^{2s+2}\delta^{2}+ 2\left|\left\langle \hat{g}(\tilde{\boldsymbol{\lambda}})-\hat{g}(\hat{\boldsymbol{\lambda}}),\hat{g}(\tilde{\boldsymbol{\lambda}})-g^{*}\right\rangle _{V}\right| \right ),
	\end{align}
	where we applied the basic inequality \eqref{thrm:basic_ineq} in the last line.
	Each summand in \eqref{eq:peel_ineq} can be bounded by splitting the event into the cases where either $2^{2s+2} \delta^2$ or $2\left|\left\langle \hat{g}(\tilde{\boldsymbol{\lambda}})-\hat{g}(\hat{\boldsymbol{\lambda}}),\hat{g}(\tilde{\boldsymbol{\lambda}})-g^{*}\right\rangle _{V}\right|$ is larger. Splitting up the probability and applying Cauchy Schwarz gives us the following bound for \eqref{eq:train_val_prob}
	\begin{align}
	& Pr\left(
	\sup_{\boldsymbol{\lambda} \in \Lambda: \left\Vert \hat{g}({\boldsymbol{\lambda}})-\hat{g}(\tilde{\boldsymbol{\lambda}})\right\Vert _{V}
		\le
		4\left\Vert \hat{g}(\tilde{\boldsymbol{\lambda}})-g^{*}\right\Vert _{V}}
	2\left\langle \epsilon,\hat{g}({\boldsymbol{\lambda}})-\hat{g}(\tilde{\boldsymbol{\lambda}})\right\rangle _{V}
	\ge 
	\delta^{2}
	\right)
	\label{eq:train_val_1}
	\\
	& + \sum_{s=0}^{\infty} Pr\left(
	\sup_{\boldsymbol{\lambda} \in \Lambda: \left\Vert \hat{g}({\boldsymbol{\lambda}})-\hat{g}(\tilde{\boldsymbol{\lambda}})\right\Vert _{V}
		\le
		2^{s+3/2}\delta}
	2\left\langle \epsilon,\hat{g}({\boldsymbol{\lambda}})-\hat{g}(\tilde{\boldsymbol{\lambda}})\right\rangle _{V}
	\ge
	2^{2s} \delta^{2}
	\right)
	\label{eq:train_val_2}.
	\end{align}
	
	We can bound both \eqref{eq:train_val_1} and \eqref{eq:train_val_2} using Lemma \ref{lemma:cor83}. For our choice of $\delta$ in \eqref{eq:train_val_delta_condn},
	there is some constant $a>0$ dependent only on $b$ such that \eqref{eq:train_val_1} is bounded above by
	\[ 
	a\exp\left(-\frac{n_{V}\delta^{4}}{4a^{2}\left(16\left\Vert \hat{g}(\tilde{\boldsymbol{\lambda}})-g^{*}\right\Vert _{V}^{2}\right)}\right).
	\]
	In addition, our choice of $\delta$ from \eqref{eq:train_val_delta_condn} and our assumption that $\psi(u)/u^2$ is non-increasing implies that the condition in Lemma \ref{lemma:cor83} is satisfied for all $s=0,1,...,\infty$ simultaneously. Hence for all $s=0,1,...,\infty$, we have
	\begin{align}
	Pr\left(
	\sup_{\boldsymbol{\lambda} \in \Lambda: \left\Vert \hat{g}({\boldsymbol{\lambda}})-\hat{g}(\tilde{\boldsymbol{\lambda}})\right\Vert _{V}
		\le
		2^{s+3/2}\delta}
	2\left\langle \epsilon,\hat{g}({\boldsymbol{\lambda}})-\hat{g}(\tilde{\boldsymbol{\lambda}})\right\rangle _{V}
	\ge
	2^{2s} \delta^{2}
	\right)
	& \le 
	a\exp\left(-n_{V}\frac{2^{4s-2}\delta^{4}}{4a^{2}2^{2s+3}\delta^{2}}\right).
	\end{align}
	
	Putting this all together, we have that there is a constant $c$ such that \eqref{eq:train_val_prob} is bounded above by
	\begin{equation}
	c\exp\left(-\frac{n_{V}\delta^{4}}{c^{2} \tilde{R}(X_V|T)}\right)
	+
	c\exp\left(-\frac{n_{V} \delta^2}{c^{2}}\right).
	\end{equation}
	
\end{proof}

We can apply Theorem \ref{thrm:train_val_complicated} to get Theorem \ref{thrm:train_val}. Before proceeding, we determine the entropy of $\mathcal{G}(T)$ when the functions are Lipschitz in the hyper-parameters.

\begin{lemma}
	\label{lemma:covering_cube}
	Let $\Lambda = [\lambda_{\min}, \lambda_{\max}]^J$ where $\lambda_{\min} \le \lambda_{\max}$. Suppose $\mathcal{G}(T)$ is Lipschitz with function $C(\cdot | T)$ over $\boldsymbol{\lambda}$.
	Then the entropy of $\mathcal{G}(T)$ with respect to $\| \cdot \|$ is
	\begin{equation}
	H\left(u, \mathcal{G}(T),\|\cdot\|\right) \le
	J \log \left(\frac{4 \|C(\cdot | T)\| \left(\lambda_{max}-\lambda_{min}\right)+2u}{u}\right).
	\end{equation}
\end{lemma}
\begin{proof}
	Using a slight variation of the proof for Lemma 2.5 in \citet{van2000empirical}, we can show
	\begin{align}
	N\left(u,\Lambda,\|\cdot\|_{2}\right) \le \left(\frac{4\left(\lambda_{max}-\lambda_{min}\right)+2u}{u}\right)^{J}.
	\end{align}
	Under the Lipschitz assumption, a $\delta$-cover for $\Lambda$
	is a $\|C(\cdot | T)\|\delta$-cover for $\mathcal{G}(T)$. The covering number for $\mathcal{G}(T)$ wrt $\|\cdot\|$ is bounded by the covering number for $\Lambda$ as follows
	\begin{eqnarray}
	N\left(u,\mathcal{G}(T),\|\cdot\|\right)
	&\le& N\left(\frac{u}{\|C(\cdot | T)\|},\Lambda,\|\cdot\|_{2}\right)\\
	&\le& \left(\frac{4\left(\lambda_{max}-\lambda_{min}\right)+2u/\|C(\cdot | T)\|}{u/\|C(\cdot | T)\|}\right)^{J}.
	\end{eqnarray}
\end{proof}

\subsubsection{Proof for Theorem \ref{thrm:train_val}}
\begin{proof}
	By Lemma \ref{lemma:covering_cube}, we have
	\begin{align}
	\int_{0}^{R}H^{1/2}(u,\mathcal{G}(T),\|\cdot\|_{V})du 
	&= \int_{0}^{R} \left ( 
	J \log \left(\frac{4 \|C_\Lambda\|_V \Delta_{\Lambda}+2u}{u}\right)
	\right )^{1/2}
	du\\
	& \le J^{1/2}\int_{0}^{R}\left[
	\log\left(
	\frac{4 \|C_\Lambda(\cdot | T)\|_V \Delta_{\Lambda} + 2R }
	{u}
	\right)
	\right]^{1/2}du\\
	& = J^{1/2}R \int_{0}^{1}\left[
	\log\left(
	\frac{4 \|C_\Lambda(\cdot | T)\|_V \Delta_{\Lambda} + 2R }
	{vR}
	\right)
	\right]^{1/2}dv\\
	& \le J^{1/2}R \int_{0}^{1}
	\log^{1/2}\left(
	\frac{4 \|C_\Lambda(\cdot | T)\|_V \Delta_{\Lambda} + 2R}
	{R}
	\right)
	+
	\log^{1/2}(1/v)
	dv\\
	& < J^{1/2}R \left (
	\log^{1/2}\left(
	\frac{4 \|C_\Lambda(\cdot | T)\|_V \Delta_{\Lambda} + 2R}
	{R}
	\right)
	+
	1
	\right ).
	% Show fewer steps maybe
	\end{align}
	If we restrict $R > n^{-1}$, then for an absolute constant $c$, we have
	\begin{equation}
	\label{eq:train_val_entropy}
	\int_{0}^{R}H^{1/2}(u,\mathcal{G}(T),\|\cdot\|_{V})du
	\le
	\mathcal{J}(R) 
	\coloneqq c R\left ( J \log(\|C_\Lambda(\cdot |T)\|_V \Delta_{\Lambda} n + 1) \right )^{1/2}.
	\end{equation}
	Applying Theorem \ref{thrm:train_val_complicated}, we get our desired result.
\end{proof}

\subsection{Cross-validation}
\label{app:cv}
As mentioned before, Theorem \ref{thrm:kfold} is an application of Theorem \ref{thrm:jean_cv}, which extends Theorem 3.5 in \citet{lecue2012oracle}.
\textred{
\textbf{Is this stuff even true still?}
Theorem 3.5 in \citet{lecue2012oracle} assumes that there is a single function $\mathcal{J}$ that bounds the complexity of the post-training model class $\mathcal{G}(T)$.
However the complexity of the post-training model class depends on training data -- for instance, if the noise in the training data were particularly large, the complexity of the post-training model class can be very high.
In our extension, we rely on the fact that the noise in the dataset are sub-gaussian, so that the complexity of the post-training model class can be controlled with high probability.
}

For this section, suppose we have a measurable space $(\mathcal{Z}, \mathcal{T})$ and $\mathcal{F}$ is a class of measurable functions from $\mathcal{Z} \mapsto \mathbb{R}$.
We consider a general, convex loss function $Q: \mathcal{Z} \times \mathcal{F} \mapsto \mathbb{R}$ (rather than the least squares loss considered in the previous section).
The model-estimation procedure selects functions from the class $\mathcal{F}$.
The risk function $R$ is defined as...

Our theorem again begins with a result reminiscent of the basic inequality.
The following lemma is Lemma 3.1 from \citet{lecue2012oracle}.

From henceforth, $c_i > 0$ denotes absolute constants, that may not necessarily be the same if they share the same subscript.
\begin{lemma}
	lemma 3.1
\end{lemma}

We need to bound the supremum of the shifted empirical process term.
To do this, we again use techniques from empirical process theory.
Instead of using metric entropy from the previous section, we use Talagrand's $\gamma$ function, which bounds supremum of stochastic processes using a generic chaining mechanism.
The $\gamma$ function is defined as follows
\begin{definition}
	admissible sequences...
	talagrand...
\end{definition}

To prove our result, we first restate Lemma 3.4 in \citet{lecue2012oracle} using notation that clarifies the conditional dependencies.
This allows us to introduce two new functions $h$ and $J_\delta$.
\begin{lemma}
Let $\mathcal{Q}(D^{(m)})\equiv\left\{ Q(\lambda|D^{(m)}):\lambda\in\Lambda\right\} $
and $\mathcal{Q}\equiv\cup_{m\in\mathbb{N}}\cup_{D^{(m)}}\mathcal{Q}(D^{(m)})$.
Suppose there exists $C_{1}>0$ and increasing function $G(\cdot)$
such that $\forall Q\in\mathcal{Q}$, 
\[
\|Q(Z)\|_{L_{2}}\le G\left(\mathbb{P}Q(Z)\right).
\]
Let $n_{T},n_{V}\in\mathbb{N}$.
Suppose there exists a function $h$ that maps training data $D^{(n_T)}$ to $\mathbb{R}^+$,
functions $J_\delta :\mathbb{R}^+ \mapsto \mathbb{R}^+$ indexed by $\delta > 0$,
and a constant $w_{\min}>0$ such that for any dataset $D^{(n_{T})}$ and any $w \ge w_{\min}$,
\begin{align}
h(D^{(n_{T})})\le\delta\implies\frac{\log n_{V}}{\sqrt{n_{V}}}\gamma_{1}\left(\mathcal{Q}_{w}^{L_{2}}(D^{(n_{T})}),\|\cdot\|_{L_{\psi_{1}}}\right)+\gamma_{2}\left(\mathcal{Q}_{w}^{L_{2}}(D^{(n_{T})}),\|\cdot\|_{L_{2}}\right)\le J_{\delta}(w)
\end{align}
where $\mathcal{Q}_{w}^{L_{2}}(D^{(n_{T})})\equiv\left\{ Q\in\mathcal{Q}(D^{(n_{T})}):\|Q(Z)\|_{L_{2}}\le G(w)\right\}$.

Then there exists absolute constants $L,c>0$ such that for all
$w\ge w_{\min}$ and all $u\ge1$,
\begin{align}
\Pr\left(
\sup_{Q\in\mathcal{Q}(D^{(n_{T})}): PQ \le w}
\left(\left(\mathbb{P}-P_{n_{V}}\right)Q\right)_{+}
\le uL\frac{J_{\delta}(w)}{\sqrt{n_{V}}}
\middle | h\left(D^{(n_{T})}\right)
\le \delta
\right)
\ge
1-L\exp(-cu).
\end{align}
\end{lemma}

Next we restate Lemma 3.2 in \citet{lecue2012oracle} by clarifying the conditional dependencies and using our new functions $h$ and $J_\delta$.
\begin{lemma}
	Let $a>0$. Let $\mathcal{Q}(D^{(m)})\equiv\left\{ Q(\lambda|D^{(m)}):\lambda\in\Lambda\right\} $
	be a set of measurable functions.
	
	Let random variable $Z$ satisfy for all $m\in\mathbb{N}$, any dataset
	$D^{(m)}$, for all $Q\in\mathcal{Q}\left(D^{(m)}\right)$, $\mathbb{P}Q(Z)\ge 0 $.
	
	Suppose for any $n_{T},n_{V}\in\mathbb{N}$ and dataset $D^{(n_{T})}$
	there exists some absolute constant $L,c>0$ such that for all $w\ge w_{\min}$
	and for all $u\ge1$,
	\[
	\Pr\left(
	\sup_{Q\in\mathcal{Q}(D^{(n_{T})}): PQ \le w}
	\left(\left(\mathbb{P}-P_{n_{V}}\right)Q\right)_{+}\le uL\frac{J_{\delta}(w)}{\sqrt{n_{V}}}\mid h\left(D^{(n_{T})}\right)\le\delta\right)
	\ge 1-L\exp(-cu).
	\]
	
	
	Suppose every function in $\left\{ J_{\delta}:\delta>0\right\} $
	is strictly increasing and its inverse is strictly convex.
	Let $\psi_{\delta}$ be the convex conjugate of $J_{\delta}^{-1}$,
	e.g. $\psi_{\delta}(u)=\sup_{v>0}uv-J_{\delta}^{-1}(v)$ for all $u>0$.
	Assume there is a $r\ge1$ such that $x>0\mapsto\psi(x)/x^{r}$ decreases
	and define for $q>1$ and $u\ge1$,
	\[
	\tilde \psi_{q,\delta}(u)=\psi_{\delta}\left(\frac{2q^{r+1}(1+a)u}{a\sqrt{n_{V}}}\right)\vee w_{\min}.
	\]
	
	
	Then there exists a constant $L_{1}$ that only depends on $L$ such
	that for every $u\ge1$,
	\[
	\Pr\left(
	\sup_{Q\in\mathcal{Q}(D^{(n_{T})})}
	\left(\left(\mathbb{P}-(1+a)P_{n_{V}}\right)Q\right)_{+}
	\le \frac{a \tilde{\psi}_{q,\delta}(u/q)}{q}\mid h\left(D^{(n_{T})}\right)\le\delta
	\right)
	\ge 1-L_{1}\exp(-cu).
	\]
	
	
	Moreover, assume that $\psi_{\delta}$ increases such that $\psi_{\delta}(\infty)=\infty$.
	Then there exists a constant $c_{1}$ that depends only on $L$ and $c$ such that
	\[
	\mathbb{P}\left[\sup_{Q\in\mathcal{Q}(D^{(n_{T})})}
	\left(\left(\mathbb{P}-(1+a)P_{n_{V}}\right)Q\right)_{+}
	\middle |
	h\left(D^{(n_{T})}\right)\le\delta
	\right]
	\le\frac{ac_{1} \tilde{\psi}_{q,\delta}(1/q)}{q}.
	\]
\end{lemma}

Now we state the following lemma, that will allow us to extend of Theorem 3.5 in \citet{lecue2012oracle}.
Using a simple chaining argument, we arrive at the following lemma.
\begin{lemma}
	\label{lemma:chain}
Consider any $a>0$.
Suppose there exists a constant $c_{1}$ such that for any $n_{T},n_{V} \in \mathbb{N}$, $\delta>0$,
and $q>1$, we have
\[
\mathbb{P}\left[
\sup_{Q\in\mathcal{Q}(D^{(n_{T})})}\left(\left(\mathbb{P}-(1+a)P_{n_{V}}\right)Q\right)_{+}
\middle | 
h\left(D^{(n_{T})}\right)\le\delta\right]
\le\frac{ac_{1}\epsilon_{q,\delta}(1/q)}{q}.
\]
Then for any $\sigma > 0$, we have
\[
\mathbb{P}\left[\sup_{Q\in\mathcal{Q}(D^{(n_{T})})}\left(\left(\mathbb{P}-(1+a)P_{n_{V}}\right)Q\right)_{+}\right]
\le
\frac{ac}{q}
\left(
\tilde{\psi}_{q,2\sigma}(1/q)
+\sum_{k=1}^{\infty}\Pr\left(h\left(D^{(n_{T})}\right)\ge2^{k}\sigma\right)
\tilde{\psi}_{q,2^{k}\sigma}(1/q)
\right)
.
\]
\end{lemma}
Obviously the above lemma is only useful if we can show that $h$ is bounded with high probability.
For instance, in our examples later, $h$ has exponential tails so the upper bound in Lemma~\ref{lemma:chain} is well-controlled.

We require the following assumption for the more general case.
\begin{assump}
	%JF renumber this to assumption 4?
	\label{assump:tail_margin_general}
	There exist constants $K_0, K_1 \ge 0$ and $\kappa \ge 1$ such that for any $m \in \mathbb{N}$ and any dataset $D^{(m)}$,
	\begin{align}
	\left \| Q(\cdot, \hat{g}^{(n_T)}(\boldsymbol{\lambda} | D^{(n_T)}) - Q(\cdot, g^*) \right \|_{L_{\psi_1}} & \le K_0
	\label{eq:cv_assump1}\\
	\left \| Q(\cdot, \hat{g}^{(n_T)}(\boldsymbol{\lambda} | D^{(n_T)}) - Q(\cdot, g^*)  \right \|_{L_2}
	& \le 
	K_1 \left ( R(\hat{g}^{(m)}(\boldsymbol{\lambda}|D^{(m)})) - R(g^*) \right )^{1/2\kappa}.
	\label{eq:cv_assump2}
	\end{align}
\end{assump}

Putting the three lemmas above together, we have the following result.

\begin{theorem}
	\label{thrm:jean_cv}
	Consider a set of hyper-parameters $\Lambda$. Consider a loss function $Q:(\mathcal{Z}, \mathcal{G}) \mapsto \mathbb{R}$ with convex risk function $R: \mathcal{G} \mapsto \mathbb{R}$. Let
	$$
	\mathcal{Q} = \{ 
	Q(\cdot, \hat{g}^{(n_T)}(\boldsymbol{\lambda} | D^{(n_T)}) - Q(\cdot, g^*) : \boldsymbol{\lambda} \in \Lambda \}.
	$$
	
	Suppose Assumption~\ref{assump:tail_margin_general} holds.
	Suppose there is an $w_{\min} > 0$ and
	% JF: is this the right notation for a training set?
	functions $h: {\mathcal{Z}}^{(n_T)} \mapsto \mathbb{R}$
	and $\mathcal{J}_\delta: \mathbb{R}\mapsto \mathbb{R}$ such that
	for all $w \ge w_{\min}$,
	\begin{align}
	\label{eq:h_to_J}
	h(D^{(n_{T})})\le\delta\implies
	\frac{\log n_{V}}{\sqrt{n_{V}}}
	\gamma_{1}\left(\mathcal{Q}_{w}^{L_{2}}(D^{(n_{T})}),\|\cdot\|_{L_{\psi_{1}}}\right)
	+\gamma_{2}\left(\mathcal{Q}_{w}^{L_{2}}(D^{(n_{T})}),\|\cdot\|_{L_{2}}\right)\le \mathcal{J}_{\delta}(w)
	\end{align}
	where $\mathcal{Q}_w = \{Q \in \mathcal{Q}: \| Q \|_{L_2} \le w^{1/2\kappa} \}$.
	Moreover, suppose that for all $\delta > 0$, $J_\delta$ is a strictly increasing function, $\mathcal{J}_\delta^{-1}(\epsilon)$ is strictly convex,
	the convex conjugate $\psi_\delta$ of $\mathcal{J}^{-1}_\delta$ increases, $\psi_\delta(\infty ) = \infty$ and there exists $r \ge 1$ such that $\psi_\delta(x)/x^r$ decreases.
	
	Consider any $\sigma > 0$. Then there is an absolute constant $c > 0$ such that for every $a > 0$ and $q > 1$, the following inequality holds
	\begin{align}
	\begin{split}
	E_{D^{(n)}} 
	\left(
	R\left(\bar{g} ( \hat{\boldsymbol \lambda} | {D^{(n)}} ) \right )
	- R (g^*)
	\right)
	&\le
	(1+a) \inf_{\boldsymbol{\lambda} \in \Lambda} 
	E_{D^{(n_T)}}
	\left(
	R\left(\bar{g} ( \hat{\boldsymbol \lambda} | {D^{(n)}} ) \right )
	- R (g^*)
	\right)
	\\
	& +
	\frac{ac}{q}
	\left(
	\tilde{\psi}_{q,2\sigma}(1/q)
	+\sum_{k=1}^{\infty}
	\Pr\left(h\left(D^{(n_{T})}\right)\ge2^{k}\sigma\right)
	\tilde{\psi}_{q,2^{k}\sigma}(1/q)
	\right).
	\label{eq:cv_oracle_ineq}
	\end{split}
	\end{align}
	where $\tilde{\psi}_{q, \delta}(u) = \psi_\delta\left(\frac{2q^{r+1}(1 + a)u}{a\sqrt{n_V}}\right) \vee w_{\min}$ for all $u > 0$.
\end{theorem}

To see why this extension of Theorem 3.5 in \citet{lecue2012oracle} was useful, we now apply it to prove Theorem~\ref{thrm:kfold} where we consider the squared error loss $Q((x,y), g) = (y - g(x))^2$ and model-estimation methods where the estimated functions are Lipschitz in the hyper-parameters.

First we need the following lemma that describes the relationship between Lipschitz functions
\begin{lemma}
	Suppose the same conditions as Theorem~\ref{thrm:jean_cv}.
	Suppose Assumptions~\ref{assump:lipschitz} and \ref{assump:tail_margin} hold.
	Also suppose that $\|\epsilon\|_{L_{\psi_2}} = b <\infty$.
	Define 
	$\mathcal{Q}_{w}^{L_{2}} = \{g^* - \hat{g}(\boldsymbol{\lambda}|D^{(n_{T})}) : P (g^* - \hat{g}(\boldsymbol{\lambda}|D^{(n_{T})}))^2 < w\}$
	for $w > 0$.
	Then there is an absolute constant $c_0 > 0$ such that
	\begin{align}
	N\left(\mathcal{Q}_{w}^{L_{2}}(D^{(n_{T})}),u,\|\cdot\|_{L_{2}}\right)\le N\left(\Lambda,\frac{u}{c_0 \left(b +\sqrt{w}\right)
		\|C_\Lambda(x|D^{(n_{T})})\|_{L_{2}}},\|\cdot\|_{2}\right).
	\end{align}
	then we also have
	\begin{align}
	N\left(\mathcal{Q}_{w}^{L_{2}}(D^{(n_{T})}),u,\|\cdot\|_{L_{\psi_{1}}}\right)
	\le N\left(
		\Lambda,
		\frac{u}{c_{K_0, b}\|C_\Lambda(x|D^{(n_{T})})\|_{L_{\psi_{2}}}},\|\cdot\|_{2}\right).
	\end{align}
	\label{lemma:covering_lipschitz}
\end{lemma}
\begin{proof}
Let us first consider a general norm $\|\cdot \|$ such that for any random variables $X, Y$, we have $\|XY\| \le \|X\|_* \|Y\|_*$.
Then for all $\boldsymbol{\lambda} \in \Lambda$ such that
$P (g^* - \hat{g}(\boldsymbol{\lambda} | D^{n_T}))^2 \le w$, we have
\begin{align}
& \left \|
Q(\cdot , \hat{g}^{(n_T)}(\boldsymbol{\lambda}^{(1)}|D^{(n_T)})(x))
- Q(\cdot , \hat{g}^{(n_T)}(\boldsymbol{\lambda}^{(2)}|D^{(n_T)})(x))
\right \|\\
& = \left \|
\left (y - \hat{g}^{(n_T)}(\boldsymbol{\lambda}^{(1)}|D^{(n_T)})(x) \right) ^2 - \left (y - g^*(x) \right) ^2
- \left (y - \hat{g}^{(n_T)}(\boldsymbol{\lambda}^{(2)}|D^{(n_T)})(x) \right) ^2 - \left (y - g^*(x) \right) ^2
\right \|
\label{eq:loss_diff}
\\
%& = \|
%\left (y - \hat{g}^{(n_T)}(\boldsymbol{\lambda}^{(2)}|D^{(n_T)})(x) \right)
%\left(\hat{g}^{(n_T)}(\boldsymbol{\lambda}^{(2)}|D^{(n_T)})(x) - \hat{g}^{(n_T)}(\boldsymbol{\lambda}^{(1)}|D^{(n_T)})(x)\right ) \\ & - \left(\hat{g}^{(n_T)}(\boldsymbol{\lambda}^{(2)}|D^{(n_T)})(x) - \hat{g}^{(n_T)}(\boldsymbol{\lambda}^{(1)}|D^{(n_T)})(x)\right )^2 \|\\
%%JF falling off page
& \le
\left \|2\epsilon + g^*(x) - \hat{g}(\boldsymbol{\lambda}^{(1)} | D^{(n_T)})(x)
+ g^*(x) - \hat{g}(\boldsymbol{\lambda}^{(2)} | D^{(n_T)})(x) \right \|_* \\
& \quad \left \| \hat{g}^{(n_T)}(\boldsymbol{\lambda}^{(2)}|D^{(n_T)})(x) - \hat{g}^{(n_T)}(\boldsymbol{\lambda}^{(1)}|D^{(n_T)})(x) \right \|_* \\
& \le  \left (2 \|\epsilon\|_* +
2 \sup_{\lambda \in \Lambda: P(g^* - \hat{g}(\boldsymbol{\lambda} | D^{n_T}))^2 \le w} 
\left \| g^*(x) - \hat{g}(\boldsymbol{\lambda}^{(1)} | D^{(n_T)})(x) \right \|_* 
\right)
\left \|C_\Lambda (x | D^{(n_T)}) \right \|_*
\|\boldsymbol{\lambda}^{(2)} - \boldsymbol{\lambda}^{(1)} \|_2
\label{eq:lipschitz_connect}
\end{align}

For $\|\cdot \| = \|\cdot\|_{L_{2}}$, the $L_2$ norm is its own dual norm so \eqref{eq:lipschitz_connect} reduces to
$$
c_0 \left(b +\sqrt{w}\right)
\|C_\Lambda (x | D^{(n_T)})\|_{L_{2}}\|\boldsymbol{\lambda}^{(1)}-\boldsymbol{\lambda}^{(2)}\|_{2}$$
for an absolute constant $c_0 > 0$.

For $\|\cdot \| = \|\cdot\|_{L_{\psi_{1}}}$, the dual of the $L_{\psi_1}$ norm is $L_{\psi_2}$.
Thus applying Assumption~\ref{assump:tail_margin} and the fact that $\|\epsilon\|_{L_{\psi_2}} = b < \infty$, \eqref{eq:lipschitz_connect} reduces to 
$$
2\left(
b
+ K_0
\right)
\|C_\Lambda (x | D^{(n_T)})\|_{L_{\psi_{2}}}
\|\boldsymbol{\lambda}^{(1)}-\boldsymbol{\lambda}^{(2)}\|_{2}.
$$

\end{proof}

Now we proceed to bound Talagrand's $\gamma$ function. Via simple calculations, one can show that there is an absolute constant $c > 0$ such that
\begin{align}
\gamma_{\alpha}(T,D)\le c\int_{0}^{\text{Diam}(T,d)}\left(\log N(T,\epsilon,d)\right)^{1/\alpha}d\epsilon
\label{eq:bound_gamma}
\end{align}
(cite talgrand).
Combining the above bound with Lemma~\ref{lemma:covering_lipschitz} gives us the following lemma.

\begin{lemma}
Suppose Assumptions~\ref{assump:tail_margin} and \ref{assump:lipschitz} hold.
Suppose $\|\epsilon\|_{L_{\psi_2}} = b < \infty$.
Define $\mathcal{Q}_w^{L_2}$ as before.
For $\Lambda$, let $\Delta_{\Lambda}=(\lambda_{\max}-\lambda_{\min}) \vee 1$.
Let $w>0$.
Let $\mathcal{Q}_{w}^{L_{2}}(D^{(n_{T})})$ be defined as before.

Then there exist absolute constants $c_0, c_1 >0$ and a constant $c_{K_0, b} > 0$ such that
\begin{align}
\gamma_{2}\left(\mathcal{Q}_{w}^{L_{2}}(D^{(n_{T})}),\|\cdot\|_{L_{2}}\right)
& \le	c_0 \sqrt{w J}
\left[\sqrt{
\log\left(
\left (\frac{b}{\sqrt{w}} + 1 \right )
\Delta_{\Lambda}\|C_\Lambda(x|D^{(n_{T})})\|_{L_{2}} + 1
\right)
}
+1\right]\\
\gamma_{1}\left(\mathcal{Q}_{w}^{L_{2}}(D^{(n_{T})}),\|\cdot\|_{L_{\psi_{1}}}\right)
& \le c_{1}JK_0\left[
\log\left(
	\Delta_{\Lambda}\|C_\Lambda(x|D^{(n_{T})})\|_{L_{\psi_{2}}} c_{K_0, b} +1
\right)
+1\right].
\end{align}
%JF note that c_{K_0, b} is not the same constant as before it
\end{lemma}

\begin{proof}
	By definition of $\mathcal{Q}_{w}^{L_{2}}$, we have $\Diam\left(\mathcal{Q}_{w}^{L_{2}}(D^{(n_{T})}),\|\cdot\|_{L_{2}}\right)	=	2\sqrt{w}.$
	Using Lemma~\ref{lemma:covering_lipschitz} and \eqref{eq:bound_gamma}, we have
	\begin{align}
	\gamma_{2}\left(\mathcal{Q}_{w}^{L_{2}}(D^{(n_{T})}),\|\cdot\|_{L_{2}}\right)
	& \le	c\int_{0}^{2\sqrt{w}}\sqrt{\log N\left(\mathcal{Q}_{w}^{L_{2}}(D^{(n_{T})}),u,\|\cdot\|_{L_{2}}\right)}du \\
	& \le	c\int_{0}^{2\sqrt{w}}\sqrt{\log N\left(\Lambda,\frac{u}{c_0 \left(b +\sqrt{w}\right)\|C_\Lambda(x|D^{(n_T)})\|_{L_{2}}},\|\cdot\|_{2}\right)}du \\
	& \le	c\int_{0}^{2\sqrt{w}}\sqrt{J\log\left(\frac{4 c_0 \Delta_{\Lambda}\left(b +\sqrt{w}\right)\|C_\Lambda(x|D^{(n_T)})\|_{L_{2}}+2u}{u}\right)}du\\
%	& \le	c\sqrt{J}\int_{0}^{2\sqrt{w}}\sqrt{\log\left(\frac{8\Delta_{\Lambda}\left(\|\epsilon\|_{L_{2}}+\sqrt{w}\right)\|C_\Lambda(x|D^{(n_T)})\|_{L_{2}}+4\sqrt{w}}{u}\right)}du \\
%	& =	2c\sqrt{\epsilon J}\int_{0}^{1}\sqrt{\log\left(\frac{8\Delta_{\Lambda}\left(\|\epsilon\|_{L_{2}}+\sqrt{w}\right)\|C_\Lambda(x|D^{(n_T)})\|_{L_{2}}+4\sqrt{w}}{2\sqrt{w}v}\right)}dv\\
%	& \le	2c\sqrt{w J}\int_{0}^{1}\sqrt{\log\left(\frac{8\Delta_{\Lambda}\left(\|\epsilon\|_{L_{2}}+\sqrt{w}\right)\|C_\Lambda(x|D^{(n_T)})\|_{L_{2}}+4\sqrt{w}}{2\sqrt{w}}\right)}+\sqrt{\log\frac{1}{v}}dv\\
	& \le	2c\sqrt{w J}\left[\sqrt{\log\left(\frac{4 c_0  \Delta_{\Lambda}\left(b +\sqrt{w}\right)\|C_\Lambda(x|D^{(n_T)})\|_{L_{2}}+4\sqrt{w}}{2\sqrt{w}}\right)}+\frac{\sqrt{\pi}}{2}\right]
	\end{align}
	Using very similar logic, we now bound the $\gamma_1$ function.
	First we bound the diameter of $\mathcal{Q}_w^{L_2}$ with respect to the norm $\|\cdot \|_{L_{\psi_1}}$.
	\begin{align}
	\Diam(\mathcal{Q}_w^{L_2}(D^{(n_T)}), \|\cdot \|_{L_{\psi_1}})
	& \le 2 \sup_{\boldsymbol{\lambda} \in \Lambda}
	\left \|
	\left(
	y - \hat{g}^{(n_T)}(\boldsymbol{\lambda}| D^{(n_T)})
	\right)^2
	-
	\left(
	y - g^*(x)
	\right)^2
	\right \|_{L_{\psi_1}}\\
	& \le
	c_1 (b^2 + K_0^2).
	\label{eq:diam_psi1}
	\end{align}
	Thus
	\begin{align}
	\gamma_{1}\left(\mathcal{Q}_{w}^{L_{2}}(D^{(n_{T})}),\|\cdot\|_{L_{\psi_{1}}}\right)
	&\le c\int_{0}^{c_1 (b^2 + K_0^2)}\log N\left(\mathcal{Q}_{w}^{L_{2}}(D^{(n_{T})}),u,\|\cdot\|_{L_{\psi_{1}}}\right)du\\
	& \le c_2 J (b^2 + K_0^2)\left[
	\log\left(
	\frac{4 \Delta_{\Lambda}c_{K_0, b} \|C_\Lambda(x|D^{(n_T)})\|_{L_{\psi_{2}}}+ 2 c_1(b^2 + K_0^2)}
	{c_1(b^2 + K_0^2)}
	\right)+1\right]
	\end{align}
\end{proof}

Using the above lemma, we can now define our functions $h$ and $\mathcal{J}_\delta$.
Let
\begin{align}
h(D^{(n_T)}) = \|C_\Lambda(x|D^{(n_T)})\|_{L_{\psi_2}}.
\end{align}
and
\begin{align}
\begin{split}
\label{eq:j_delta}
\mathcal{J}_{\delta}(w) & =
c_{1} \frac{\log n_{V}}{\sqrt{n_{V}}}
JK_0\left[\log\left(\Delta_{\Lambda}\delta c_{K_0, b}+1\right)+1\right]
+c_{3}\sqrt{Jw}
\left[\sqrt{\log\left(\Delta_{\Lambda}b \delta n +1\right)}+1\right].
\end{split}
\end{align}

Finally using the results above, we can prove Theorem~\ref{thrm:kfold}.
\begin{proof}[Proof for Theorem~\ref{thrm:kfold}]
	We now apply Theorem~\ref{thrm:jean_cv} to our Lipschitz case.
	From \eqref{eq:diam_psi1}, we find that Assumption~\ref{assump:tail_margin_general} is satisfied.
	As defined in \eqref{eq:j_delta}, $J_\delta(w)$ is clearly a valid bound for \eqref{eq:h_to_J} for all $w \ge 1/n$.
	Moreover, $\mathcal{J}_{\delta}(w)$ is strictly increasing and concave in $w$.
	This implies that $\mathcal{J}_{\delta}^{-1}$ is strictly convex.
	Also by definition,
	\begin{align}
	\psi_{\delta}(u)
	= c_{1} u \frac{\log n_{V}}{\sqrt{n_{V}}}
	JK_0\left[\log\left(\Delta_{\Lambda}\delta c_{K_0, b}+1\right)+1\right]
	+ u^2 c_{4} J
	\left[\sqrt{\log\left(\Delta_{\Lambda}b \delta n +1\right)}+1\right]^2.
	\end{align}
%	Using the fact that
%	\begin{align}
%	\log(w+1)\le\left(\log w+1\right)1\left\{ w\ge1\right\} +c1\left\{ w<1\right\}
%	\label{eq:log_bound}
%	\end{align}
%	for all $w>0$ and $c\ge\log2$,
%	then
%	\begin{align}
%	\psi_{\delta}(u)&=\sup_{v>0}uv-\mathcal{J}_{\delta}^{-1}(v)\\
%	&= \sup_{w>0}u\mathcal{J}_{\delta}(w)-w\\
%	&= uA_{0}+\sup_{w>0}\left\{ uA_{1}\left(\left(\log w+1\right)1\left\{ w\ge1\right\} +c1\left\{ w<1\right\} \right)+uA_{2}\sqrt{w}-w\right\} \\
%	&\le uA_{0}+\max\left\{
%	\sup_{w>0} \left(uA_{1}\left(\log w+1\right)+uA_{2}\sqrt{w}-w\right),
%	\sup_{w>0} \left(uA_{1}c+uA_{2}\sqrt{w}-w \right)
%	\right\} 
%	\end{align}
%	We can easily solve for the first term in the $\max$ by setting its derivative to zero.
%	It is maximized at 
%	$$
%	\tilde w = \frac{uA_{2}}{4}+\sqrt{\left(\frac{uA_{2}}{4}\right)^{2}+uA_{1}}.
%	% \le\frac{uA_{2}}{2}+\sqrt{uA_{1}}
%	$$
%	Thus the first term in the $\max$ is at most
%	$$
%	\sup_{w>0} \left(uA_{1}\left(\log w+1\right)+uA_{2}\sqrt{w}-w\right)
%	\le
%	uA_{1}\log\tilde{w}+\left(\frac{uA_{2}}{2}\right)^{2}.
%	$$
%	The second term in the $\max$ is equal to
%	$
%	uA_{1}c+\left(\frac{uA_{2}}{2}\right)^{2}.
%	$
%	So as long as we choose $c\ge \log\left(\frac{uA_{2}}{2}+\sqrt{uA_{1}}\right)\vee\log2$ in \eqref{eq:log_bound},
%	then
%	\begin{align}
%	\psi_{\delta}(u)
%	& \le u\left[
%	A_{0}+
%	A_{1} \log\left(\frac{uA_{2}}{2}+\sqrt{uA_{1}}\right)
%	\right]
%	+\left(\frac{u}{2}A_{2}\right)^{2}\\
%	& \le c_{2}u\frac{\log n_{V}}{\sqrt{n_{V}}}J\tilde{c}\log\left(\tilde{c}J\Delta_{\Lambda}u\delta\log n_{v}+1\right)+c_{3}u^{2}J\left[\log\left(\tilde{c}\Delta_{\Lambda}\delta n_{v}+1\right)+1\right]
%	\end{align}
%	for some constant $\tilde{c}$ that only depends on $r, K, K_2, \|\epsilon\|_{L_{\psi_2}}$.
%	From above, we see that $\psi_{\delta}(u)/u^2$ decreases.
	Now let us determine $\tilde{\psi}_{q, \delta}(1/q)$ as $q \rightarrow 1$.
	We have
	\begin{align}
	\lim_{q\rightarrow1}\tilde{\psi}_{q,\delta}(1/q)
	& = \psi_{\delta}\left(\frac{2(1+a)}{a}\frac{1}{\sqrt{n_{V}}}\right)\vee\frac{1}{n_{V}}\\
	& \le
	c_{5} \left (\frac{1+a}{a} \right )^2 \frac{J\log n_{V}}{n_{V}}
	K_0\left[\log\left(\Delta_{\Lambda}\delta c_{K_0, b} n +1\right)+1\right].
	\end{align}
	Now we are ready to calculate the summation in \eqref{eq:cv_oracle_ineq}.
	\begin{align}
	& \lim_{q \rightarrow 1} \left(
	\tilde{\psi}_{q,2\sigma_0}(1/q)
	+\sum_{k=1}^{\infty}
	\Pr\left(h\left(D^{(n_{T})}\right)\ge2^{k}\sigma\right)
	\tilde{\psi}_{q,2^{k}\sigma_0}(1/q)
	\right)
	\\
	& \le
	c_{6} \left (\frac{1+a}{a} \right )^2 \frac{J\log n_{V}}{n_{V}}
	K_0\left[\log\left(\Delta_{\Lambda} c_{K_0, b} n \sigma_0 +1\right)+1\right]
	\left(
	1 + 
	\sum_{k=1}^{\infty}
	k \Pr\left(\|C_\Lambda(x|D^{(n_{T})})\|_{L_{\psi_{2}}} \ge 2^{k} \sigma_0\right)
	\right)
	\\
	& \le
	c_{6} \left (\frac{1+a}{a} \right )^2 \frac{J\log n_{V}}{n_{V}}
	K_0\left[\log\left(\Delta_{\Lambda} c_{K_0, b} n \sigma_0 +1\right)+1\right]
	\tilde{h}(n_{T}).
	\label{eq:sum_prob_bound}
	\end{align}
	Taking $q \rightarrow 1$ in \eqref{eq:cv_oracle_ineq} and plugging in \eqref{eq:sum_prob_bound} to Theorem~\ref{thrm:jean_cv}, we get our desired result.
\end{proof}

\subsection{Penalized regression for additive models}

We now show that penalized regression problems for additive models satisfies the Lipschitz condition.
\subsubsection{Proof for Lemma \ref{lemma:param_add}}
\begin{proof}
	We will use the notation $\hat{\boldsymbol{\theta}}(\boldsymbol{\lambda}) \coloneqq \hat{\boldsymbol{\theta}}(\boldsymbol{\lambda} | T)$. By the gradient optimality conditions, we have
	\begin{equation}
	\label{eq:grad_opt}
	\left.\nabla_{\theta} \left [
	\frac{1}{2}\left\Vert y-g(\boldsymbol{\theta})\right\Vert _{T}^{2}+\lambda_{j}P_{j}(\boldsymbol{\theta}^{(j)})
	\right ]
	\right|_{\boldsymbol{\theta}=\hat{\boldsymbol{\theta}}(\lambda)}=0.
	\end{equation}
	
	After implicitly differentiating with respect to $\boldsymbol{\lambda}$, we have
	\begin{equation}
	\label{eq:implicit_diff}
	\nabla_{\lambda}\left\{ \left.\nabla_{\theta}
	\left [
	\frac{1}{2}\left\Vert y-g(\boldsymbol{\theta})\right\Vert _{T}^{2}+\lambda_{j}P_{j}(\boldsymbol{\theta}^{(j)})
	\right ]
	\right|_{\boldsymbol{\theta}=\hat{\boldsymbol{\theta}}(\boldsymbol{\lambda})}\right\} =0.
	\end{equation}
	From the product rule and chain rule, we can then write the system of equations in \eqref{eq:implicit_diff} as
	\begin{align}
	\label{eq:param_grad}
	\left . \nabla_{\lambda}\hat{\boldsymbol{\theta}}(\boldsymbol{\lambda})
	\right|_{\boldsymbol{\theta}=\hat{\boldsymbol{\theta}}(\lambda)}
	= -
	\left (
	\left.\nabla_{\theta}^2
	L_T(\boldsymbol{\theta}, \boldsymbol{\lambda})
	\right|_{\boldsymbol{\theta}=\hat{\boldsymbol{\theta}}(\lambda)} 
	\right)^{-1}
	\diag \left \{
	\left.
	\nabla_{\theta^{(j)}}P_{j}(\boldsymbol{\theta}^{(j)})
	\right|_{\boldsymbol{\theta}=\hat{\boldsymbol{\theta}}(\lambda)}
	\right \}_{j=1:J}
	.
	\end{align}
	We can bound the norm of the second term in \eqref{eq:param_grad} by rearranging \eqref{eq:grad_opt} and using the Cauchy-Schwarz inequality:
	\begin{align*}
		\left\Vert \left.\nabla_{\theta^{(j)}}P_{j}(\boldsymbol{\theta}^{(j)})\right|_{\boldsymbol{\theta}=\hat{\boldsymbol{\theta}}(\lambda)}\right\Vert_2
		&
		\le  \frac{1}{\lambda_{min}}\left\Vert y-g(\hat{\boldsymbol{\theta}}(\boldsymbol{\lambda}))\right\Vert _{T}
		\left \|
		\left\Vert
		\nabla_{\theta^{(j)}}g_{j}(x|\boldsymbol{\theta}^{(j)})
		\right\Vert_{2}
		\right \|_T.
	\end{align*}
	Since $g_j$ is Lipschitz by assumption, then
	\begin{align}
	\left\Vert
	\nabla_{\theta^{(j)}}g_{j}(x|\boldsymbol{\theta}^{(j)})
	\right\Vert_{2}
	\le \ell_j(x).
	\end{align}
	Also, by the definition of $\hat{\boldsymbol{\theta}}(\boldsymbol{\lambda})$, we have
	\begin{align}
	\frac{1}{2}\left\Vert y-g(\hat{\boldsymbol{\theta}}(\boldsymbol{\lambda}))\right\Vert _{T}^{2}
	& \le \frac{1}{2}\left\Vert \epsilon \right \Vert_T^2 + C^*_{\Lambda}.
	\end{align}
	Hence
	\begin{align}
	\left\Vert \left.\nabla_{\theta}P_{j}(\boldsymbol{\theta}^{(j)})\right|_{\boldsymbol{\theta}=\hat{\boldsymbol{\theta}}(\lambda)}\right\Vert_2
	& \le \frac{\|\ell_j\|_T}{\lambda_{min}}
	\sqrt{\left\Vert \epsilon \right \Vert_T^2 + 2 C^*_{\Lambda}}.
	\end{align}
	Plugging in the results from above and using the assumption that the Hessian of the objective function has a minimum eigenvalue of $m(T)$, we have for all 
	\begin{align}
	\left .
	\nabla_{\lambda_k}\hat{\boldsymbol{\theta}}^{(j)}(\boldsymbol{\lambda})
	\right|_{\boldsymbol{\theta}=\hat{\boldsymbol{\theta}}(\lambda)}
	& = \boldsymbol{0}
	\text{ if } j\ne k
	\\
	\left \|
	\left .
	\nabla_{\lambda_j}\hat{\boldsymbol{\theta}}^{(j)}(\boldsymbol{\lambda})
	\right|_{\boldsymbol{\theta}=\hat{\boldsymbol{\theta}}(\lambda)}
	\right \|_2
	& = \left \|
	\left .
	\nabla_{\lambda_j}\hat{\boldsymbol{\theta}}(\boldsymbol{\lambda})
	\right|_{\boldsymbol{\theta}=\hat{\boldsymbol{\theta}}(\lambda)}
	\right \|_2
	\\
	& \le \frac{1}{m(T)} \frac{\|\ell_j\|_T}{\lambda_{min}}
	\sqrt{\left\Vert \epsilon \right \Vert_T^2 + 2 C^*_{\Lambda}}.
	\end{align}
	Since the norm of the gradient is bounded, $\hat{\boldsymbol{\theta}}^{(j)}(\boldsymbol{\lambda})$ must be Lipschitz:
	\begin{align}
	\left\Vert \hat{\boldsymbol{\theta}}^{(j)}(\boldsymbol{\lambda}^{(1)})
	-\hat{\boldsymbol{\theta}}^{(j)}(\boldsymbol{\lambda}^{(2)})\right\Vert _{2}
	& \le
	\frac{1}{m(T)} \frac{\|\ell_j\|_T}{\lambda_{min}}
	\sqrt{\left\Vert \epsilon \right \Vert_T^2 + 2 C^*_{\Lambda}}
	\left |{\lambda}^{(1)}_j-{\lambda}^{(2)}_j \right |.
	\label{eq:lipschitz_params}
	\end{align}
	Finally we combine the above results to get
	\begin{align}
	& \left |
	g \left (x \middle | \hat{\boldsymbol{\theta}}(\boldsymbol{\lambda}^{(1)})\right )
	- g \left (x \middle | \hat{\boldsymbol{\theta}}(\boldsymbol{\lambda}^{(2)})\right )
	\right |\\
	& \le
	\sum_{j=1}^J
	\left |
	g_j \left (x \middle | \hat{\boldsymbol{\theta}}(\boldsymbol{\lambda}^{(1)})\right )
	- g_j \left (x \middle | \hat{\boldsymbol{\theta}}(\boldsymbol{\lambda}^{(2)})\right )
	\right | \\
	& \le \sum_{j=1}^J \ell_j(x_j)
	\left\Vert \hat{\boldsymbol{\theta}}^{(j)}(\boldsymbol{\lambda}^{(1)})
	-\hat{\boldsymbol{\theta}}^{(j)}(\boldsymbol{\lambda}^{(2)})\right\Vert _{2}\\
	& \le \sum_{j=1}^J \ell_j(x_j)
	\frac{1}{m(T)} \frac{\|\ell_j\|_T}{\lambda_{min}}
	\sqrt{\left\Vert \epsilon \right \Vert_T^2 + 2 C^*_{\Lambda}}
	\left |{\lambda}^{(1)}_j-{\lambda}^{(2)}_j\right |\\
	& \le
	\frac{1}{m(T) \lambda_{min}}
	\sqrt{
		\left(
		\left\Vert \epsilon \right \Vert_T^2 + 2 C^*_{\Lambda}
		\right)
		\left(
		\sum_{j=1}^J \|\ell_j\|_T^2 \ell_j^2(x_j)
		\right)
	}
	\left \|
	\boldsymbol{\lambda}^{(1)}-\boldsymbol{\lambda}^{(2)}
	\right \|_{2}
	\end{align}
\end{proof}

\subsubsection{Proof for Lemma \ref{lemma:nonsmooth}}

Before proving Lemma~\ref{lemma:nonsmooth}, we need to introduce some notation.
Let $\mathcal{L}(\boldsymbol{\lambda}^{(1)},\boldsymbol{\lambda}^{(2)})$
be the line segment connecting $\boldsymbol{\lambda}^{(1)}$ and $\boldsymbol{\lambda}^{(2)}$.
Let $\mu_{1}(z)$ be the 1-dimensional Lebesgue measure in the direction
of $z$ (so if $z$ is a continuous line segment, $\mu_{1}(z)=\|z\|_{2}$;
if $z$ is composed of multiple line segments $z_{i}$, then $\mu(z)=\sum\mu(z_{i})$).

Before proving the Lipschitz property over all of $\Lambda$, we show that the fitted function is Lipschitz over $\Lambda_{smooth}$.
For convenience, define $\Lambda_{smooth}^{c} \coloneqq \Lambda \setminus \Lambda_{smooth}$.

\begin{lemma}
	Suppose that $g_j(\boldsymbol{\theta})(x)$ satisfies the Lipschitz condition in Lemma~\ref{lemma:param_add}.
	Let $T\equiv D^{(n_{T})}$ be a fixed set of training data. Suppose
	the penalized loss function $L_{T}\left(\boldsymbol{\theta},\boldsymbol{\lambda}\right)$
	has a unique minimizer $\hat{\boldsymbol{\theta}}(\boldsymbol{\lambda}|T)$
	for every $\boldsymbol{\lambda}\in\Lambda$. Let $\boldsymbol{U}_{\lambda}$
	be an orthonormal matrix with columns forming a basis for the differentiable
	space of $L_{T}(\cdot,\boldsymbol{\lambda})$ at $\hat{\boldsymbol{\theta}}(\boldsymbol{\lambda}|T)$.
	Suppose there exists a constant $m(T)>0$ such that the Hessian of
	the penalized training criterion at the minimizer taken with respect
	to the directions in $\boldsymbol{U}_{\lambda}$ satisfies 
	\begin{equation}
	\left._{U_{\lambda}}\nabla_{\theta}^{2}L_{T}(\boldsymbol{\theta},\boldsymbol{\lambda})\right|_{\theta=\hat{\theta}(\boldsymbol{\lambda})}\succeq m(T)\boldsymbol{I}\quad\forall\boldsymbol{\lambda}\in\Lambda
	\end{equation}
	where \textup{$\boldsymbol{I}$ is the identity matrix.}
	Suppose Condition~\ref{condn:nonsmooth1} is satisfied by some $\Lambda_{smooth}\subseteq\Lambda$.
	Define
	\begin{align}
	\Lambda_{ext}=\left\{ (\boldsymbol{\lambda}^{(1)},\boldsymbol{\lambda}^{(2)}):
	\boldsymbol{\lambda}^{(1)},\boldsymbol{\lambda}^{(2)} \in \Lambda,
	\mu_{1}\left(\mathcal{L}(\boldsymbol{\lambda}^{(1)},\boldsymbol{\lambda}^{(2)})\cap\Lambda_{smooth}^{c}\right)>0\right\}.
	\label{eq:lambda_ext}
	\end{align}
	Then any $(\boldsymbol{\lambda}^{(1)},\boldsymbol{\lambda}^{(2)})\in\Lambda_{ext}^{c}$
	satisfies \eqref{eq:param_add_lipschitz}.
	\label{lemma:lipschitz_lambda_ext_c}
\end{lemma}
\begin{proof}
	From Condition 1, every point $\boldsymbol{\lambda}\in\Lambda_{smooth}$
	is the center of a ball $B(\boldsymbol{\lambda})$ with nonzero radius
	where the differentiable space within $B(\boldsymbol{\lambda})$ is
	constant.

	Now consider any $\boldsymbol{\lambda^{(1)}},\boldsymbol{\lambda^{(2)}}\in \Lambda_{ext}$.
	By \eqref{eq:lambda_ext}, there must exist a countable set of points $\cup_{i=1}^{\infty}\boldsymbol{\ell}^{(i)}
	\subset\mathcal{L}(\boldsymbol{\lambda^{(1)}},\boldsymbol{\lambda^{(2)}})$ where
	$\cup_{i=1}^{\infty} \boldsymbol{\ell}^{(i)} \subset \Lambda_{smooth}$,
	$\boldsymbol{\lambda}^{(1)},\boldsymbol{\lambda}^{(2)}\in \cup_{i=1}^{\infty}\boldsymbol{\ell}^{(i)}$,
	and the union of their differentiable neighborhoods cover $\mathcal{L}(\boldsymbol{\lambda^{(1)}},\boldsymbol{\lambda^{(2)}})$
	entirely: 
	\[
	\mathcal{L}\left(\boldsymbol{\lambda^{(1)}},\boldsymbol{\lambda^{(2)}}\right)\subseteq\cup_{i=1}^{\infty}B\left(\boldsymbol{\ell}^{(i)}\right).
	\]
	Consider the intersections of boundaries of the differentiable neighborhoods
	with the line segment: 
	\begin{align}
	P =
	\cup_{i=1}^{\infty}
	\left[
	bd \left (
	B\left(\boldsymbol{\ell}^{(i)}\right)
	\right )
	\cap\mathcal{L}(\boldsymbol{\lambda^{(1)}},\boldsymbol{\lambda^{(2)}})
	\right].
	\end{align}
	Every point $p\in P$ can be expressed as $\alpha_{p}\boldsymbol{\lambda^{(1)}}+(1-\alpha_{p})\boldsymbol{\lambda^{(2)}}$
	for some $\alpha_{p}\in[0,1]$.
	We can order the points in $P$ by increasing $\alpha_{p}$ to get the sequence $\boldsymbol{p}^{(1)},\boldsymbol{p}^{(2)},...$.

	By Condition 1, the differentiable space of the training criterion
	is constant over $\mathcal{L}\left(\boldsymbol{p}^{(i)},\boldsymbol{p}^{(i+1)}\right)$
	since each of these sub-segments are contained in some $B(\boldsymbol{\ell}^{(i)})$
	for $i\in\mathbb{N}$.
	Moreover, the differentiable space over the interior of line segment $\mathcal{L}\left(\boldsymbol{p^{(i)},p^{(i+1)}}\right)$ can be decomposed as the product of differentiable spaces, which we denote as
	\begin{align}
	\Omega_{i}^{(1)} \times ... \times \Omega_{i}^{(J)}.
	\label{eq:diff_space_product}
	\end{align}
	By Condition 1, \eqref{eq:diff_space_product} is also a local optimality space.
	Let $U^{(i,j)}$ be an orthonormal basis of $\Omega_{i}^{(j)}$ for $j = 1,...,J$.
	For each $i$, we can express $\hat{\boldsymbol{\theta}}(\boldsymbol{\lambda}|T)$
	for all $\boldsymbol{\lambda}\in\mbox{Int}\left\{ \mathcal{L}\left(\boldsymbol{p^{(i)},p^{(i+1)}}\right)\right\} $
	as 
	\[
	\hat{\boldsymbol{\theta}}^{(j)}(\boldsymbol{\lambda}|T)
	=U^{(i,j)}\hat{\boldsymbol{\beta}}^{(j)}(\boldsymbol{\lambda}|T)
	\]
	\[
	\hat{\boldsymbol{\beta}}(\boldsymbol{\lambda}|T)
	= \left (
	\begin{matrix}
	\hat{\boldsymbol{\beta}}^{(1)}(\boldsymbol{\lambda}|T)
	& ... &
	\hat{\boldsymbol{\beta}}^{(J)}(\boldsymbol{\lambda}|T)
	\end{matrix}
	\right )
	=\arg\min_{\beta}L_{T}
	\left (
	\{U^{(i,j)}\boldsymbol{\beta}^{(j)} \}_{j=1}^J,
	\boldsymbol{\lambda}
	\right ).
	\]
	We can show that the fitted parameters satisfy the Lipschitz condition \eqref{eq:lipschitz_params} over $\Lambda=\mathcal{L}\left(\boldsymbol{p^{(i)},p^{(i+1)}}\right)$ by using the same proof in Lemma~\ref{lemma:param_add} but taking directional derivatives along the columns of $U^{(i)} = (U^{(i,1)} ... U^{(i,J)})$ instead.
	Then for all $j$ and $i$, we have
	\begin{align}
	\left\Vert
	\boldsymbol{\hat{\beta}}^{(j)}(\boldsymbol{p}^{(i)}|T)
	-\boldsymbol{\hat{\beta}}^{(j)}(\boldsymbol{p}^{(i)}|T)
	\right\Vert _{2}
	\le
	\frac{1}{m(T)} \frac{\|\ell_j\|_T}{\lambda_{min}}
	\sqrt{\left\Vert \epsilon \right \Vert_T^2 + 2 C^*_{\Lambda}}
	\left |p^{(i)}_j-p^{(i+1)}_j\right |.
	\end{align}
	We can sum these inequalities by the triangle inequality:
	\begin{align*}
	\left\Vert
	\hat{\boldsymbol{\theta}}^{(j)}(\boldsymbol{\lambda}^{(1)}|T)
	-\hat{\boldsymbol{\theta}}^{(j)}(\boldsymbol{\lambda}^{(2)}|T)
	\right\Vert _{2}
	& \le
	\sum_{i=1}^{\infty}
	\left \|
	\boldsymbol{\hat{\theta}}^{(j)}(\boldsymbol{p}^{(i)}|T)
	-\boldsymbol{\hat{\theta}}^{(j)}(\boldsymbol{p}^{(i + 1)}|T)
	\right \|_{2}\\
	& \le
	\frac{1}{m(T)} \frac{\|\ell_j\|_T}{\lambda_{min}}
	\sqrt{\left\Vert \epsilon \right \Vert_T^2 + 2 C^*_{\Lambda}}
	\sum_{i=1}^{\infty} 
	\left |
	{p}^{(i)}_j-{p}^{(i+1)}_j
	\right |\\
	& =
	\frac{1}{m(T)} \frac{\|\ell_j\|_T}{\lambda_{min}}
	\sqrt{\left\Vert \epsilon \right \Vert_T^2 + 2 C^*_{\Lambda}}
	\left |{\lambda}^{(1)}_j - {\lambda}^{(2)}_j\right |.
	\end{align*}
	Finally, using the fact that $g_j$ is $\ell_j$-Lipschitz, we have
	\begin{align}
	C_\Lambda(\boldsymbol{x})
	= \frac{\sqrt{\| \epsilon \|_T^2 + 2 C^*_{\Lambda}}}{m(T) \lambda_{min}}
	\sqrt{\sum_{j=1}^J \|\ell_j\|_T^2 \ell_j^2(x_j)}.
	\label{eq:nonsmooth_lipschitz_func}
	\end{align}
\end{proof}

In order to extend the result in Lemma~\ref{lemma:lipschitz_lambda_ext_c} to all of $\Lambda$, we need to show that $\Lambda_{ext}$ is a set with measure zero.
\begin{lemma}
Suppose Condition~\ref{condn:nonsmooth2}.
Then $\mu_{2J}(\Lambda_{ext})=0$ where $\mu_{2J}$ is the Lebesgue measure in $\mathbb{R}^{2J}$ and $\Lambda_{ext}$ was defined in \eqref{eq:lambda_ext}.
\label{lemma:ext_measure_zero}
\end{lemma}

\begin{proof}
	Suppose for contradiction that $\mu_{2J}(\Lambda_{ext})>0$.
	If this is the case, then there exists a ball $B_{r}\left(\left(\boldsymbol{\lambda}^{(1)},\boldsymbol{\lambda}^{(2)}\right)\right)$ contained in $\Lambda_{ext}$ with nonzero radius $r>0$ centered at $\left(\boldsymbol{\lambda}^{(1)},\boldsymbol{\lambda}^{(2)}\right)$
	where $\boldsymbol{\lambda}^{(1)} \ne \boldsymbol{\lambda}^{(2)}$ and
	\begin{align}
	\mu_{1}\left(\mathcal{L}\left(\boldsymbol{\lambda}^{'},\boldsymbol{\lambda}^{''}\right)\cap\Lambda_{smooth}^{c}\right) >0
	\quad \forall\left(\boldsymbol{\lambda}^{'},\boldsymbol{\lambda}^{''}\right)\in B_{r}\left(\left(\boldsymbol{\lambda}^{(1)},\boldsymbol{\lambda}^{(2)}\right)\right).
	\end{align}
	Suppose that $\mu_{1}\left(\mathcal{L}\left(\boldsymbol{\lambda}^{(1)},\boldsymbol{\lambda}^{(2)}\right)\cap\Lambda_{smooth}^{c}\right)=\delta>0$.
	We claim that for a sufficiently small radius $r'$, we also have
	\begin{align}
	\mu_{1}\left(\mathcal{L}\left(\boldsymbol{\lambda}^{'},\boldsymbol{\lambda}^{''}\right)\cap\Lambda_{smooth}^{c}\right)>\delta/2>0
	\quad \forall
	\left(\boldsymbol{\lambda}^{'},\boldsymbol{\lambda}^{''}\right)
	\in B_{r'}\left(\left(\boldsymbol{\lambda}^{(1)},\boldsymbol{\lambda}^{(2)}\right)\right).
	\end{align}
	To see why this claim is true, let us define a monotonically decreasing
	sequence $\left\{ r_{i}\right\} $ where $r_{i} > 0$ for all $i \in \mathbb{N}$ and $\lim_{i\rightarrow\infty}r_{i}=0$.
	By the monotone convergence theorem,
	\begin{align}
	\lim_{i\rightarrow\infty}\inf_{\left(\boldsymbol{\lambda}^{'},\boldsymbol{\lambda}^{''}\right)\in B_{r_{i}}\left(\left(\boldsymbol{\lambda}^{(1)},\boldsymbol{\lambda}^{(2)}\right)\right)}\mu_{1}\left(\mathcal{L}\left(\boldsymbol{\lambda}^{'},\boldsymbol{\lambda}^{''}\right)\cap\Lambda_{smooth}^{c}\right)=\mu_{1}\left(\mathcal{L}\left(\boldsymbol{\lambda}^{(1)},\boldsymbol{\lambda}^{(2)}\right)\cap\Lambda_{smooth}^{c}\right)=\delta>0.
	\end{align}
	By the definition of limits, there is some sufficiently
	large $i'$ such that for $r'\coloneqq r_{i'} > 0$, we have
	\begin{align}
	\inf_{\left(\boldsymbol{\lambda}^{'},\boldsymbol{\lambda}^{''}\right)\in B_{r'}\left(\left(\boldsymbol{\lambda}^{(1)},\boldsymbol{\lambda}^{(2)}\right)\right)}\mu_{1}\left(\mathcal{L}\left(\boldsymbol{\lambda}^{'},\boldsymbol{\lambda}^{''}\right)\cap\Lambda_{smooth}^{c}\right)>\delta/2.
	\end{align}
	Given our ball is non-empty, there exist points $\left(\boldsymbol{\lambda}^{(3)},\boldsymbol{\lambda}^{(4)}\right),\left(\boldsymbol{\lambda}^{(5)},\boldsymbol{\lambda}^{(6)}\right)\in B_{r'}\left(\left(\boldsymbol{\lambda}^{(1)},\boldsymbol{\lambda}^{(2)}\right)\right)$
	where 
	\begin{align}
	{\lambda}_{j}^{(3)} > {\lambda}_{j}^{(5)},
	{\lambda}_{j}^{(4)} > {\lambda}_{j}^{(6)}
	\quad \forall j=1,..,J.
	\end{align}
	For any $\alpha \in (0,1)$, the line
	\begin{align}
	\mathcal{L}_\alpha =
	\mathcal{L}\left(\alpha\boldsymbol{\lambda}^{(3)}+(1-\alpha)\boldsymbol{\lambda}^{(5)},\alpha\boldsymbol{\lambda}^{(4)}+(1-\alpha)\boldsymbol{\lambda}^{(6)}\right)
	\end{align}
	has
	\begin{align}
	\mu_1\left(
	\mathcal{L}_\alpha
	\cap
	\Lambda_{smooth}^{c}
	\right)
	> \delta/2.
	\end{align}
	As the lines $\mathcal{L}_\alpha$ do not intersect for $\alpha \in (0,1)$, then
	\begin{align}
	\mu\left(
	\cup_{\alpha\in[0,1]}
	\left(
	\mathcal{L}_\alpha
	\cap\Lambda_{smooth}^{c}
	\right)\right)
	= \int_{0}^{1} \mu_1\left(
	\mathcal{L}_\alpha
	\cap
	\Lambda_{smooth}^{c}
	\right) d\alpha
	> \delta/2
	\end{align}
	Thus
	\begin{align}
	\mu\left(\Lambda_{smooth}^{c}\right)
	\ge
	\mu\left(
	\cup_{\alpha\in[0,1]}
	\left(
	\mathcal{L}_\alpha
	\cap\Lambda_{smooth}^{c}
	\right)\right)
	>\delta/2.
	\end{align}
	However this is a contradiction of our assumption that $\mu\left(\Lambda_{smooth}^{c}\right)=0$.
\end{proof}

Finally, combining Lemmas~\ref{lemma:lipschitz_lambda_ext_c} and \ref{lemma:ext_measure_zero}, we can show that the Lipschitz condition is satisfied over all of $\Lambda$.

\begin{proof}[Proof for Lemma~\ref{lemma:nonsmooth}]
	Since we already showed Lemma~\ref{lemma:lipschitz_lambda_ext_c}, it suffices to show that the Lipschitz condition is satisfied for any $\boldsymbol{\lambda^{(1)}},\boldsymbol{\lambda^{(2)}}\in\Lambda_{ext}$.
	Lemma~\ref{lemma:ext_measure_zero} states that $\mu_{2J}(\Lambda_{ext}) = 0$, which means that there exists a sequence
	$\left(\boldsymbol{\lambda}^{(1,i)},\boldsymbol{\lambda}^{(2,i)}\right)\in\Lambda_{ext}^{c}$
	such that $\lim_{i\rightarrow\infty}\left(\boldsymbol{\lambda}^{(1,i)},\boldsymbol{\lambda}^{(2,i)}\right)=\left(\boldsymbol{\lambda}^{(1)},\boldsymbol{\lambda}^{(2)}\right)$.
	As $L_T$ is continuous and we have assumed that there exists a unique minimizer of $\hat{\boldsymbol{\theta}}(\boldsymbol{\lambda})$ for all $\boldsymbol{\lambda} \in \Lambda$, then $\hat{\boldsymbol{\theta}}(\boldsymbol{\lambda})$ is continuous in $\boldsymbol{\lambda}$ over all $\Lambda$.
	As $g(\boldsymbol{\theta})(x)$ is also continuous in $\boldsymbol{\theta}$, then for any $\boldsymbol{\lambda}^{(1)},\boldsymbol{\lambda}^{(2)}\in\Lambda$,
	we have
	\begin{align}
		\left |
		g(\hat{\boldsymbol{\theta}}(\boldsymbol{\lambda}^{(1)} | T)(\boldsymbol{x})
		-
		g(\hat{\boldsymbol{\theta}}(\boldsymbol{\lambda}^{(2)} | T)(\boldsymbol{x})
		\right |
		& = \lim_{i\rightarrow\infty}
		\left |
		g(\hat{\boldsymbol{\theta}}(\boldsymbol{\lambda}^{(1,i)} |T))(\boldsymbol{x})
		-
		g(\hat{\boldsymbol{\theta}}(\boldsymbol{\lambda}^{(2,i)} |T))(\boldsymbol{x})
		\right |\\
		& \le \lim_{i\rightarrow\infty}
		C_\Lambda(\boldsymbol{x})
		\|\boldsymbol{\lambda}^{(1,i)}-\boldsymbol{\lambda}^{(2,i)}\|_{2}\\
		& = C_\Lambda(\boldsymbol{x})
		\|\boldsymbol{\lambda}^{(1)}-\boldsymbol{\lambda}^{(2)}\|_{2}
	\end{align}
	where $C_\Lambda(\boldsymbol{x})$ is defined in \eqref{eq:nonsmooth_lipschitz_func}.
\end{proof}

\subsubsection{Proof for Lemma \ref{lemma:nonparam_smooth}}

\begin{proof}
	Let $H_{0} = \left \{
	j:\left\Vert \hat{g}_{j}(\boldsymbol{\lambda}^{(2)}|T)-\hat{g}_{j}(\boldsymbol{\lambda}^{(1)}|T)\right\Vert _{D^{(n)}} \ne 0\,\, \forall j = 1,...,J
	\right \}$.
	\todo{notation is confusing. not for all.}
	For all $j \in H_0$, let 
	\[
	h_{j}=
	\frac{\hat{g}_{j}(\boldsymbol{\lambda}^{(2)}|T)-\hat{g}_{j}(\boldsymbol{\lambda}^{(1)}|T)}{\left\Vert \hat{g}_{j}(\boldsymbol{\lambda}^{(2)}|T)-\hat{g}_{j}(\boldsymbol{\lambda}^{(1)}|T)\right\Vert _{D^{(n)}}}.
	\]
	For notational convenience, let $\hat{g}_{1,j} = \hat{g}_{j}(\boldsymbol{\lambda}^{(1)}|T)$. Consider the optimization problem
	\begin{equation}
	\hat{\boldsymbol{m}}(\boldsymbol{\lambda})=\left\{ \hat{m}_{j}(\boldsymbol{\lambda})\right\} _{j\in H_0}
	=\argmin_{m_{j} \in \mathbb{R}: j\in H_0}
	\frac{1}{2}
	\left \|y-\sum_{j=1}^{J}\left(\hat{g}_{1,j}+m_{j}h_{j}\right) \right \|_{T}^{2}
	+\sum_{j=1}^{J}\lambda_{j}
	P_{j} \left (\hat{g}_{1,j}+m_{j}h_{j} \right ).
	\end{equation}
	\todo{notation is confusing cause m is already defined}
	By the gradient optimality conditions, we have
	\begin{equation}
	\nabla_{m} \left .
	\left[\frac{1}{2}\|y-\sum_{j=1}^{J}\left(\hat{g}_{1,j}+m_{j}h_{j}\right)\|_{T}^{2}+\sum_{j=1}^{J}\lambda_{j}P_{j}(\hat{g}_{1,j}+m_{j}h_{j})\right] \right |_{m=\hat{m}(\lambda)}
	= 0.
	\label{eq:nonparam_grad_opt}
	\end{equation}
	Implicit differentiation with respect to $\boldsymbol{\lambda}$ gives us
	\begin{equation}
	\nabla_\lambda 
	\nabla_m
	\left . \left[
	\frac{1}{2}\|y-\sum_{j=1}^{J}\left(\hat{g}_{1,j}+m_{j}h_{j}\right)\|_{T}^{2}+\sum_{j=1}^{J}\lambda_{j}P_{j}(\hat{g}_{1,j}+m_{j}h_{j})\right] \right |_{m=\hat{m}(\lambda)}
	= 0.
	\label{eq:nonparam_imp_diff}
	\end{equation}
	From the product rule and chain rule, we can write the system of equations from \eqref{eq:nonparam_imp_diff} as
	\begin{align}
	\nabla_{\lambda} \hat{\boldsymbol{m}}(\boldsymbol{\lambda}) = - \left(
	\nabla_{m}^2 L_T(\boldsymbol{m}, \boldsymbol{\lambda})
	\right )^{-1}
	\diag \left \{ \left.
	\frac{\partial}{\partial m_{j}}P_{j}(\hat{g}_{1,j}+m_{j}h_{j})\right|_{m=\hat{m}(\lambda)}
	\right \}_{j=1}^J
	\label{eq:nonparam_grad}
	\end{align}
	where $L_T(\boldsymbol{m}, \boldsymbol{\lambda})$ is the loss in \eqref{eq:nonparam_grad_opt}.

	We now bound the second term in \eqref{eq:nonparam_grad}.
	From \eqref{eq:nonparam_grad_opt} and Cauchy Schwarz, we have for all $k=1,...,J$
	\begin{equation}
	\left|\frac{\partial}{\partial m_{k}}P_{k}(\hat{g}_{1,k}+m_{k}h_{k})\right|_{m=\hat{m}(\lambda)}
	\le 
	\frac{1}{\lambda_{min}}
	\left\Vert y-\sum_{j=1}^{J}\left(\hat{g}_{1,j}+\hat{m}_{j}(\boldsymbol{\lambda})h_{j}\right)
	\right\Vert _{T}\|h_{k}\|_{T}.
	\end{equation}
	From the definition of $h_k$, we know that $\|h_{k}\|_{T} \le \sqrt{\frac{n_{D}}{n_{T}}}$.
	By definition of $\hat{m}(\boldsymbol{\lambda})$ and $\hat{g}_{1}$, we also have
	\begin{align*}
	\frac{1}{2}\left\Vert y-\sum_{j=1}^{J}\left(\hat{g}_{1,j}+\hat{m}_{j}(\boldsymbol{\lambda})h_{j}\right)\right\Vert _{T}^{2}
	\le
	\frac{1}{2}
	\left\Vert y-\sum_{j=1}^{J}\hat{g}_{1,j}\right\Vert _{T}^{2}
	+\sum_{j=1}^{J}\lambda_{j}P_{j}(\hat{g}_{1,j})
	\le \frac{1}{2} \|\epsilon\|_T^2 + C^*_\Lambda.
	\end{align*}
	Hence
	\begin{equation}
	\left|\frac{\partial}{\partial m_{k}}P_{k}(\hat{g}_{1,k}+m_{k}h_{k})\right|_{m=\hat{m}(\lambda)}
	\le
	\frac{1}{\lambda_{min}}
	\sqrt{
		\left(
		\|\epsilon\|_T^2 + 2 C^*_\Lambda
		\right)
		\frac{n_{D}}{n_{T}}
	}.
	\end{equation}
	By \eqref{eq:gateuax}, we know $\nabla_{m}^2 L_T(\boldsymbol{m}, \boldsymbol{\lambda}) \succeq m(T)I$.
	So for all $k$,
	\begin{align}
	\|\nabla_{\lambda}\hat{m}_{k}(\boldsymbol{\lambda})\|_2
	& \le
	\frac{m(T)}{\lambda_{min}}
	\sqrt{
		\left(
		\|\epsilon\|_T^2 + 2 C^*_\Lambda
		\right)
		\frac{n_{D}}{n_{T}}
	}
	\end{align}
	By the mean value inequality and Cauchy Schwarz, we have
	\begin{equation}
	\left|\hat{m}_{k}(\boldsymbol{\lambda}^{(2)})-\hat{m}_{k}(\boldsymbol{\lambda}^{(1)})\right| 
	\le
	\frac{m(T)}{\lambda_{min}}
	\sqrt{
		\left(
		\|\epsilon\|_T^2 + 2 C^*_\Lambda
		\right)
		\frac{n_{D}}{n_{T}}
	}.
	\end{equation}
	By construction,
	$
	\left|
	\hat{m}_k(\boldsymbol{\lambda}^{(2)})-\hat{m}_k(\boldsymbol{\lambda}^{(1)})
	\right|  =
	\left \| 
	\hat{g}_k(\boldsymbol{\lambda}^{(2)}|T)-\hat{g}_k(\boldsymbol{\lambda}^{(1)}|T)
	\right  \|_{D^{(n)}}
	$.
	So we obtain our desired result in \eqref{eq:nonparam_lipshitz_thrm}.
\end{proof}

\subsection{Examples: detailed derivations}

\begin{example}[Multiple Sobolev penalties]
	Since the solution to \eqref{eq:smoothing_spline} must be the sum of natural cubic splines \citep{buja1989linear}, we can parameterize the space using a Reproducing Kernel Hilbert Space with inner product
	\begin{align}
	\langle f, g \rangle = \int_{0}^1 f^{''}(x) g^{''}(x) dx
	\end{align}
	and the reproducing kernel
	\begin{align}
	R(s, t) = st(s \wedge t)
	+ \frac{s + t}{2} (s \wedge t)^2
	+ \frac{1}{3}
	(s \wedge t)^3
	\end{align}
	\citep{heckman2012theory}.
	Then one can instead solve for \eqref{eq:smoothing_spline} over the functions $g$ of the form
	\begin{align}
	g(x_1,..., x_J) = \alpha_0 + \sum_{j=1}^J g_j(x_j)
	\end{align}
	where the functions $g_j$ are split into a linear component and an orthogonal non-linear component
	\begin{align}
	g_j(x_j) = \alpha_{1j} x_j + \sum_{i=1}^n \theta_{ij} R(x_{ij}, x_j).
	\end{align}
	For notational simplicity, we will also denote $\vec{R}(x | D)_{ij} = R(x_{ij}, x_j)$.
	We will also write
	\begin{align}
	g_{j, \perp}(x_j) = \sum_{i=1}^n \theta_{ij} R(x_{ij}, x_j).
	\end{align}
	
	Using this finite-dimensional representation, we find that
	\begin{align}
	\int_{0}^1 \left(g_j^{''}(x)\right)^{2} dx
	= \sum_{u = 1}^n \sum_{v=1}^n \theta_{uj} \theta{vj} R(x_{uj}, x_{vj})
	= \theta_j^\top K_j \theta_j
	\label{eq:sobolev_finite}
	\end{align}
	where the matrix $K_j$ has elements
	$
	K_{j, (u, v)} = R(x_{uj}, x_{vj}).
	$
	Since any $g_j$ with non-zero $\boldsymbol{\theta}_j$ will have a positive Sobolev penalty, then the matrix $K_j$ must be positive definite.
	Using the formulation above, we re-express \eqref{eq:smoothing_spline} as the finite-dimensional problem
	\begin{align}
	\hat{\alpha_0}(\boldsymbol{\lambda}),
	\hat{\boldsymbol{\alpha}_1}(\boldsymbol{\lambda}),
	\hat{\boldsymbol{\theta}}(\boldsymbol{\lambda})
	& = \argmin_{\alpha_0, \boldsymbol{\alpha}_1, \boldsymbol{\theta}}
	\frac{1}{2}
	\left \|
	\boldsymbol{y} -
	\alpha_0 \boldsymbol{1}
	- X \boldsymbol{\alpha}_1
	- K \boldsymbol{\theta}
	\right \|^2
	+
	\frac{1}{2}
	\boldsymbol{\theta}^\top
	\diag \left (
	\left \{
	\lambda_j K_j
	\right \} \right ) \boldsymbol{\theta}.
	\label{eq:matrix_sobolev}
	\end{align}
	where the matrix $K = (K_1 ... K_J)$.
	In order to make the fitted functions $\hat{g}_j$ identifiable, we add the usual constraint that $\sum_{i=1}^n g_j(x_{ij}) = 0$ for all $j$.
	We also assume that $X^\top X$ is nonsingular to ensure that there is a unique $\hat{\alpha}_1$.
	
	The KKT conditions then gives us
	\begin{align}
	\hat{\alpha}_0 &= \frac{1}{n}\sum_{i=1}^n y_i \\
	\hat{\boldsymbol{\alpha}}_1(\boldsymbol{\lambda})
	& = (X^\top X)^{-1} X^\top
	(
	\boldsymbol{y} - \hat{\alpha}_0 \boldsymbol{1}
	- K \hat{\boldsymbol{\theta}}(\boldsymbol{\lambda})
	)
	\label{eq:kkt_sobolev_linear}
	\\
	\begin{split}
	\hat{\boldsymbol{\theta}}(\boldsymbol{\lambda})
	& =
	\diag(K_j^{-1/2})
	\left(
	K^{(1/2)\top}
	P_X^\top
	K^{(1/2)} + \diag(\lambda_j I)
	\right)^{-1}
	K^{(1/2)\top}
	P_X^\top
	(I - \frac{1}{n}\boldsymbol{1} \boldsymbol{1}^\top)
	\boldsymbol{y}
	\end{split}
	\label{eq:kkt_sobolev}
	\end{align}
	where $K^{(1/2)} = (K_1^{1/2} ... K_J^{1/2})$, $I$ is the $n\times n$ identity matrix, and $P_X^\top = I - X (X^\top X)^{-1} X^\top$.
	
	To apply Theorem~\ref{thrm:train_val}, we need to characterize how $\hat{g}(\boldsymbol{\lambda})(\cdot)$ varies with $\boldsymbol{\lambda}$.
	Since we have the closed form solution to \eqref{eq:kkt_sobolev}, we will use it directly to come up with bounds for the Lipschitz factor $C_\Lambda(x | D^{(n_T)})$.
	From \citet{green1993nonparametric}, we know that the value of the cubic $\hat{g}_j$ on the interval $[t_L, t_R]$ can be defined using its values and second derivatives at the ends of the interval.
	Let $h = t_R - t_L$.
	Then the value of the cubic
	\begin{align}
	\begin{split}
	\hat{g}_j(x_j)
	& = \hat{\alpha_{1j}} x_j
	+ \frac{(x_j - t_L) \hat{g}_{j, \perp} (t_R) + (t_R - t) \hat{g}_{j, \perp}(t_L)}{h}\\
	& \quad - \frac{1}{6}(x_j - t_L)(t_R - x_j) \left\{
	\left(
	1 + \frac{x_j - t_L}{h}
	\right) \hat{g}''_{j, \perp}(t_R^+)
	\left(
	1 + \frac{t_R - x_j}{h}
	\right) \hat{g}''_{j, \perp}(t_L^+)
	\right \}.
	\label{eq:interp}
	\end{split}
	\end{align}
	Let $\hat{\boldsymbol{\gamma}}_j$ be the vector of second derivatives of $\hat{g}''_{j, \perp}$ at the training covariates for the $j$th axis.
	Since the fitted functions $\hat{g}_{j, \perp}$ must be natural cubic splines, there is a closed form for $\hat{\boldsymbol{\gamma}}_j$:
	\begin{align}
	\hat{\boldsymbol{\gamma}}_j & = R^{-1}_j Q^\top_j K_j \hat{\boldsymbol{\theta}}_j
	\label{eq:second_deriv}
	\end{align}
	where the matrix $R$ is a banded diagonally dominant matrix and $Q$ is a banded negative-semi-definite matrix defined in \citet{green1993nonparametric}.
	Let $h_j(D^{(n_T)})$ be the smallest distance between the $j$th covariates from the training data.
	Then using the Gershgorin circle theorem, one can show that all the eigenvalues of $R$ are larger than $\frac{1}{3} h_j(D^{(n_T)})$ and all the eigenvalues of $Q$ have magnitudes no greater than $4/h_j(D^{(n_T)})$.
	Thus using \eqref{eq:interp} and \eqref{eq:second_deriv}, we have that
	\begin{align}
	\left \|
	\nabla_{\lambda} \hat{g}_{j, \perp}(\boldsymbol{\lambda})(x_j)
	\right \|_2
	\le
	\frac{c}{h_j(D^{(n_T)})^2}
	\left\|
	\nabla_{\lambda} K_j \hat{\boldsymbol{\theta}}_j(\boldsymbol{\lambda})
	\right \|_2
	\end{align}
	for some absolute constant $c > 0$ (again, assuming validation covariates between 0 and 1).
	
	From \eqref{eq:kkt_sobolev}, we have that
	\begin{align}
	\nabla_{\lambda_\ell} K_j \hat{\boldsymbol{\theta}}_j(\boldsymbol{\lambda})
	& = 
	\left[
	\begin{matrix}
	0 & .. & 0 & K_j^{1/2} & 0 & .. & 0
	\end{matrix}
	\right]
	\left(
	K^{(1/2), \top} P_X^\top K^{(1/2)} + \diag(\lambda_j I_j)
	\right)^{-2}
	K^{(1/2), \top}
	P_X^\top (I - \frac{1}{n} 1 1^\top) y
	\end{align}
	if $\ell = j$.
	Otherwise $\nabla_{\lambda_\ell} K_j \hat{\boldsymbol{\theta}}_j(\boldsymbol{\lambda}) = 0$.
	Thus
	\begin{align}
	\left\|
	\nabla_{\lambda_\ell} K_j \hat{\boldsymbol{\theta}}_j(\boldsymbol{\lambda})
	\right \|_2
	\le
	\lambda_{\min}^{-2} \|y\|_2 \sqrt{\|K_j\|_2 \sum_{{j'}=1}^J \|K_{j'}\|_2^2}
	\end{align}
	The eigenvalues of $K_j$ are bounded above by the largest row sum, which is no more than $2 n_T$ (assuming all training covariates are between 0 and 1).
	Putting the results above together, we have
	\begin{align}
	\left \|
	\nabla_{\lambda} \hat{g}_{j, \perp}(\boldsymbol{\lambda})(x_j)
	\right \|_2
	\le
	\frac{c \sqrt{J} n_T}{h_j(D^{(n_T)})^2 \lambda_{\min}^2}
	\|y\|_2.
	\end{align}
	
	Also, we have from \eqref{eq:kkt_sobolev_linear} that
	\begin{align}
	\left \| \nabla_{\lambda} \hat{\boldsymbol{\alpha}_1}(\boldsymbol{\lambda}) \right \|_2
	& =
	\left \|
	\left(
	X^\top X
	\right)^{-1}
	X^\top
	\nabla_{\lambda_j} K \hat{\boldsymbol{\theta}}(\boldsymbol{\lambda})
	\right \|_2 \\
	& =
	\left \|
	\left(
	X^\top X
	\right)^{-1}
	X^\top
	\nabla_{\lambda_j} K_j \hat{\boldsymbol{\theta}}_j(\boldsymbol{\lambda})
	\right \|_2 \\
	& \le
	\left \|
	\left(
	X^\top X
	\right)^{-1}
	X^\top
	\right \|_2
	\lambda_{\min}^{-2} \|y\|_2 n_T \sqrt{J}
	\end{align}
	Finally we can conclude that
	\begin{align}
	\begin{split}
	\left \|
	\hat{g}_j(\boldsymbol{\lambda}^{(1)})(x_j)
	- \hat{g}_j(\boldsymbol{\lambda}^{(2)})(x_j)
	\right \|_2
	& \le
	\left(
	|x_j|
	\left \|
	\left(
	X^\top X
	\right)^{-1}
	X^\top
	\right \|_2
	+
	\frac{c }{h_j(D^{(n_T)})^2}
	\right)\\
	& \quad \times
	\sqrt{J}
	n_T
	\lambda_{\min}^{-2}
	\|y\|_2
	\| \boldsymbol{\lambda}^{(1)} - \boldsymbol{\lambda}^{(2)}\|_2
	\end{split}
	\end{align}
	By triangle inequality, we obtain our desired result in \eqref{eq:sobolev_lipschitz}.
\end{example}

\begin{example}[Multiple elastic nets, training-validation split]
	Here we check that Condition~\ref{condn:nonsmooth1} is satisfied.
	Since the absolute value function $|\cdot|$ is twice-continuously differentiable everywhere except at zero, the directional derivatives of $||\boldsymbol \theta^{(j)}||_1$ at $\hat{\boldsymbol{\theta}}(\boldsymbol{\lambda})$ only exist along directions spanned by the columns of $\boldsymbol I_{I^{(j)}(\boldsymbol \lambda)}$.
	Thus the penalized training loss $L_T(\cdot, \boldsymbol{\lambda})$ is twice differentiable with respect to the directions in \eqref{eq:en_diff_space}.
	Moreover, the elastic net solution paths are piecewise linear \citep{zou2003regression}, which means that the nonzero indices of the elastic net estimates stay locally constant for almost every $\boldsymbol{\lambda}$. Therefore, \eqref{eq:en_diff_space} is also a local optimality space for $L_T(\cdot, \boldsymbol{\lambda})$.
	
	We also show that the Hessian of the penalized training loss has a minimum eigenvalue of $\lambda_{\min}$.
	Consider the following orthogonal basis of \eqref{eq:en_diff_space} at $\hat{\boldsymbol{\theta}}(\boldsymbol{\lambda})$: $U(\boldsymbol{\lambda}) = \{U^{(j)}(\boldsymbol{\lambda})\}_{j = 1}^J$ where
	\begin{align}
	U^{(j)} =
	\left(
	\begin{matrix}
	\boldsymbol{0} \\
	I_{I^{(j)}(\boldsymbol \lambda)}\\
	\boldsymbol{0}
	\end{matrix}
	\right)
	\quad \forall j = 1,...,J.
	\end{align}
	The Hessian matrix of $L_T(\cdot, \boldsymbol{\lambda})$ with respect to directions $U(\boldsymbol{\lambda})$ is
	\begin{align}
	\boldsymbol U(\boldsymbol{\lambda})^\top \boldsymbol{X}_{T}^\top \boldsymbol{X}_{T} \boldsymbol U(\boldsymbol{\lambda}) + \lambda_1 w \boldsymbol{I}
	\label{eq:en_hessian}
	\end{align}
	where $\boldsymbol{X}_{T} = (\boldsymbol{X}^{(1)} ... \boldsymbol{X}^{(J)})$
	and $\boldsymbol{I}$ is the identity matrix with length equal to the number of nonzero elements in $\hat{\boldsymbol{\theta}}(\boldsymbol{\lambda})$.
	Since the first summand is positive semi-definite and $\lambda_1 > \lambda_{\min}$, \eqref{eq:en_hessian} has a minimum eigenvalue of $\lambda_{\min}$.
\end{example}


\bibliographystyle{unsrtnat}
\bibliography{hyperparam-theory-appendix}

\end{document}
