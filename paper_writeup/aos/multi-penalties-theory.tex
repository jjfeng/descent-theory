\documentclass[aos,preprint]{imsart}

\RequirePackage[OT1]{fontenc}
\RequirePackage{amsthm,amsmath}
\RequirePackage[numbers]{natbib}
\RequirePackage[colorlinks,citecolor=blue,urlcolor=blue]{hyperref}

\usepackage{amsmath}
\usepackage{graphicx,psfrag,epsf}
\usepackage{enumerate}
\usepackage{url} % not crucial - just used below for the URL
\usepackage{color}
\usepackage{mathtools}
\usepackage{xr}
\usepackage{amssymb}

\usepackage{xr}
\externaldocument{supp-multi-penalties-theory}

\usepackage[utf8]{inputenc}
\usepackage{cite}


\newtheorem{theorem}{Theorem}
\newtheorem{corollary}{Corollary}
\newtheorem{lemma}{Lemma}
\newtheorem{definition}{Definition}
\newtheorem{condition}{Condition}

\DeclareMathOperator*{\argmin}{arg\,min}


\begin{document}

\begin{frontmatter}
\title{Oracle inequalities for hyper-parameter selection via split-sample validation with applications to penalized regression}
\runtitle{Oracle inequalities for hyper-parameter selection}

\begin{aug}
\author{\fnms{Jean} \snm{Feng}\thanksref{t1,m1}\ead[label=e1]{jeanfeng@uw.edu}},
\author{\fnms{Noah} \snm{Simon}\thanksref{t2,m1}\ead[label=e2]{nrsimon@uw.edu}}

\thankstext{t1}{J.F. was supported by NIH Grant T32CA206089.}
\thankstext{t2}{N.S. was supported by NIH Grant DP5OD019820.}
\runauthor{Feng and Simon}

\affiliation{Department of Biostatistics, University of Washington\thanksmark{m1}}

\address{Health Sciences Building \\
	F-650, Box 357232 \\
	University of Washington \\
	Seattle, WA 98195\\
\printead{e1}\\
\phantom{E-mail:\ }\printead*{e2}}

\end{aug}

\begin{abstract}
In the regression setting, given a set of hyper-parameters, a model-estimation procedure constructs a model from training data. The optimal hyper-parameters that minimize generalization error of the model are usually unknown. In practice they are often estimated using split-sample validation. Up to now, there is an open question regarding how the generalization error of the selected model grows with the number of hyper-parameters to be estimated. To answer this question, we establish finite-sample oracle inequalities for selection based on a single training/test split and based on cross-validation. We show that if the model-estimation procedures are smoothly parameterized by the hyper-parameters, the error incurred from tuning hyper-parameters shrinks at nearly a parametric rate. Hence for semi- and non-parametric model-estimation procedures with a fixed number of hyper-parameters, this additional error is negligible. For parametric model-estimation procedures, adding a hyper-parameter is roughly equivalent to adding a parameter to the model itself. In addition, we specialize these ideas for penalized regression problems with multiple penalty parameters. We establish that the fitted models are Lipschitz in the penalty parameters and thus our oracle inequalities apply. This result encourages development of regularization methods with many penalty parameters.
\end{abstract}


%\begin{keyword}[class=MSC]
%\kwd[Primary ]{}
%\kwd{}
%\kwd[; secondary ]{}
%\end{keyword}

\begin{keyword}
\kwd{Cross-validation}
\kwd{Hyper-parameter selection}
\kwd{Oracle inequalities}
\kwd{Regression}
\kwd{Regularization}
\end{keyword}

\end{frontmatter}

\section{Introduction}

Per the usual regression framework, suppose we observe response $y \in \mathbb{R}$ and predictors $\boldsymbol {x} \in \mathbb{R}^p$. Suppose $y$ is generated by a true model $g^*$ plus random error $\epsilon$ with $\operatorname{E}\left[\epsilon\right] = 0$, as follows
\begin{equation}
\label{true_model}
y = g^*(\boldsymbol x) + \epsilon
\end{equation}
Our goal is to estimate $g^*$.

Many model-estimation procedures can be formulated as selecting a model from some function class $\mathcal{G}$ given training data $T$ and $J$-dimensional hyper-parameter vector $\boldsymbol{\lambda}$. For example, in penalized regression problems, the fitted model can be expressed as the minimizer of the penalized training criterion
\begin{equation}
\label{eq:intro_pen_reg}
\hat{g}(\boldsymbol \lambda | T) = \argmin_{g\in \mathcal{G}} \sum_{(x_i, y_i) \in T} \left (y_i -  g(x_i) \right )^2 + \sum_{j=1}^J \lambda_j P_j(g)
\end{equation}
where $P_j$ are penalty functions and $\lambda_j$ are penalty parameters. As suggested by the notation in \eqref{eq:intro_pen_reg}, the penalty parameters are the hyper-parameters in this model-estimation procedure.

Given a set of possible hyper-parameters $\Lambda$, for a given training dataset $T$ and norm $\|\cdot\|$, there is some oracle hyper-parameter $\tilde{\boldsymbol{\lambda}} \in \Lambda$ that minimizes the difference between the fitted model $\hat{g}\left(\boldsymbol{\lambda|}T\right)$ and the true model:
\[
\tilde{\boldsymbol{\lambda}} = \argmin_{\lambda \in \Lambda} \left\|g^{*} - \hat{g}\left(\boldsymbol{\lambda|}T\right)\right\|^2
\]
$\tilde{\boldsymbol{\lambda}}$ is unknown and often estimated using a single training/validation split or cross-validation. The basic idea is to fit models on a random partition of the observed data and evaluate their error on the remaining data. The final hyper-parameters $\hat{\boldsymbol{\lambda}}$ are the minimizer of the error on this validation set. For a more complete review of cross-validation, refer to \citet{arlot2010survey}.

The performance of split-sample validation procedures is typically characterized by an oracle inequality that bounds the generalization error of the expected model selected from the validation set procedure. For $\Lambda$ that are finite, oracle inequalities have been established for a single training/validation split \citet{gyorfi2006distribution} and a general cross-validation framework \citep{van2003unified, van2004asymptotic}. To handle continuous $\Lambda$, one can use entropy-based approaches \citep{lecue2012oracle}. 

The goal of this paper is to characterize the performance of models when the hyper-parameters must be tuned by some split-sample validation procedure. We are particularly interested in an open question raised in \citet{bengio2000gradient}: what is the ``amount of overfitting... when too many hyper-parameters are optimized''? To do this, we establish finite-sample oracle inequalities of the form
\begin{equation}
\label{thrm:intro_oracle_ineq}
\left \| g^* - \hat{g}\left (\hat{\boldsymbol{\lambda}}, T \right ) \right \|^2
\le
(1+a)
\underbrace{\inf_{\lambda \in \Lambda} \left \| g^* - \hat{g}\left (\boldsymbol{\lambda} , T \right ) \right \|^2}_{\text{Oracle error}}
+ \delta\left(J,n\right)
\end{equation}
for some norm $\| \cdot \|$ and constant $a \ge 0$; where $\delta(J,n)$ is a function of the number of parameters to tune and the number of validation samples $n$. Under the assumption that the model-estimation procedure is smoothly parameterized by the hyper-parameters, we find that the contribution to $\delta$ from tuning $J$ hyper-parameters scales linearly in $J$. For parametric model-estimation procedures, the additional error from adding a hyper-parameter is roughly equivalent to adding a parameter to the model itself. For semi- and non-parametric model-estimation procedures, this error is generally dominated by the oracle error and the number of hyper-parameters can actually grow without affecting the asymptotic convergence rate.

In this paper, we also specialize these results to penalized regression models of the form \eqref{eq:intro_pen_reg}. We show that the fitted model is indeed smoothly parameterized by the penalty parameters so our oracle inequalities apply. Again, we find that additional penalty parameters only add a near-parametric error term, which has a negligible effect in semi- and non-parametric settings. This result suggests that the recent interest in combining penalty functions (e.g. elastic net and sparse group lasso \citep{zou2003regression, simon2013sparse}) may have artificially restricted themselves to two-way combinations. Adding more penalties may lead to better models.

During our literature search, we found few theoretical results addressing the relationship between the number of hyper-parameters and generalization error of the selected model. Most oracle inequalities only consider tuning a one-dimensional hyper-parameter over a finite $\Lambda$ \citep{van2003unified, van2004asymptotic, gyorfi2006distribution}. Others have addressed split-sample validation for specific penalized regression problems but only those with a single penalty parameter \citep{golub1979generalized, chetverikov2016cross, chatterjee2015prediction}. Only \citet{lecue2012oracle} has a result that is relevant to answering our question of interest by using techniques from empirical process theory. A potential reason for this dearth of literature is that, historically, tuning multiple hyper-parameters has been computationally difficult. However, there have been many proposals recently for overcoming this computational hurdle \citep{bengio2000gradient, foo2008efficient, snoek2012practical}.

Section \ref{sec:main_results} presents oracle inequalities for model-estimation procedures that are smoothly parameterized by the hyper-parameters. These results answer our question regarding how the number of hyper-parameters affects the model error.
Section \ref{sec:examples} applies these results to penalized regression models.
Section \ref{sec:simulations} provides a simulation study to support our theoretical results.
Section \ref{sec:discussion} discusses our findings and potential future work.
Oracle inequalities for general model-estimation procedures and proofs for all the results are given in the Appendix.


\section{Main Result} \label{sec:main_results}

In this section, we establish oracle inequalities for models where the hyper-parameters are tuned by a single training/validation split and cross-validation. We first introduce some notation and formalize the model-estimation procedure. 

Let $D^{(n)}$ denote a dataset with $n$ samples from the model \eqref{true_model}. The model-estimation procedure accepts some hyper-parameter of dimension $J$ and training data of size $n_T$ to output a fitted model from some model class $\mathcal{G}$. This can be formulated as an operator $\hat{g}^{(n_T)}(\cdot | D^{(n_T)})$ that maps a hyper-parameter vector $\boldsymbol{\lambda}$ from some set $\Lambda \subseteq \mathbb{R}^J$ to a function in $\mathcal{G}$. 

In this section, we focus on model-estimation procedures that are Lipschitz.
\begin{definition}
	\label{def:smooth_funcs}
	Let $\mathcal{F}$ be a function class. Let $\Lambda \subseteq \mathbb{R}^J$.
	The operator $\hat{f}: \Lambda \mapsto \mathcal{F}$ is $C$-Lipschitz in $\boldsymbol{\lambda}$ with respect to norm $\| \cdot \|$ over $\Lambda$ if
	\begin{equation}
	\left \| \hat{f}(\boldsymbol \lambda) - \hat{f}(\boldsymbol \lambda ') \right \|
	\le
	C \| \boldsymbol \lambda - \boldsymbol \lambda' \|_2 
	\quad
	\forall \boldsymbol \lambda,\boldsymbol \lambda' \in \Lambda
	\label{eq:smooth_funcs}
	\end{equation}
\end{definition}
We hypothesize that many model-estimation procedures satisfy this Lipschitz assumption since it ensures that the procedure is well-behaved. Section \ref{sec:examples} shows that penalized regression models indeed satisfy this assumption. The following results in Sections~\ref{sec:single} and \ref{sec:cv} show that the contribution to the error from tuning multiple hyper-parameters for such procedures is roughly parametric. Hence for semi- or non-parametric model-estimation procedures, the error from tuning a fixed number of hyper-parameters is negligible. In fact, we specify a bound on the rate at which the number of hyper-parameters can grow without asymptotically increasing the generalization error of the selected model.

\subsection{A single Training/Validation Split}\label{sec:single}

For a single training/validation split, the dataset $D^{(n)}$ is randomly partitioned into a training set $T = (X_T, Y_T)$ and validation set $V = (X_V, Y_V)$ with $n_T$ and $n_V$ observations, respectively. The selected hyper-parameter $\hat{\boldsymbol{\lambda}}$ is the minimizer of the validation loss
\begin{equation}
\label{eq:train_val_lambda}
\hat{\boldsymbol \lambda} = \argmin_{\boldsymbol{\lambda} \in\Lambda} \frac{1}{2} \left \| y-\hat{g}^{(n_T)}( \boldsymbol \lambda | T) \right \|_{V}^{2}
\end{equation}
where $\| h \|_{V}=\frac{1}{n_V}\sum_{i\in V} h^2(x_i)$ for any function $h$. 

We now present a finite-sample oracle inequality for the single training/validation split assuming the model-estimation procedure is Lipschitz. Our oracle inequality is sharp, i.e. $a=0$ in \eqref{thrm:intro_oracle_ineq}, unlike most other work \citep{gyorfi2006distribution, lecue2012oracle, van2003unified}. Note that the result below is a special case of Theorem \ref{thrm:train_val_complicated} in Appendix \ref{appendix:train_val}, which applies to general model-estimation procedures.
\begin{theorem}
	\label{thrm:train_val}
	Let $\Lambda=[\lambda_{\min},\lambda_{\max}]^{J}$ where $\Delta_{\lambda} = \lambda_{\max} - \lambda_{\min} \ge 0$. Suppose independent random variables $\epsilon_1, ... \epsilon_n$ have expectation zero and are uniformly sub-Gaussian with parameters $b$ and $B$:
	$$
	\max_{i=1,...,n} B^2 \left ( \mathbb{E} e^{|\epsilon_i|^2/B^2} - 1 \right ) \le b^2
	$$
	Suppose there is a constant $C_\Lambda \ge 32e/(n \Delta_{\Lambda})$ such that $\hat g^{(n_T)}(\boldsymbol{\lambda} |D^{(n_T)})$ is $C_\Lambda$-Lipschitz with respect to $\| \cdot \|_V$ over $\Lambda$.
	
	Let 
	\begin{equation}
	\tilde{\boldsymbol \lambda} = \argmin_{\lambda \in \Lambda} \left \| g^*-\hat{g}^{(n_T)}( \boldsymbol{\lambda} | T) \right \|_{V}^{2}
	\label{eq:tilde_lambda_def}
	\end{equation}
	
	Then there is a constant $c>0$ only depending on $b$ and $B$ such that for all $\delta$ satisfying
	\begin{equation}
	\delta^{2}
	\ge
	c \left ( 
	\frac{J\log (n C_\Lambda \Delta_{\Lambda})}{n_{V}}
	\vee 
	\sqrt{\frac{J \log (n C_\Lambda \Delta_{\Lambda})}{n_{V}}\left\Vert g^* - \hat{g}^{(n_T)}( \tilde{\boldsymbol{\lambda}} | T)\right\Vert_{V}^2}
	\right )
	\label{thrm:train_val_delta}
	\end{equation}
	we have
	\begin{align}
	& Pr\left(
	\left\Vert g^* - \hat{g}^{(n_T)}( \hat{\boldsymbol{\lambda}} | T) \right\Vert _{V}^2 -
	\left\Vert g^* - \hat{g}^{(n_T)}( \tilde{\boldsymbol{\lambda}} | T) \right\Vert _{V}^2
	\ge\delta^2
	\middle | 
	T, X_V
	\right )\\
	&\le c\exp\left(-\frac{n_{V}\delta^{4}}{
		c^{2}
		\left\Vert g^* - \hat{g}^{(n_T)}( \tilde{\boldsymbol{\lambda}} | T) \right\Vert _{V}^2
	}\right) \\
	& +c\exp\left(-\frac{n_{V}\delta^{2}}{c^{2}}\right)
	\end{align}
	
\end{theorem}

Theorem \ref{thrm:train_val} states that with high probability, the difference between the true model and the selected model is bounded above by the difference between the true model and the oracle model plus $\delta^2$. Hence $\delta^2$ can be thought of as the error incurred during the hyper-parameter selection process. As seen in \eqref{thrm:train_val_delta}, it is the maximum of two terms: a near-parametric term and a geometric mean of the near-parametric term and the oracle error. To see this more clearly, we express Theorem \ref{thrm:train_val} using asymptotic notation.
\begin{corollary}
	\label{corr:train_val}
	Under the assumptions given in Theorem \ref{thrm:train_val}, we have
	\begin{align}
	\left\Vert g^* - \hat{g}^{(n_T)}( \hat{\boldsymbol{\lambda}} | T) \right\Vert _{V}^2 &\le \left\Vert g^* - \hat{g}^{(n_T)}( \tilde{\boldsymbol{\lambda}} | T) \right \Vert^2_{V}\\
	& + O_p \left(\frac{J\log (n C_\Lambda \Delta_{\Lambda} )}{n_{V}} \right) 
	\label{eq:asym_train_val_theorem1} \\
	& + O_p \left(
	\sqrt{
		\frac{J \log (n C_\Lambda \Delta_{\Lambda} )}{n_{V}}
		\left\Vert g^* - \hat{g}^{(n_T)}( \tilde{\boldsymbol{\lambda}}| T) \right \Vert^2_{V}
	}
	\right )
	\label{eq:asym_train_val_theorem2}
	\end{align}
\end{corollary}
Hence the error of the selected model is bounded by the error of the oracle model, the near-parameteric term \eqref{eq:asym_train_val_theorem1}, and the geometric mean of the two values \eqref{eq:asym_train_val_theorem2}. We refer to \eqref{eq:asym_train_val_theorem1} as near-parametric because the error term in parametric regression models are usually $O_p(J/n)$, where $J$ is the parameter dimension and $n$ is the number of training samples. Analogously, \eqref{eq:asym_train_val_theorem1} is roughly $O_p(J/n_V)$ modulo a $\log n$ term in the numerator.

In the semi- and non-parametric regression settings, the oracle error usually shrinks at a rate of $n^{-\omega}$ where $\omega \in (0, 1)$, which means that for large $n$, the oracle error will tend to dominate both the error terms. Therefore increasing the number of hyper-parameters for such problems only results in a small increase in the model error. In fact, if the oracle error rate is $O_p(n_T^{-\omega})$, the number of hyper-parameters $J$ can grow at the rate
\begin{equation}
\frac{n_{V} n_T^{-\omega}}{\log (n C_\Lambda\Delta_{\Lambda})}
\end{equation}
without affecting the asymptotic convergence rate.

The appearance of the parametric term \eqref{eq:asym_train_val_theorem1} suggests that we can interpret the problem of tuning hyper-parameters as a parametric regression problem over a $J$-dimensional parameter space where the validation data is the training data. However, this interpretation is an oversimplification. Recall that we perform the training/validation split over the model class
\begin{equation}
\mathcal{G}(T) = \left \{ \hat{g}^{(n_T)}( {\boldsymbol{\lambda}}| T) : \boldsymbol{\lambda} \in \Lambda \right \}
\end{equation}
$\mathcal{G}(T)$ is unlikely to contain the true model $g^*$ and its minimum bias is the oracle error. This bias contributes to the convergence rate via the geometric mean \eqref{eq:asym_train_val_theorem2}.

\subsection{Cross-Validation}\label{sec:cv}

In this section, we give an oracle inequality for $K$-fold cross-validation. Previously, the oracle inequality was with respect to the $L_2$ norm over the validation covariates. Now we give our result with respect to functional $L_2$ norm
\begin{equation}
\left \| g - g^* \right \|^2 = \int \left |g(x) - g^*(x) \right |^2 dx
\end{equation}

For $K$-fold cross-validation our setup is as follows. Let dataset $D^{(n)}$ be randomly partitioned into $K$ sets, which we assume to have equal size for simplicity. Partition $k$ will be denoted $D_k^{(n_V)}$ and its complement will be denoted $D_{-k}^{(n_T)} = D^{(n)} \setminus D_k^{(n_V)}$. We perform our model-estimation procedure given $D_{-k}^{(n_T)}$ for $k=1,...,K$ and select the hyper-parameter that minimizes the average validation loss
\begin{eqnarray}
\label{kfold_opt}
\hat{\boldsymbol \lambda} &=& \argmin_{\boldsymbol{\lambda} \in\Lambda} \frac{1}{2K} \sum_{k=1}^K  \left \| y-\hat{g}^{(n_T)}(\boldsymbol \lambda | D_{-k}^{(n_T)}) \right \|_{D_k^{(n_V)}}^{2}
\end{eqnarray}

In traditional cross-validation, the final model is retrained on all the data with $\hat{\boldsymbol{\lambda}}$. However, bounding the generalization error of the retrained model requires additional regularity assumptions \citep{lecue2012oracle}. We consider the ``averaged version of $K$-fold cross-validation'' instead
\begin{equation}
\label{thrm:avg_cv}
\bar{g}\left ( \hat{\boldsymbol \lambda} \middle | {D^{(n)}} \right ) = 
\frac{1}{K} \sum_{k=1}^K 
\hat{g}^{(n_T)} \left (\hat{\boldsymbol \lambda} \middle | D^{(n_T)}_{-k} \right )
\end{equation}

The following theorem bounds the generalization error of \eqref{thrm:avg_cv}. It is an application of Theorem 3.5 in \citet{lecue2012oracle}, which is reproduced in Theorem \ref{thrm:mitchell} in the Appendix \ref{app:cv} for convenience.

\begin{theorem}
	\label{thrm:kfold}
	Let $\Lambda=[\lambda_{\min},\lambda_{\max}]^{J}$ where $\Delta_{\Lambda} = \lambda_{\max} - \lambda_{\min} \ge 0$. Suppose there is a $G \ge  2$ such that $\sup_{g \in \mathcal{G}} \|g\|_\infty \le G$.
	
	Consider the setting of averaged version of $K$-fold cross-validation where $K \ge 2$ and the datasets are of size $n$ where $n$ is divisible by $K$. Let $n_V = n/K = n_V$ and $n_T = n - n_V$. Suppose random variables $\epsilon_i$ are independent with expectation zero and are bounded $\| \epsilon_i \|_\infty \le \sigma$. Suppose there is a constant $C_\Lambda > e^5/G\Delta_{\Lambda}$ such that for any dataset $D^{(n_T)}$, $\hat g (\boldsymbol{\lambda} | D^{(n_T)})$ is $C_\Lambda$-Lipschitz with respect to $\| \cdot \|_\infty$ over $\Lambda$.
	
	Then there is an absolute constant $c > 0$ such that for all $a > 0$,
	\begin{align*}
	E_{D^{(n)}} \left \| \bar{g} ( \hat{\boldsymbol \lambda} | {D^{(n)}} ) - g^* \right \|^2 &\le
	(1+a) \min_{\boldsymbol{\lambda} \in \Lambda}  E_{D^{(n_T)}} \left \| \hat{g}^{(n_T)}(\boldsymbol \lambda | D^{(n_T)}) - g^* \right \|^2 \\
	& + c \frac{(1+a)^2}{a} \frac{J G \log (GC_\Lambda \Delta_{\Lambda} ) \log n_V }{n_V} 
	\end{align*}
\end{theorem}

As we can see, Theorems \ref{thrm:train_val} and \ref{thrm:kfold} are quite similar. The upper bounds in both theorems depend on the oracle error and a near-parametric term. For parametric model-estimation procedures, tuning hyper-parameters incurs a similar cost as the model-estimation procedure itself. In semi- and non-parametric regression settings, tuning hyper-parameters is a relatively ``cheap'' and incurs an error that is negligible asymptotically.

There are also some notable differences between Theorems \ref{thrm:train_val} and \ref{thrm:kfold}. The Lipschitz condition in Theorem \ref{thrm:kfold} is required to hold with respect to $\| \cdot \|_\infty$, which is stricter than that in Theorem \ref{thrm:train_val}. Also, we no longer have a sharp oracle inequality since the oracle error is scaled by $1+a$ where $a > 0$. These differences occur because we are interested in characterizing the functional $L_2$ error instead of the $L_2$ error over the observed covariates.

\section{Penalized regression models}
\label{sec:examples}

Penalized regression is a class of model-generating procedures where hyper-parameters are of particular interest. In this manuscript, we consider penalized regression procedures of the form \eqref{eq:intro_pen_reg}. Penalty functions are used to control model complexity and induce desired characteristics (e.g. smoothness or sparsity). When multiple penalty functions are used, the resulting model exhibits a combination of the desired characteristics. Hence there has been recent interest in combining penalty functions, such as the elastic net or sparse group lasso. However few popular methods use more than two penalties. This has been due to a) the concern that models may overfit the data when selection of many penalty parameters is required; and b) computational issues in optimizing multiple penalty parameters. In this section, we evaluate the validity of concern (a) using the results of Section~\ref{sec:main_results}. We see that, contrary to popular wisdom, using split-sample validation to select multiple penalty parameters should not result in overfitting (even if many parameters need to be tuned). For computational concerns (b), we refer the reader to recent papers on hyper-parameter estimation \citep{bengio2000gradient, foo2008efficient, snoek2012practical}.

Recall that in our framework, the model-estimation procedure takes in penalty parameters (i.e. hyper-parameters) and then finds the minimizer of the penalized training criterion \eqref{eq:intro_pen_reg}. Models are fit for penalty parameters within some set $\Lambda$. As long as we can show that the fitted models are Lipschitz in the penalty parameters over $\Lambda$, we can apply Theorems \ref{thrm:train_val} and \ref{thrm:kfold}.

Our results do not allow us to consider split-sample validation over all $\lambda\in\mathbb{R}^J_+$: generally as $\lambda_{min} \rightarrow 0$, for any finite $n$, our fitted models become very widely behaved. Instead we restrict ourselves to 
\begin{equation}
\label{thrm:lambda_range}
\Lambda = [ n^{-t_{\min}}, n^{t_{\max}}]^J
\end{equation}
for sufficiently large $t_{\min}, t_{\max} \ge 0$. This regime works well for two reasons: one, our rates depend only quite weakly on $t_{\min}$ and $t_{\max}$; and two, generally oracle $\lambda$-values are $\sim n^{-\alpha}$ for some $\alpha \in (0,1)$ \citep{van2000empirical, van2014additive, buhlmann2011statistics}. So long as $t_{\min} > \alpha$, we ensure that $\Lambda$ contains the optimal penalty parameters over all of $\mathbb{R}^J_+$.

If $\Lambda$ is of the form \eqref{thrm:lambda_range}, we find that the Lipschitz constant for many penalized regression examples are polynomial in $n$, such as in Sections \ref{sec:param_add_models} and \ref{sec:nonparam_smooth}. That is, there exist constants $C, \kappa \ge 0$ such that
\begin{equation}
\label{thrm:lipschitz_pen_reg}
\left \| \hat{g}^{(n_T)}(\boldsymbol{\lambda}|T) - \hat{g}^{(n_T)}(\boldsymbol{\lambda}'|T) \right \| \le C n^\kappa \left \| \boldsymbol{\lambda} - \boldsymbol{\lambda}' \right \| \quad \forall \boldsymbol{\lambda}, \boldsymbol{\lambda}' \in \Lambda
\end{equation}
In our examples, the Lipschitz constant is inversely proportional to $\lambda_{\min}$, so $\kappa$ grows linearly in $t_{\min}$. Assuming that \eqref{thrm:lambda_range} and \eqref{thrm:lipschitz_pen_reg} hold, we get the following oracle inequalities when penalty parameters are tuned via a single training/validation split and cross-validation. 
\begin{corollary}
	Suppose $\lambda_{\min} = n^{-t_{\min}}$ and $\lambda_{\max} = n^{t_{\max}}$ for $t_{\min}, t_{\max} \ge 0$. Let $\kappa  > 0$.
	
	\begin{enumerate}
		\item Single training/validation split: Suppose conditions in Theorem \ref{thrm:train_val} hold with $C_\Lambda = O_p (n^\kappa)$. Then
		\begin{align*}
		\left\Vert g^* - \hat{g}^{(n_T)}( \hat{\boldsymbol{\lambda}} | T) \right\Vert _{V}^2
		&\le \left\Vert g^* - \hat{g}^{(n_T)}( \tilde{\boldsymbol{\lambda}} | T) \right \Vert^2_{V}\\
		& + O_p \left(\frac{J (1 + \kappa + t_{\max})\log n}{n_{V}} \right) 
		\\
		& + O_p \left(
		\sqrt{
			\frac{J (1 + \kappa + t_{\max})\log n}{n_{V}}
			\left\Vert g^* - \hat{g}^{(n_T)}( \tilde{\boldsymbol{\lambda}}| T) \right \Vert^2_{V}
		}
		\right )
		\end{align*}
		
		\item Averaged version of $K$-fold cross-validation: Suppose conditions in Theorem \ref{thrm:kfold}  hold with $C_\Lambda = O_p(n^\kappa)$. For sufficiently large $n$, there is an absolute constant $c > 0$ such that for all $a > 0$,
		\begin{eqnarray*}
			E_{D^{(n)}} \left \| \bar{g} ( \hat{\boldsymbol \lambda} | {D^{(n)}} ) - g^* \right \|^2 &\le&
			(1+a) \min_{\boldsymbol{\lambda} \in \Lambda}  E_{D^{(n_T)}} \left \| \hat{g}^{(n_T)}(\boldsymbol \lambda | D^{(n_T)}) - g^* \right \|^2 \\
			&& + c \frac{(1+a)^2}{a} 
			\frac{JG (t_{\max} + \kappa) (\log n)^2 }{n_V} 
		\end{eqnarray*}
	\end{enumerate}
\end{corollary}
These results are very similar to those in Corollary \ref{corr:train_val} and Theorem \ref{thrm:kfold}. We still have the same near-parametric term and geometric mean in the oracle inequality for the single training/validation split and a near-parametric term in the oracle inequality for cross-validation. The near-parametric terms have the familiar $\log n$ in the numerator because the error terms are proportional to the log of the Lipschitz constant and $\lambda_{\max}$. 

Therefore we reach the same conclusion for the relationship between penalty parameters and the generalization error of the selected model. In high-dimensional and non-parametric penalized regression problems satisfying \eqref{thrm:lipschitz_pen_reg}, the error from tuning penalty parameters is negligible compared to the error from solving the penalized regression problem with the oracle penalty parameters.

It remains to characterize penalized problems wherein the fitted models are Lipschitz in the penalty parameters. In this manuscript we will do an in-depth study of additive models, which have the form
\begin{equation}
g(x_1, ..., x_J)= \sum_{j=1}^J g_j(x_j)
\end{equation}
though we believe that these results hold much more generally. We first consider parametric additive models (with potentially growing numbers of parameters) fitted with smooth and non-smooth penalties and then nonparametric additive models. Results for more general penalized regression may be derived in a similar manner (note that if the fitted models are not Lipschitz, one would need the general oracle inequalities from Theorems \ref{thrm:train_val_complicated} and \ref{thrm:mitchell} instead).

\subsection{Parametric additive models}
\label{sec:param_add_models}
Parametric additive models have the form
\begin{equation}
g(\boldsymbol{\theta}^{(1)}, ..., \boldsymbol{\theta}^{(J)}) = \sum_{j=1}^J g_j(\boldsymbol{\theta}^{(j)})
\end{equation}
where $\boldsymbol{\theta}^{(j)} \in \mathbb{R}^{p_j}$ and $p = \sum_{j=1}^J p_j$. The number of dimensions $p_j$ is allowed to grow with $n$, as commonly done in sieve estimation. For simplicity, let the full parameter vector be denoted $\boldsymbol{\theta} = \left (\boldsymbol{\theta}^{(1)}, ..., \boldsymbol{\theta}^{(J)} \right )^\top$. Then we can write the training criterion for training data $T$ as
\begin{equation}
\label{eq:param_add}
L_T \left (\boldsymbol{\theta}, \boldsymbol{\lambda} \right) 
\coloneqq \frac{1}{2} \left  \| y -  g(\boldsymbol{\theta}) \right \|^2_T 
+ \sum_{j=1}^J \lambda_j P_j(\boldsymbol{\theta}^{(j)})
\end{equation}
Suppose $\boldsymbol{\theta}^*$ is the minimizer of the expected generalization error
\begin{equation}
\boldsymbol{\theta}^* = \argmin_{\theta \in \mathbb{R}^p} E_T \left [ \left \| y - g(\boldsymbol{\theta}) \right\|^2 \right ]
\end{equation}

\subsubsection{Parametric regression with smooth penalties}
\label{sec:param_smooth}
We begin with the simple case where the penalty functions are smooth. The following lemma states that the fitted models are Lipschitz in the penalty parameter vector.
\begin{lemma}
	\label{lemma:param_add}
	Let 
	\begin{equation}
	\label{eq:param_add_estimator}
	\hat{\boldsymbol{\theta}}^{(j)}\left (\boldsymbol{\lambda} | T \right )  = 
	\argmin_{\boldsymbol{\theta} \in \mathbb{R}^p} L_T \left (\boldsymbol{\theta}, \boldsymbol{\lambda} \right)
	\end{equation}
	
	\noindent
	Suppose that $g_j(\boldsymbol{\theta}^{(j)})$ are $L$-Lipschitz in $\boldsymbol{\theta}^{(j)}$ with respect to $\| \cdot \|_\infty$ for all $j=1,..,J$.
	
	\noindent
	Further suppose $P_j(\boldsymbol{\theta}^{(j)})$ and $g_j(x | \boldsymbol{\theta}^{(j)})$ are twice-differentiable and convex with respect to $\boldsymbol{\theta}^{(j)}$ for any fixed $x$ and all $j=1,..,J$. Also suppose $L_T \left (\boldsymbol{\theta}, \boldsymbol{\lambda} \right)$ is twice-differentiable and convex with respect to $\boldsymbol{\theta}$.
	
	Suppose there exists $m > 0$ such that the Hessian of the penalized training criterion at the minimizer satisfies 
	\begin{equation}
	\left . \nabla_{\theta}^2 L_T \left (\boldsymbol{\theta}, \boldsymbol{\lambda} \right) \right |_{\theta = \hat{\theta}(\boldsymbol{\lambda} | T )} \succeq mI
	\end{equation}
	
	Let $\lambda_{\max} > \lambda_{\min} > 0 $. Let
	\begin{equation}
	C_{\theta^{*},\Lambda}=
	\frac{1}{2}\left\Vert y- g(\boldsymbol{\theta}^{*})\right\Vert _{T}^{2}
	+\lambda_{max}\sum_{j=1}^{J} P_{j}(\boldsymbol{\theta}^{(j),*})
	\end{equation}
	
	Then, for any $\boldsymbol{\lambda}^{(1)}, \boldsymbol{\lambda}^{(2)} \in \Lambda \coloneqq \left [ \lambda_{\min}, \lambda_{\max} \right ]^J$, we have
	\begin{equation}
	\label{eq:param_add_lipschitz}
	\left\Vert g\left(\hat{\boldsymbol{\theta}}(\boldsymbol{\lambda^{(1)}} | T)\right)-
	g\left(\hat{\boldsymbol{\theta}}(\boldsymbol{\lambda^{(2)}}| T)\right)\right\Vert _{\infty}
	\le
	\frac{L^{2}J^{2}\sqrt{2C_{\theta^{*},\Lambda}}}{m \lambda_{min}}
	\left \|\boldsymbol{\lambda}^{(1)}-\boldsymbol{\lambda}^{(2)} \right \|_2
	\end{equation}
\end{lemma}

Notice that the result assumes that the training criterion is strongly convex at its minimizer. If this is not true, one can augment the penalty function $P_j(\boldsymbol{\theta}^{(j)})$ with a ridge penalty $\| \boldsymbol{\theta}^{(j)} \|_2^2$ so that the training criterion becomes
\begin{equation}
\label{eq:param_add_models_ridge}
\frac{1}{2} \left  \| y -  g(\boldsymbol{\theta}) \right \|^2_T 
+ \sum_{j=1}^J \lambda_j \left ( P_j(\boldsymbol{\theta}^{(j)}) + \frac{w}{2} \| \boldsymbol{\theta}^{(j)} \|^2_2 \right )
\end{equation}

\subsubsection{Parametric regression with non-smooth penalties}
\label{sec:param_nonsmooth}

If the regression problem contains non-smooth penalty functions, similar results do not necessarily hold. Nonetheless we find that for many popular non-smooth penalty functions, such as the lasso \citep{tibshirani1996regression} and group lasso \citep{yuan2006model}, the fitted functions are still smoothly parameterized by $\boldsymbol \lambda$ almost everywhere. To characterize such problems, we begin with the following definitions:

\begin{definition}
	The differentiable space of function $f:\mathbb{R}^p \mapsto \mathbb{R}$ at $\boldsymbol{\theta}$ is
	\begin{equation}
	\Omega^{f}(\boldsymbol{\theta}) = \left \{ \boldsymbol{\beta} \middle | \lim_{\epsilon \rightarrow 0} \frac{f(\boldsymbol{\theta} + \epsilon \boldsymbol{\beta}) - f(\boldsymbol{\theta})}{\epsilon} \text{ exists } \right \}
	\end{equation}
\end{definition}

\begin{definition}
	Let $f(\cdot, \cdot): \mathbb{R}^p \times \mathbb{R}^J \mapsto \mathbb{R}$ be a function with a unique minimizer.
	$S \subseteq \mathbb{R}^p$ is a local optimality space of $f$ over $W \subseteq \mathbb{R}^J$ if
	\begin{equation}
	\argmin_{\boldsymbol{\theta} \in \mathbb{R}^p} f(\boldsymbol{\theta}, \boldsymbol \lambda) =
	\argmin_{\boldsymbol{\theta} \in S} f(\boldsymbol{\theta}, \boldsymbol \lambda) \quad \forall \boldsymbol \lambda \in W
	\end{equation}
\end{definition}

We are interested in the set $\Lambda_{smooth} \subseteq \Lambda$ in which the fitted functions are well-behaved. Consider $\Lambda_{smooth}$ that satisfies the following conditions:

\begin{condition}
	\label{condn:nonsmooth1}
	For every $\boldsymbol{\lambda} \in \Lambda_{smooth}$, there exists a ball $B(\boldsymbol{\lambda})$ with nonzero radius centered at $\boldsymbol{\lambda}$ such that
	\begin{itemize}
		\item For all $\boldsymbol{\lambda}'\in B(\boldsymbol{\lambda})$, the training criterion $L_{T}$ is twice differentiable at $(\hat{\boldsymbol{\theta}}(\boldsymbol{\lambda}'|T), \boldsymbol{\lambda}')$
		along directions in $\Omega^{L_{T}(\cdot,\cdot)}\left(\hat{\boldsymbol{\theta}}(\boldsymbol{\lambda} | T), \boldsymbol{\lambda}\right)$.
		\item $\Omega^{L_T(\cdot, \boldsymbol{\lambda})} \left (\hat{\boldsymbol \theta}\left(\boldsymbol{\lambda}|T \right) \right)$ is a local optimality space for $L_T\left(\cdot,\boldsymbol{\lambda}\right)$ over $B(\boldsymbol{\lambda})$.
	\end{itemize}
\end{condition}
\begin{condition}
	\label{condn:nonsmooth2}
	For every $\boldsymbol{\lambda^{(1)}},\boldsymbol{\lambda^{(2)}}\in\Lambda_{smooth}$, let the line segment between the two points be denoted 
	$$
	\mathcal{L}(\boldsymbol{\lambda^{(1)}},\boldsymbol{\lambda^{(2)}})=\left\{ \alpha\boldsymbol{\lambda^{(1)}}+(1-\alpha)\boldsymbol{\lambda^{(2)}}:\alpha\in[0,1]\right\} 
	$$
	Suppose the intersection $\mathcal{L}(\boldsymbol{\lambda^{(1)}},\boldsymbol{\lambda^{(2)}})\cap\Lambda_{smooth}^{C}$
	is countable.
\end{condition}
For example, the lasso satisfies these conditions since its solution follows a piecewise linear path \citep{efron2004least, tibshirani2011solution}. We believe other $L_p$-norm penalties, like the group lasso, also satisfy these conditions.

Equipped with these conditions, we can characterize the smoothness of the fitted functions when the penalties are non-smooth. In fact the Lipschitz constant is exactly the same as that in Lemma \ref{lemma:param_add}.

\begin{lemma}
	\label{lemma:nonsmooth}
	Define $\hat{\boldsymbol{\theta}}^{(j)}\left (\boldsymbol{\lambda} | T\right )$ as in \eqref{eq:param_add_estimator}.
	
	\noindent
	Suppose $g_j(\boldsymbol{\theta}^{(j)})$ is $L$-Lipschitz in $\boldsymbol{\theta}^{(j)}$ with respect to $\| \cdot \|_\infty$ for all $j=1,..,J$.
	
	\noindent		
	Further suppose that $P_j(\boldsymbol{\theta}^{(j)})$ and $g_j(x | \boldsymbol{\theta}^{(j)})$ are convex with respect to $\boldsymbol{\theta}^{(j)}$ for any fixed $x$ and all $j=1,..,J$.  Also suppose that $L_T\left ( \boldsymbol{\theta} , \boldsymbol{\lambda} \right )$ is convex with respect to $\boldsymbol{\theta}$.
	
	Let $U_\lambda$ be an orthonormal matrix with columns forming a basis for the differentiable space of $L_T(\cdot , \boldsymbol{\lambda})$ at $\hat{\boldsymbol{\theta}}(\boldsymbol{\lambda})$. Suppose there exists $m > 0$ such that the Hessian of the penalized training criterion at the minimizer taken with respect to the directions in $U_\lambda$ satisfies 
	\begin{equation}
	\left . _{U_\lambda}\nabla_{\theta}^2 L_T(\boldsymbol{\theta}, \boldsymbol{\lambda}) \right |_{\theta = \hat{\theta}(\boldsymbol{\lambda})} \succeq mI
	\end{equation}		
	
	Suppose $\Lambda_{smooth} \subseteq \Lambda$ satisfies Conditions \ref{condn:nonsmooth1} and \ref{condn:nonsmooth2}. Then any $\boldsymbol{\lambda}^{(1)}, \boldsymbol{\lambda}^{(2)} \in \Lambda_{smooth}$ satisfies  \eqref{eq:param_add_lipschitz}.
\end{lemma}

\subsection{Nonparametric additive models}
\label{sec:nonparam_smooth}

We now generalize the results to nonparametric additive models. We consider estimators of the form
\begin{align}
\label{eq:train_crit_nonparam}
\left\{ \hat{g}_j( \boldsymbol \lambda) \right \}_{j=1}^J &= 
\argmin_{g_j\in \mathcal{G}_j: j=1,...,J}  L_T\left (\left \{ g_j \right \}_{j=1}^J, \boldsymbol{\lambda} \right ) \\
\text{where} \quad L_T \left (\left \{ g_j \right \}_{j=1}^J, \boldsymbol{\lambda} \right ) &=
\frac{1}{2} \left \| \boldsymbol y -  \sum_{j=1}^J g_j(\boldsymbol x_j) \right \|^2_T 
+ \sum_{j=1}^J \lambda_j P_j(g_j)
\end{align}
where $P_j$ are now penalty functionals. For some problems, e.g. smoothing splines, we can show that the solution to~\eqref{eq:train_crit_nonparam} lies in a parametric family (of growing dimension); and use the results of Sections~\ref{sec:param_smooth}, and \ref{sec:param_nonsmooth} to give our Lipschitz bounds. However, this is not always possible. Nevertheless we can still obtain similar results in the non-parametric problem. The following lemma states that the fitted functions are Lipschitz with respect to $\| \cdot \|_{D^{(n)}}$. Let $\left\{ g_j^* \right \}_{j=1}^J$ be the minimizer of the generalization error
\begin{equation}
\left\{ g_j^* \right \}_{j=1}^J = \argmin_{g_j \in \mathcal{G}_j: j=1,...,J} E_T\left[ \left \| y - \sum_{j=1}^J g_j^* \right \|^2 \right]
\end{equation}

\begin{lemma}
	\label{lemma:nonparam_smooth}
	Suppose $\mathcal{G}_1, ..., \mathcal{G}_J$ be linear spaces of univariate functions.
	
	Suppose the penalty functions $P_{j}$ are twice Gateaux differentiable and convex over $\mathcal{G}_j$. Suppose there is a $m > 0$ such that for all $j=1,...,J$, the second Gateaux derivative of the training criterion at $\hat{g}^{(n_T)}_j( \boldsymbol{\lambda} | T)$ satisfies
	\begin{equation}
	\left \langle 
	\left . D^2_{g_j} L_T \left ( \left \{ g_j \right \}_{j=1}^J, \boldsymbol{\lambda} \right ) \right |_{g_j= \hat{g}_j( \boldsymbol{\lambda} | T) }
	\circ h_j, h_j
	\right \rangle 
	\ge m
	\quad \forall h_j \in \mathcal{G}_j,  \|h_j \|_{D^{(n)}} = 1
	\label{eq:gateuax}
	\end{equation}
	where $D^2_{g_j}$ is the second Gateaux derivative where both derivatives are taken in the direction of $g_j$.
	
	Let $\lambda_{\max} > \lambda_{\min} > 0 $. Let
	\begin{equation}
	C_{\Lambda}^*=
	\frac{1}{2}\left\Vert y- \sum_{j=1}^J g^*_j\right\Vert _{T}^{2}
	+\lambda_{max}\sum_{j=1}^{J} P_{j}(g^*_j)
	\end{equation}
	
	For any $\boldsymbol{\lambda}^{(1)}, \boldsymbol{\lambda}^{(2)} \in \Lambda \coloneqq \left [ \lambda_{\min}, \lambda_{\max} \right ]^J$, we have
	\begin{align}
	\label{eq:nonparam_lipshitz_thrm}
	& \left\Vert 
	\sum_{j=1}^J \hat{g}_j\left(\boldsymbol{\lambda}^{(1)} |T \right)-\hat{g}_j\left(\boldsymbol{\lambda}^{(2)} |T \right)\right\Vert _{D^{(n)}}\\
	& \le
	\frac{J}{m \lambda_{min}}\sqrt{2 C_{\Lambda}^* \frac{n}{n_{T}}\left(1+\frac{J\lambda_{max}}{\lambda_{min}}\right)}
	\left \|\boldsymbol{\lambda}^{(1)}-\boldsymbol{\lambda}^{(2)} \right \|
	\end{align}
\end{lemma}


\section{Simulations}\label{sec:simulations}

We now provide a simulation study for the prediction error bound given in Theorem \ref{thrm:train_val}. We tune the hyper-parameters by a single training/validation split and show that the error of the selected model converges to the oracle error at a near-parametric rate.

Observations were generated from the model
\begin{equation}
y = \exp(x_1) + x_2^2 + \sigma \epsilon
\end{equation}
where $\epsilon \sim N(0,1)$ and $\sigma$ scaled the error term such that the signal to noise ratio was 2.
The covariates $x_1$ and $x_2$ were uniformly distributed over the interval $(-1, 1)$.

We fit an additive smoothing spline model using the Sobolev penalty \citep{de1978practical, wahba1990spline, green1994nonparametric}. The fitted models were of the form
\begin{eqnarray*}
	\left \{ \hat{g}_j^{(n_T)}(\boldsymbol{\lambda}|T) \right \}^2_{j=1} &=& 
	\argmin_{g_1, g_2} \left \| y - g_1(x_1) - g_2(x_2) \right \|_T^2 \\
	&&+ \lambda_1 \int_{-1}^1 (g_1^{(2)}(x))^2 dx 
	+ \lambda_2 \int_{-1}^1 (g_2^{(2)}(x))^2 dx
\end{eqnarray*}
The training set contained 100 samples and models were fitted with 10 knots. A grid search was performed over the penalty parameter values $\{10^{-9 + 0.05i}: i = 0, ..., 140 \}$. We tested 36 validation set sizes $n_V = \lfloor 20 * 2^{i} \rfloor$ for equally log-spaced intervals from $i = 0$ to $i = 7$. A total of 20 simulations were run for each validation set size.

Figure \ref{fig:simulations} plots the difference of between the model loss and the oracle loss
$$
\left \| \sum_{j=1}^2 \hat{g}^{(n_T)}_j(\hat{\boldsymbol{\lambda}}|T) - g^*_j \right \|_V^2 - 
\left \| \sum_{j=1}^2 \hat{g}^{(n_T)}_j(\tilde{\boldsymbol{\lambda}} | T) - g^*_j \right \|_V^2
$$
as the validation set size increases. The difference of the validation losses drops at a rate of about $n^{-1}$. This rate is in fact faster than the bound given in Theorem \ref{thrm:train_val}; the geometric mean in the oracle inequality seems to play no role in the convergence rate. We conjecture that smoothing splines may satisfy additional regularity conditions such that the geometric mean may be discarded.

\begin{figure}
	\caption{
		Validation loss difference between oracle and selected model as validation set size grows
	}
	\centering
	\includegraphics[height=80mm]{validation_size_loss_diff_final.pdf}
	\label{fig:simulations}
\end{figure}

\section{Discussion}\label{sec:discussion}

In this paper, we considered model-estimation procedures where the hyper-parameters were tuned via split-sample validation. Our goal was to address the following open question: how does the generalization error of a model grow with the number of hyper-parameters to be estimated? To do so, we established finite-sample oracle inequalities for selection based on a single training/validation split and based on $K$-fold cross-validation. We showed that if the model-estimation procedure is smoothly parameterized by the hyper-parameters, the generalization error of the model is bounded by a combination of the oracle error, a near-parametric term, and the geometric mean of the two. In a semi- or non-parametric setting, the error incurred from tuning hyper-parameters is asymptotically negligible compared to the oracle error. In the parametric setting, tuning hyper-parameters contributes an error that is roughly on the same order as the oracle error. 

We then specialized our results to penalized regression problems with multiple penalty parameters. We showed that in many cases the fitted models are Lipschitz in the penalty parameters and thus the same relationship between the number of penalty parameters and model error holds. This result suggests that the recent efforts to combine penalty functions should push past the artificial barrier of two-way combinations and consider regularization methods with tens or even hundreds of penalty parameters.

Our results may also address the phenomenon that for many model-estimation procedures, the local minimizers seem to perform fairly well \citep{kunapuli2008bilevel}. The oracle inequalities do not need the assumption that the hyper-parameters are global minimizers of the validation loss. The only criteria necessary for the proofs to carry through is that the selected hyper-parameter vector has a smaller validation loss compared to the oracle validation loss.

Finally, an interesting direction of work would be to understand the behavior of hyper-parameter selection in the hyper-parameter space. Bounds on the distance between the selected and oracle hyper-parameters may lend a more intuitive understanding of hyper-parameter selection methods.

\section{Acknowledgments}

J.F. was supported by NIH Grant T32CA206089 and N.S. was supported by NIH Grant DP5OD019820.


\bibliographystyle{agsm}
\bibliography{multi-penalties-theory}

\end{document}
