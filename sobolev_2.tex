%% LyX 2.1.3 created this file.  For more info, see http://www.lyx.org/.
%% Do not edit unless you really know what you are doing.
\documentclass[english]{article}
\usepackage[T1]{fontenc}
\usepackage[latin9]{inputenc}
\usepackage{geometry}
\geometry{verbose,tmargin=3cm,bmargin=3cm,lmargin=3cm,rmargin=3cm}
\usepackage{amsbsy}
\usepackage{esint}
\usepackage{babel}
\begin{document}

\subsubsection*{Sobolev Take 2}

Given a function $h$, the Sobolev penalty for $h$ is

\[
P(h)=\int(h^{(r)}(x))^{2}dx
\]


Consider the class of smoothing splines 
\[
\left\{ \hat{g}(\cdot|\lambda)=\arg\min_{g\in\mathcal{G}}\frac{1}{2}\|y-g\|_{T}^{2}+\lambda P(g):\lambda\in\Lambda\right\} 
\]


Ever function $\hat{g}(\cdot|\lambda)$ is a spline that can be expressed
as the weighted sum of normalized B-splines of degree $r+1$ for a
given set of $k$ knots:
\[
\hat{g}(x|\lambda)=\sum_{i=1}^{k}\theta_{i}N_{i,r+1}(x)
\]


Note that the normalized B-splines have the property that they sum
up to one at all points within the boundary of the knots. Also recall
that B-splines are non-negative. (I might be wrong regarding the relationship
between the number of knots and the number of basis functions, but
the number of basis functions should be linear in the number of knots)

Therefore we can re-express the class of smoothing splines as a set
of function parameters
\[
\left\{ \hat{\boldsymbol{\theta}}_{\lambda}=\arg\min_{\theta}\frac{1}{2}\|y-N_{T}\boldsymbol{\theta}\|^{2}+\lambda P(\boldsymbol{\theta}):\lambda\in\Lambda\right\} 
\]
where $N_{T}$ is the normalized B-spline basis for the given set
of knots evaluated at the points in the training set. $P(\theta)$
is the Sobolev penalty and can be written as $\theta^{T}\Omega\theta$
for an appropriate penalty matrix $\Omega$. We will not need to express
anything in terms of $\Omega$ so the penalty will be just written
as $P(\theta)$.

Instead of considering the original smoothing spline problem with
the roughness penalty, we will add a ridge penalty on the function
parameters
\[
\left\{ \hat{\boldsymbol{\theta}}_{\lambda}=\arg\min_{\theta}\frac{1}{2}\|y-N_{T}\boldsymbol{\theta}\|^{2}+\lambda\left(P(\boldsymbol{\theta})+\frac{w}{2}\|\boldsymbol{\theta}\|_{2}^{2}\right):\lambda\in\Lambda\right\} 
\]



\subsubsection*{Univariate ``Sobolev'' norm}

Suppose we have the set of function parameters 
\[
\left\{ \hat{\boldsymbol{\theta}}_{\lambda}=\arg\min_{\theta}\frac{1}{2}\|y-N_{T}\boldsymbol{\theta}\|^{2}+\lambda\left(P(\boldsymbol{\theta})+\frac{w}{2}\|\boldsymbol{\theta}\|_{2}^{2}\right):\lambda\in\Lambda\right\} 
\]


Let $k$ be the number of normalized B-spline basis functions.

Suppose that we restrict the function parameters such that$\sup_{\theta\in\Theta}\|\boldsymbol{\theta}\|_{2}\le G$.
(\textbf{Is this a reasonable assumption?})

For any $\lambda^{(1)},\lambda^{(2)}\in\Lambda$ such that 
\[
\left|\lambda^{(1)}-\lambda^{(2)}\right|\le\frac{d}{2}\lambda_{min}w\left(\left(\frac{1}{\lambda_{min}}\left(2G+\|\epsilon\|_{T}\right)+wG\right)+wG\right)^{-1}
\]
we can show that 
\[
\|\hat{\boldsymbol{\theta}}_{\lambda^{(1)}}-\hat{\boldsymbol{\theta}}_{\lambda^{(2)}}\|_{2}\le d
\]


Hence 
\[
\|\hat{g}(\cdot|\lambda^{(1)})-\hat{g}(\cdot|\lambda^{(2)})\|_{\infty}=\sup_{x}\sum_{i=1}^{k}\left(\hat{\boldsymbol{\theta}}_{i,\lambda^{(1)}}-\hat{\boldsymbol{\theta}}_{i,\lambda^{(2)}}\right)N_{i}(x)\le kd
\]



\subsubsection*{Proof}

Let $\boldsymbol{\beta}=\hat{\boldsymbol{\theta}}_{\lambda^{(1)}}-\hat{\boldsymbol{\theta}}_{\lambda^{(2)}}$.
Suppose $\|\boldsymbol{\beta}\|_{2}>d$ for contradiction.

Consider the optimization problem

\[
\hat{m}_{\beta}(\lambda)=\arg\min_{m}\frac{1}{2}\|y-N_{T}(\hat{\boldsymbol{\theta}}_{\lambda^{(1)}}+m\boldsymbol{\beta})\|^{2}+\lambda\left[P\left(\hat{\boldsymbol{\theta}}_{\lambda^{(1)}}+m\boldsymbol{\beta}\right)+\frac{w}{2}\|\hat{\boldsymbol{\theta}}_{\lambda^{(1)}}+m\boldsymbol{\beta}\|_{2}^{2}\right]
\]


The KKT conditions give us
\[
-\langle N_{T}\boldsymbol{\beta},y-N_{T}(\hat{\boldsymbol{\theta}}_{\lambda^{(1)}}+\hat{m}_{\beta}(\lambda)\boldsymbol{\beta})\rangle+\lambda\left[\left.\nabla_{m}P\left(\hat{\boldsymbol{\theta}}_{\lambda^{(1)}}+m\boldsymbol{\beta}\right)\right|_{m=\hat{m}_{\beta}(\lambda)}+w\langle\boldsymbol{\beta},\hat{\boldsymbol{\theta}}_{\lambda^{(1)}}+\hat{m}_{\beta}(\lambda)\boldsymbol{\beta}\rangle\right]=0
\]


Implicit differentiation wrt $\lambda$ gives us

\begin{eqnarray*}
\frac{\partial}{\partial\lambda}\hat{m}_{\beta}(\lambda) & = & -\left(\|N_{T}\boldsymbol{\beta}\|^{2}+\lambda\left[\left.\nabla_{m}^{2}P\left(\hat{\boldsymbol{\theta}}_{\lambda^{(1)}}+m\boldsymbol{\beta}\right)\right|_{m=\hat{m}_{\beta}(\lambda)}+w\|\boldsymbol{\beta}\|_{2}^{2}\right]\right)^{-1}\\
 &  & \left(\left.\nabla_{m}P\left(\hat{\boldsymbol{\theta}}_{\lambda^{(1)}}+m\boldsymbol{\beta}\right)\right|_{m=\hat{m}_{\beta}(\lambda)}+w\langle\boldsymbol{\beta},\hat{\boldsymbol{\theta}}_{\lambda^{(1)}}+\hat{m}_{\beta}(\lambda)\boldsymbol{\beta}\rangle\right)
\end{eqnarray*}


To bound the first multiplicand, we use the convexity of the Sobolev
penalty to get 
\[
\left(\|N_{T}\boldsymbol{\beta}\|^{2}+\lambda\left[\left.\nabla_{m}^{2}P\left(\hat{\boldsymbol{\theta}}_{\lambda^{(1)}}+m\boldsymbol{\beta}\right)\right|_{m=\hat{m}_{\beta}(\lambda)}+w\|\boldsymbol{\beta}\|_{2}^{2}\right]\right)^{-1}\le\left(\lambda w\|\boldsymbol{\beta}\|_{2}^{2}\right)^{-1}
\]


To bound the second multiplicand, the KKT conditions give us
\begin{eqnarray*}
\left\Vert \left.\nabla_{m}P\left(\hat{\boldsymbol{\theta}}_{\lambda^{(1)}}+m\boldsymbol{\beta}\right)\right|_{m=\hat{m}_{\beta}(\lambda)}\right\Vert  & = & \left\Vert \frac{1}{\lambda}\langle N_{T}\boldsymbol{\beta},y-N_{T}(\hat{\boldsymbol{\theta}}_{\lambda^{(1)}}+\hat{m}_{\beta}(\lambda)\boldsymbol{\beta})\rangle-w\langle\boldsymbol{\beta},\hat{\boldsymbol{\theta}}_{\lambda^{(1)}}+\hat{m}_{\beta}(\lambda)\boldsymbol{\beta}\rangle\right\Vert \\
 & \le & \|\beta\|\left(\frac{1}{\lambda}\left\Vert N_{T}^{T}\left(y-N_{T}(\hat{\boldsymbol{\theta}}_{\lambda^{(1)}}+\hat{m}_{\beta}(\lambda)\boldsymbol{\beta})\right)\right\Vert +w\left\Vert \hat{\boldsymbol{\theta}}_{\lambda^{(1)}}+\hat{m}_{\beta}(\lambda)\boldsymbol{\beta}\right\Vert \right)
\end{eqnarray*}


We can bound the maximum eigenvalue of $N_{T}^{T}$ using the matrix
entries. Since each row of $N_{T}^{T}$ is the value of each of the
normalized B-spline basis functions at a given point, each row of
$N_{T}^{T}$ sums to one. By the Gershgorin circle theorem ({*}),
the maximum eigenvalue of a non-negative matrix is bounded by its
maximum row sum, which in the case of $N_{T}^{T}$ is 1. Hence

\begin{eqnarray*}
\left\Vert \left.\nabla_{m}P\left(\hat{\boldsymbol{\theta}}_{\lambda^{(1)}}+m\boldsymbol{\beta}\right)\right|_{m=\hat{m}_{\beta}(\lambda)}\right\Vert  & \le & \|\boldsymbol{\beta}\|\left(\frac{1}{\lambda}\left\Vert y-N_{T}(\hat{\boldsymbol{\theta}}_{\lambda^{(1)}}+\hat{m}_{\beta}(\lambda)\boldsymbol{\beta})\right\Vert +w\left\Vert \hat{\boldsymbol{\theta}}_{\lambda^{(1)}}+\hat{m}_{\beta}(\lambda)\boldsymbol{\beta}\right\Vert \right)\\
 & \le & \|\boldsymbol{\beta}\|\left(\frac{1}{\lambda_{min}}\left(2kG+\|\epsilon\|_{T}\right)+wG\right)
\end{eqnarray*}


The second inequality follows from the fact that for any function
parameters $\theta$ , we have that 
\[
\|N_{T}\boldsymbol{\theta}\|\le k\|\boldsymbol{\theta}\|\le kG
\]


since the maximum eigenvalue of $N_{T}$ is bounded by the maximum
row sum of $N_{T}$, which is at most $k$.

Then
\begin{eqnarray*}
\left\Vert \frac{\partial}{\partial\lambda}\hat{m}_{\beta}(\lambda)\right\Vert  & \le & \left(\lambda w\|\boldsymbol{\beta}\|_{2}^{2}\right)^{-1}\left(\|\boldsymbol{\beta}\|\left(\frac{1}{\lambda_{min}}\left(2kG+\|\epsilon\|_{T}\right)+wG\right)+w\|\boldsymbol{\beta}\|\|\hat{\boldsymbol{\theta}}_{\lambda^{(1)}}+\hat{m}_{\beta}(\lambda)\boldsymbol{\beta}\|\right)\\
 & \le & \left(\lambda_{min}w\|\boldsymbol{\beta}\|_{2}\right)^{-1}\left(\left(\frac{1}{\lambda_{min}}\left(2kG+\|\epsilon\|_{T}\right)+wG\right)+wG\right)
\end{eqnarray*}


By the MVT, there is some $\alpha\in(0,1)$ such that
\begin{eqnarray*}
\left|\hat{m}_{\beta}(\lambda^{(2)})-\hat{m}_{\beta}(\lambda^{(1)})\right| & = & \left|\lambda^{(2)}-\lambda^{(1)}\right|\left(\left.\frac{\partial}{\partial\lambda}\hat{m}_{\beta}(\lambda)\right|_{\lambda=\alpha\lambda^{(1)}+(1-\alpha)\lambda^{(2)}}\right)\\
 & \le & \left|\lambda^{(2)}-\lambda^{(1)}\right|\left(\lambda_{min}w\|\boldsymbol{\beta}\|_{2}\right)^{-1}\left(\left(\frac{1}{\lambda_{min}}\left(2kG+\|\epsilon\|_{T}\right)+wG\right)+wG\right)\\
 & \le & \left|\lambda^{(2)}-\lambda^{(1)}\right|\left(\lambda_{min}wd\right)^{-1}\left(\left(\frac{1}{\lambda_{min}}\left(2kG+\|\epsilon\|_{T}\right)+wG\right)+wG\right)
\end{eqnarray*}


By our choice of $\lambda^{(2)},\lambda^{(1)}$ , we have that
\[
\left|\hat{m}_{\beta}(\lambda^{(2)})-\hat{m}_{\beta}(\lambda^{(1)})\right|\le1/2
\]


However this is a contradiction since we know that $\hat{m}_{\beta}(\lambda^{(2)})=1$
and $\hat{m}_{\beta}(\lambda^{(1)})=0$.


\subsubsection*{Footnotes}

({*}) https://en.wikipedia.org/wiki/Perron\%E2\%80\%93Frobenius\_theorem\#Inequalities\_for\_Perron.E2.80.93Frobenius\_eigenvalue
\end{document}
